\documentclass[bulgarian, 12pt]{article}
\usepackage[
	a4paper,
	includeheadfoot,
	margin = 1.5 cm]
{geometry}

% Fonts
\usepackage[T2A]{fontenc}
% \usepackage{tempora}
\usepackage[utf8]{inputenc}
\usepackage{bm}

% % Languages
% \usepackage[base]{babel}
% % Some languages define these commands, so you have to write this to protect yourself from namespace collision errors
% \AfterBabelLanguage{bulgarian}{%
%   \let\sh\relax\let\ch\relax\let\tg\relax
%   \let\arctg\relax\let\arcctg\relax
%   \expandafter\let\expandafter\th\csname ltx@th\endcsname
%   \let\ctg\relax\let\cth\relax\let\cosec\relax
% }
\usepackage[bulgarian, english]{babel}
\usepackage{hyphenat}

% Indent first line in paragraph
\usepackage{indentfirst}

% Place tags on the left
\usepackage[leqno]{amsmath}

% Better math
\usepackage{amsbsy}
\usepackage{amssymb}
\usepackage{mathtools}
\usepackage{comment}
\usepackage{mathptmx}
\usepackage[makeroom]{cancel}
\usepackage{mathrsfs}

% Better theorems
\usepackage{amsthm}

% SI units
%\usepackage{siunitx}

% Derivative notations
\usepackage{physics}
%\usepackage{derivative}

\usepackage{blindtext}
\usepackage{scrextend}

\title{Изследване на ефектите на репелент срещу комари в малариен модел на Ross-Macdonald с две местообитания}
\author{Калоян Стоилов}

\begin{document}
\selectlanguage{bulgarian}
\maketitle
Имаме системата:
\begin{align*}
  &\dot{x}_1(t) = \beta_{vh} (N_1-x_1) \left(\frac{p_{11} e^{-\mu_1 \tau} a_1 (1-\kappa u_1) y_1}{p_{11} N_1 + p_{21} N_2} + \frac{p_{12} e^{-\mu_2 \tau} a_2 (1-\kappa u_1) y_2}{p_{12} N_1 + p_{22} N_2 }\right) - \gamma_1 x_1 \\
  &\dot{y}_1(t) = \beta_{hv} a_1 (M_1-y_1) \frac{p_{11} (1-\kappa u_1) x_1 + p_{21} (1-\kappa u_2) x_2}{p_{11} N_1 + p_{21} N_2} - \mu_1 y_1 \\
  &\dot{x}_2(t) = \beta_{vh} (N_2-x_2) \left(\frac{p_{21} e^{-\mu_1 \tau} a_1 (1-\kappa u_2) y_1}{p_{11} N_1 + p_{21} N_2 } + \frac{p_{22} e^{-\mu_2 \tau} a_2 (1-\kappa u_2) y_2}{p_{12} N_1 + p_{22} N_2}\right) - \gamma_2 x_2 \\
  &\dot{y}_2(t) = \beta_{hv} a_2 (M_2-y_2) \frac{p_{12} (1-\kappa u_1) x_1 +p_{22} (1-\kappa u_2) x_2}{p_{12} N_1 + p_{22} N_2} - \mu_2 y_2
\end{align*}

Първо, отбелязваме, че ако е в сила $z, z' < C_z$ и $s, s' < C_s$, то е изпълнено:
\begin{align*}
  &|(C_z - z) s - (C_z - z') s'| =
  |C_z s - z s - C_z s' + z' s' + z s' - z s'| =
  |C_z (s - s') - z (s - s') - s' (z - z')| \leq \\
  &|C_z| |s - s'| + |z| |s - s'| + |s'| |z - z'| \leq
  2 |C_z| |s - s'|  + |C_s| |z - z'| \leq
  \max\{2 |C_z|, |C_s|\} (|s-s'| + |z - z'|)
\end{align*}

Ще използваме това твърдение при дозателството на Липшицовата непрекъснатост на дясната страна. Взимаме произволни допустими двойки $(x_1, y_1, x_2, y_2), (x'_1, y'_1, x'_2, y'_2) \in \Omega$ и $(u_1, u_2), (u'_1, u'_2) \in [0, \bar{u}_1] \cross [0, \bar{u}_2]$.
Първо от неравенството на триъгълника имаме, че:
\begin{align*}
&\|F(x_1, y_1, x_2, y_2, u_1, u_2) - F(x'_1, y'_1, x'_2, y'_2, u'_1, u'_2)\| \leq \\
&|f_1(x_1, y_1, x_2, y_2, u_1, u_2) - f_1(x'_1, y'_1, x'_2, y'_2, u'_1, u'_2)| + |g_2(x_1, y_1, x_2, y_2, u_1, u_2) - g_2(x'_1, y'_1, x'_2, y'_2, u'_1, u'_2)| + \\
&|g_1(x_1, y_1, x_2, y_2, u_1, u_2) - g_1(x'_1, y'_1, x'_2, y'_2, u'_1, u'_2)| + |f_2(x_1, y_1, x_2, y_2, u_1, u_2) - f_2(x'_1, y'_1, x'_2, y'_2, u'_1, u'_2)|
\end{align*}
Сега може неколкократно да ползваме горната оценка за $f_1$:
\begin{align*}
&\bigg|\beta_{vh} (N_1-x_1) \left(\frac{a_1 p_{11} e^{-\mu_1 \tau} (1-\kappa u_1) y_1}{p_{11} N_1 + p_{21} N_2} + \frac{a_2 p_{12} e^{-\mu_2 \tau} (1-\kappa u_1) y_2}{p_{12} N_1 + p_{22} N_2 }\right) - \gamma_1 x_1 - \\
&\beta_{vh} (N_1-x'_1) \left(\frac{a_1 p_{11} e^{-\mu_1 \tau} (1-\kappa u'_1) y'_1}{p_{11} N_1 + p_{21} N_2} + \frac{a_2 p_{12} e^{-\mu_2 \tau} (1-\kappa u'_1) y'_2}{p_{12} N_1 + p_{22} N_2 }\right) + \gamma_1 x'_1\bigg| \leq \\
&\frac{\beta_{vh} a_1 p_{11} e^{-\mu_1 \tau}}{p_{11} N_1 + p_{21} N_2} \left|(N_1-x_1) [(1-\kappa u_1) y_1] - (N_1-x'_1) [(1-\kappa u'_1) y'_1]\right| + \\
&\frac{\beta_{vh} a_2 p_{12} e^{-\mu_2 \tau}}{p_{12} N_1 + p_{22} N_2} \left|(N_1-x_1) [(1-\kappa u_1) y_2] - (N_1-x'_1) [(1-\kappa u'_1) y'_2]\right| + \\
&\gamma_1 |x_1-x'_1|
\end{align*}
Имаме, че $x_1, x'_1 \leq N_1, \quad (1-\kappa u_1)y_1, (1-\kappa u_1) y'_1 \leq M_1, \quad (1-\kappa u_1)y_2, (1-\kappa u_1) y'_2 \leq M_2$:
\begin{align*}
  &\left|(N_1-x_1) [(1-\kappa u_1) y_1] - (N_1-x'_1) [(1-\kappa u'_1) y'_1]\right| \leq 2 N_1 |(1-\kappa u_1) y_1 - (1-\kappa u'_1) y'_1| + M_1 |x_1 - x'_1| \leq \\
  &2 N_1 (2|y_1 - y'_1| + M_1 \kappa |u_1 - u'_1|) + M_1 |x_1 - x'_1| \\
  &\left|(N_1-x_1) [(1-\kappa u_1) y_2] - (N_1-x'_1) [(1-\kappa u'_1) y'_2]\right| \leq 2 N_1 |(1-\kappa u_1) y_2 - (1-\kappa u'_1) y'_2| + M_2 |x_1 - x'_1| \leq \\
  &2 N_1 (2|y_2 - y'_2| + M_2 \kappa |u_1 - u'_1|) + M_2 |x_1 - x'_1|
\end{align*}
Тук също ползвахме $1-\kappa u_1, 1-\kappa u'_1 \leq 1, \quad y_1, y'_1 \leq M_1, \quad y_2, y'_2 \leq M_2$. Така получихе оценка отгоре за първото събираемо \\
Тъй като видът на $f_2$ е същият с точност до индекси, то директно получаваме и оценка за третото събираемо. \\

Сега да разгледаме за $g_1$:
\begin{align*}
  & \bigg|\beta_{hv} a_1 (M_1-y_1) \frac{p_{11} (1-\kappa u_1) x_1 + p_{21} (1-\kappa u_2) x_2}{p_{11} N_1 + p_{21} N_2} - \mu_1 y_1 - \\
  &\beta_{hv} a_1 (M_1-y'_1) \frac{p_{11} (1-\kappa u'_1) x'_1 + p_{21} (1-\kappa u'_2) x'_2}{p_{11} N_1 + p_{21} N_2} + \mu_1 y'_1\bigg| \leq \\
  & \frac{\beta_{hv} a_1 p_{11}}{p_{11} N_1 + p_{21} N_2} \left|(M_1-y_1) [(1-\kappa u_1) x_1] - (M_1-y'_1) [(1-\kappa u'_1) x'_1]\right| + \\
  &\frac{\beta_{hv} a_1 p_{21}}{p_{11} N_1 + p_{21} N_2} \left|(M_1-y_1) [(1-\kappa u_2) x_2] - (M_1-y'_1) [(1-\kappa u'_2) x'_2]\right| + \\
  &\mu_1 |y_1 - y'_1|
\end{align*}
Ограниченията са $y_1, y'_1 \leq M_1, \quad (1-\kappa u_1)x_1, (1-\kappa u'_1)x'_1 \leq N_1, \quad (1-\kappa u_2)x_2, (1-\kappa u'_2)x'_2 \leq N_2$:
\begin{align*}
  &\left|(M_1-y_1) [(1-\kappa u_1) x_1] - (M_1-y'_1) [(1-\kappa u'_1) x'_1]\right| \leq 2 M_1 |(1-\kappa u_1) x_1 - (1-\kappa u'_1) x'_1| + N_1 |y_1 - y'_1| \leq \\
  & 2 M_1 (2|x_1 - x'_1| + N_1 \kappa |u_1 - u'_1|) + N_1 |y_1 - y'_1| \\
  &\left|(M_1-y_1) [(1-\kappa u_2) x_2] - (M_1-y'_1) [(1-\kappa u'_2) x'_2]\right| \leq 2 M_1 |(1-\kappa u_2) x_2 - (1-\kappa u'_2) x'_2| + N_2 |y_1 - y'_1| \\
  & 2 M_1 (2|x_2 - x'_2| + N_2 \kappa |u_2 - u'_2|) + N_2 |y_1 - y'_1|
\end{align*}
Тук също ползвахме $1-\kappa u_1, 1-\kappa u'_1, 1-\kappa u_2, 1-\kappa u'_2 \leq 1, \quad x_1, x'_1 \leq M_1, \quad x_2, x'_2 \leq M_2$. Така получихе оценка отгоре за второто събираемо \\
Тъй като видът на $g_2$ е същият с точност до индекси, то директно получаваме и оценка за третото събираемо. \\
Тогава заместваме всичко и за цялата дясна страна е в сила:
\begin{align*}
  &\|F(x_1, y_1, x_2, y_2, u_1, u_2) - F(x'_1, y'_1, x'_2, y'_2, u'_1, u'_2)\| \leq \\
  &\frac{\beta_{vh} a_1 p_{11} e^{-\mu_1 \tau}}{p_{11} N_1 + p_{21} N_2} (2 N_1 (2|y_1 - y'_1| + M_1 \kappa |u_1 - u'_1|) + M_1 |x_1 - x'_1|) + \\
  &\frac{\beta_{vh} a_2 p_{12} e^{-\mu_2 \tau}}{p_{12} N_1 + p_{22} N_2}(2 N_1 (2|y_2 - y'_2| + M_2 \kappa |u_1 - u'_1|) + M_1 |x_1 - x'_1|) + \gamma_1 |x_1-x'_1| + \\
  &\frac{\beta_{hv} a_1 p_{11}}{p_{11} N_1 + p_{21} N_2} (2 M_1 (2|x_1 - x'_1| + N_1 \kappa |u_1 - u'_1|) + N_1 |y_1 - y'_1|) + \\
  &\frac{\beta_{hv} a_1 p_{21}}{p_{11} N_1 + p_{21} N_2} (2 M_1 (2|x_2 - x'_2| + N_2 \kappa |u_2 - u'_2|) + N_2 |y_1 - y'_1|) + \mu_1 |y_1 - y'_1| + \\
  &\frac{\beta_{vh} a_1 p_{21} e^{-\mu_1 \tau}}{p_{11} N_1 + p_{21} N_2} (2 N_1 (2|y_1 - y'_1| + M_1 \kappa |u_1 - u'_1|) + M_1 |x_1 - x'_1|) + \\
  &\frac{\beta_{vh} a_2 p_{22} e^{-\mu_2 \tau}}{p_{12} N_1 + p_{22} N_2}(2 N_1 (2|y_2 - y'_2| + M_2 \kappa |u_1 - u'_1|) + M_1 |x_1 - x'_1|) + \gamma_2 |x_2-x'_2| + \\
  &\frac{\beta_{hv} a_2 p_{12}}{p_{12} N_1 + p_{22} N_2} (2 M_2 (2|x_1 - x'_1| + N_1 \kappa |u_1 - u'_1|) + N_1 |y_2 - y'_2|) + \\
  &\frac{\beta_{hv} a_2 p_{22}}{p_{12} N_1 + p_{22} N_2} (2 M_2 (2|x_2 - x'_2| + N_2 \kappa |u_2 - u'_2|) + N_2 |y_2 - y'_2|) + \mu_2 |y_2 - y'_2| \leq
\end{align*}
\end{document}
