\section{Ядро на слаба инвариантност}
\subsection{Система на Marchaud/Peano}
За да има търсената задача решение, то трябва ядрото на допустимост $\eqref{eq:ViabilityKernel}$ да съществува и да не е празно. От \color{Red} ЦИТАТ!!!
\color{Black} за съществуването е необходимо да покажем, че $\mathscr{F}(\mathbf{z}(t))=\mathbf{F}(\mathbf{z}(t), U)$ е изображение на Marchaud/Peano.

$\mathscr{F}$ e изображение на Marchaud/Peano, ако е нетривиално, отгоре полунепрекъснато, с компактни изпъкнали образи и с линейно нарастване.

Стига параметрите ни да не са всички нулеви, то $\mathscr{F}$ е нетривиално. \\
От факта, че $\mathbf{F}$ е непрекъснато по всяка компонента (а даже и диференцируемо), то е и отгоре полунепрекъснато, откъдето и $\mathscr{F}$ е. \\
$\Omega$ е затворено и ограничено, а е крайномерно, съответно е компактно. Аналогично за $U$. Тогава и $\Omega \cross U$ е компактно. Директно от дефинициите на $\Omega$, $U$ те също така са изпъкнали, тоест $\Omega \cross U$ е изпъкнало. Вече видяхме, че $\Omega$ положително инвариантно за системата. Тогава образите на $\mathscr{F}$ ще са в $\Omega \cross U$, с други думи са компактни и изпъкнали. \\

\subsubsection{Линейно нарастване}
За да покажем линейното нарастване, да забележим, че $\mathbf{F}(\mathbf{0}) = \mathbf{0}$. Тогава може да запишем:
\begin{align*}
  &\|\mathbf{F}(x_1, y_1, x_2, y_2, u_1, u_2)\| = \|\mathbf{F}(x_1, y_1, x_2, y_2, u_1, u_2) - \mathbf{0}\| = \|\mathbf{F}(x_1, y_1, x_2, y_2, u_1, u_2) - \mathbf{F}(\mathbf{0})\| = \\
  &\frac{\beta_{vh} a_1 p_{11} e^{-\mu_1 \tau}}{p_{11} N_1 + p_{21} N_2} (2 N_1 (2 |y_1| + M_1 \kappa |u_1|) + M_1 |x_1|) + \\
  &\frac{\beta_{vh} a_2 p_{12} e^{-\mu_2 \tau}}{p_{12} N_1 + p_{22} N_2}(2 N_1 (2 |y_2| + M_2 \kappa |u_1|) + M_1 |x_1|) + \gamma_1 |x_1| + \\
  &\frac{\beta_{hv} a_1 p_{11}}{p_{11} N_1 + p_{21} N_2} (2 M_1 (2|x_1| + N_1 \kappa |u_1|) + N_1 |y_1|) + \\
  &\frac{\beta_{hv} a_1 p_{21}}{p_{11} N_1 + p_{21} N_2} (2 M_1 (2|x_2| + N_2 \kappa |u_2|) + N_2 |y_1|) + \mu_1 |y_1| + \\
  &\frac{\beta_{vh} a_1 p_{21} e^{-\mu_1 \tau}}{p_{11} N_1 + p_{21} N_2} (2 N_1 (2|y_1| + M_1 \kappa |u_1|) + M_1 |x_1|) + \\
  &\frac{\beta_{vh} a_2 p_{22} e^{-\mu_2 \tau}}{p_{12} N_1 + p_{22} N_2}(2 N_1 (2|y_2| + M_2 \kappa |u_1|) + M_1 |x_1|) + \gamma_2 |x_2| + \\
  &\frac{\beta_{hv} a_2 p_{12}}{p_{12} N_1 + p_{22} N_2} (2 M_2 (2|x_1| + N_1 \kappa |u_1|) + N_1 |y_2|) + \\
  &\frac{\beta_{hv} a_2 p_{22}}{p_{12} N_1 + p_{22} N_2} (2 M_2 (2|x_2| + N_2 \kappa |u_2|) + N_2 |y_2|) + \mu_2 |y_2| \leq \\
  &\tilde{C}_1|u_1| + \tilde{C}_2|u_2| + \tilde{C}_3 \|(x_1,y_1,x_2,y_2)\| \leq \tilde{C}_1 + \tilde{C}_2 + \tilde{C}_3 \|(x_1,y_1,x_2,y_2)\| \leq \tilde{C}(1 + \|(x_1,y_1,x_2,y_2)\|)
\end{align*}

Така $V(\bar{\mathbf{I}}, \bar{\mathbf{u}})$ съществува. Веднага може да видим, че $V(\bar{\mathbf{I}}, \bar{\mathbf{u}}) \neq \emptyset$, понеже $\mathbf{0}$ е неподвижна точка за кое да е управление и следователно $\mathbf{0} \in V(\bar{\mathbf{I}}, \bar{\mathbf{u}})$.
