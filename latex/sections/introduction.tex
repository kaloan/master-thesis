\section{Въведение}
\subsection{Малария}
Разглеждат се само женските комари в популяцията, понеже те са хапещите.
Комарите нямат имунна система и не оздравяват от маларийния плазмодий, откъдето заразени комари се отстраняват от популацията само чрез смъртност.

\subsection{SIS модел на Ross-Macdonald}

Основни факти за живота на Ronald Ross може да намерим в \cite{Bacaer2011}.
Ronald Ross е роден в Индия през 1857 г.
Израства там, след което получава медицинско образование в Англия през 1888 г., а после започва изследване на маларията.
През 1897 г. извършва експерименти върху птици.
Намирайки паразита в слюнчестите жлези на комари от рода \textit{Anopheles}, доказва, че маларията се предава чрез тяхното ухапване.
След кратко завръщане за преподаване в Англия, обикаля по много места с цел лансиране борбата срещу комарите. Идеята, че намаляването на популацията комари би могло да премахне маларията, била посрешната с недоверие.
През 1902 г. става носител на Нобеловата награда за физиология или медицина.
Понеже през младините си изучава в свободното си време математика, решава да създаде модел през 1910 г. в книгата "The Prevention of Malaria". Модела изгражда на базата на две диференциални уравнения.

% \begin{figure}[h]
%   \caption{Sir Ronald Ross, 1857-1932}
%   \centering
%   \includegraphics{ronald-ross-national-library-of-medicine.jpg}
% \end{figure}

% \begin{figure}[h]
%   \caption{Dr George Macdonald, 1903-1967}
%   \centering
%   \includegraphics{george macdonald.jpg}
% \end{figure}

\begin{figure}[h]
  \centering
  \begin{minipage}{.5\textwidth}
    \caption{Dr George Macdonald, 1903-1967}
    \centering
    \includegraphics{george macdonald.jpg}
    \label{fig:Ross}
  \end{minipage}%
  \begin{minipage}{.5\textwidth}
    \caption{Sir Ronald Ross, 1857-1932}
    \centering
    \includegraphics{ronald-ross-national-library-of-medicine.jpg}
    \label{fig:Macdonald}
  \end{minipage}
\end{figure}

Моделът прави следните допускания, които се пренасят и в неговите усложнения:
\begin{enumerate}
  \item Човешката популация и популацията комари е постоянна.
  \item Хората и комарите са разпределени равномерно в средата.
  \item Смъртността от заразата се пренебрегва, както при хората, така и при комарите.
  \item Веднъж заразени, комарите не се възстановяват.
  \item Само податливи се заразяват.
  \item Хората не придобиват никакъв имунитет.
\end{enumerate}

$X(t)$ е броя заразени с малария хора в момент $t$.
$Y(t)$ е броя заразени с малария комари в момент $t$.
$N$ е човешката популация.
$M$ е популацията от комари.
$\gamma$ е скоростта на оздравяване на хората.
$\mu$ е скоростта на смъртност на комарите.
$b$ е честотата на ухапване на комарите.
$\beta_{vh}$ е константна вероятност за заразяване на здрав човек с патогена, когато бъде ухапан от заразен комар, а $\beta_{hv}$ е константна вероятност за заразяване на здрав комар с патогена, когато ухапе заразен човек.

За интервал $\delta t$, заразените хора ще са $\beta_{vh} b Y(t) \frac{N-X(t)}{N} \delta t$, а възстановилите се заразени ще са $\gamma X(t) \delta t$, откъдето $\delta X(t) = \beta_{vh} b Y(t) \frac{N-X(t)}{X(t)} \delta t - \gamma X(t)$.
За този интервал пък заразените комари ще са $\beta{hv} b (M - Y(t)) \frac{X(t)}{N} \delta t$, а от тях ще измрат $\mu Y(t) \delta t$. След деление на $\delta t$ и граничен преход се достига до следния модел:

\begin{equation}
  \label{eq:BasicProblem}
  \begin{split}
    &\dot{x}(t) = \beta_{vh} (N-x(t)) e^{-\mu \tau} b y(t) - \gamma x(t) \\
    &\dot{y}(t) = \beta_{hv} (M-y(t)) b x(t) - \mu y(t) \\
  \end{split}
\end{equation}

Изследвайки системата, веднага забелязва че една неподвижна точка е $(0,0)$, а открива, че еднемична е:
\begin{equation*}
  E^* = (X^*, Y^*) = \left(N \frac{1 - \frac{\gamma \mu N}{b^2 \beta_{vh} \beta_{hv} M}}{1 + \frac{\gamma N}{b \beta_{vh} M}}, M \frac{1 - \frac{\gamma \mu N}{b^2 \beta_{vh} \beta_{hv} M}}{1 + \frac{\mu}{b \beta_{hv}}}\right)
\end{equation*}.
Двете точки са с неотрицателни координати и това наистина представлява епидемична неподвижна точка на системата, стига да е изпълнено:
\begin{equation}
  \label{eq:RossM}
  M > M^* = \frac{\gamma \mu N}{b^2 \beta_{vh} \beta_{hv}}
\end{equation}
Така Ross показва, че не е необходимо да бъдат изтребени всички комари, за да се изкорени маларията, а само броят им да се сведе под $M^*$.

Прави заключението, че
\begin{displayquote}
  As a matter of fact all epidemiology, concerned as it is with the variation of disease from
  time to time or from place to place, must be considered mathematically, however many
  variables are implicated, if it is to be considered scientifically at all. To say that a disease
  depends upon certain factors is not to say much, until we can also form an estimate as to how
  largely each factor influences the whole result. And the mathematical method of treatment
  is really nothing but the application of careful reasoning to the problems at issue.
\end{displayquote}
През следващата година публикува второ издание на "The Prevention of Malaria" и бива произведен в рицар. Ross почива през 1932.

В модела на Ross не се взима предвид явно факта, че не всеки заразèн комар е зарàзен, това неявно участва в $\beta_{vh}$.
$\tau$ е инкубационният период на комарите. Така математическото очакване заразèн комар да е станал зарàзен може да се изрази като $e^{-\frac{\tau}{\text{ср. продължителност на живот}}}$. Но средната продължителност на живот на комарите е точно $\frac{1}{\mu}$, откъдето $e^{-\mu\tau}Y$ е броя зарàзни комари.
Така можеше да разложим $\beta_{vh} = \tilde{\beta}_{vh} e^{-\mu\tau}$, където $\tilde{\beta}_{vh}$ е същинската вероятност за заразяване на човек от комар. Това е направено в по-нататъшните модели.

\color{Red} ДА СЕ НАПИШЕ НЕЩО ЗА Macdonald
\color{Black}

\subsection{Репродукционно число $\mathscr{R}_0$}
Оказва се, че в най-различни модели за разпространение на заразни болести има важен параметър $\mathscr{R}_0$, наричан репродукционно число, който носи смисъла на брой вторични случая на заразата, причинени от един първичен. За да може болестта да има ендемично състояние, то е необходимо $\mathscr{R}_0 > 1$, иначе броят заразени веднага щеше да намалее и съответно нямаше да има неподвижна точка, различна от $\mathbf{0}$.

За модела на Ross, репродукционното число може да бъде изведено лесно.
Човек остава заразèн (както и зарàзен) средно $frac{1}{\gamma}$ време, а пък за единица време средно получава $\beta_{hv} b \frac{M}{N}$ ухапвания от комар, които предават патогена.
Комар остава заразèн за средно $frac{1}{\mu}$ време и хапе предавайки болестта $\beta_{vh} b$ пъти.
Оттук достигаме до:
\begin{equation}
  \mathscr{R}_0 = frac{1}{\gamma} \times \beta_{hv} b \frac{M}{N} \times frac{1}{\mu} \times \beta_{vh} b = \frac{b^2 \beta{vh} \beta_{hv} M}{ \gamma \mu N}
\end{equation}
Но от тази оценка веднага получаваме, че:
\begin{equation}
  \mathscr{R}_0 > 1 \iff M > M^* = \frac{\gamma \mu N}{b^2 \beta_{vh} \beta_{hv}}
\end{equation}
Но това е точно оценката на Ross \ref{eq:RossM}.

За пресмятане на $\mathscr{R}_0$ в многомерни модели може да се подходи с метода на идното поколение на van den Driessche и Watmough \cite{Driesche2002}.
Нека имаме $\mathbf{z}$ групи от заразени. $\dot{\mathbf{z}} = \mathbf{G}{\mathbf{z}} = \mathscr{F}(\mathbf{z}) - \mathscr{V}(\mathbf{z})$, където $\mathscr{F}$ определя новите заразени, а $\mathscr{V}(\mathbf{z}) = \mathscr{V}^-(\mathbf{z}) - \mathscr{V}^+(\mathbf{z})$ е мобилността, която сме разделили на прииждащи и замиващи за съответните групи.

\begin{definition}
  $A = (a_{ij})$ е M-матрица, ако $a_{ij} \leq 0, i \neq j$ и собствените ѝ стойности имат неотрицателни реални части.
\end{definition}

\begin{theorem}
  При изпълнени следните условия:
  \begin{enumerate}
    \item $\mathbf{z} \geq \mathbf{0} \implies \mathscr{V}(\mathbf{z}) \geq 0, \mathscr{V}^+(\mathbf{z}) \geq 0, \mathscr{V}^-(\mathbf{z}) \geq 0$
    \item $z_i = 0 \implies \mathscr{V}_{i}^- = 0$
    \item $\mathscr{F}(\mathbf{0}) = \mathbf{0}, \mathscr{V}(\mathbf{0}) = \mathbf{0}$
    \item $\mathscr{F}(\mathbf{z}) = \mathbf{0} \implies$ всички собствени стойности на $D\mathbf{F}{\mathbf{0}}$ са с отрицателна реална част
  \end{enumerate}
  в сила за репродукционното число е $\mathscr{R}_0 = \rho(F V^{-1})$, където $\rho$ е спектралния радиус, а $F = D\mathscr{F}(\mathbf{0}), V = D\mathscr{V}(\mathbf{0})$, където $F \geq \mathscr{O}$, а $V$ е несингулярна M-матрица. \\
  Допълнително, $\mathbf{0}$ е локално асимптотично устойчива, ако $\mathscr{R}_0 < 1$ и неустойчива, ако $\mathscr{R}_0 > 1$.
  % \begin{equation*}
  %   D\mathscr{F}(\mathbf{0})=
  %   \begin{pmatrix}
  %     F & \mathscr{O} \\
  %     mathscr{O} & mathscr{O}
  %   \end{pmatrix}, \quad
  %   D\mathscr{F}(\mathbf{0})=
  %   \begin{pmatrix}
  %     F & \mathscr{O} \\
  %     mathscr{O} & mathscr{O}
  %   \end{pmatrix}
  % \end{equation*}
\end{theorem}

$F_{ij}$ е скоростта, с която индивид от група $j$ заразява индивиди от група $i$, а $V^{-1}_{jk}$ е средната продължителност на пребиваване на индивид от група $k$ сред индивидите от група $j$, съответно $(F V^-1)_{ik}$ са средния брой новозаразени от $i$ заради индивид от $k$.

\subsection{Модел на Ross-Macdonald с мобилност}
Разглежда се леко опростена форма на модела, предложен от Bichara \cite{Bichara2016}. Дадени са $m$ местообитания с популации на комари и $n$ популации с хора, като всяка от тях е с постоянен размер. Всяка от популациите си има своите съответни $\mu_j$ смъртности (комари) и $\gamma_i$ скорости на оздравяване (хора). Комарите се приема, че не мигрират (което е разумно предположение с оглед \color{Red} ЦИТАТ ДВИЖЕНИЕ КОМАРИ!!!
\color{Black}). Предполага се, че индивидите от всяка от популациите хора, пребивават в местообитанията на комарите за $p_{ij}$ част от времето, $\sum_{j=1}^m p_{ij} = 1$. \\
Нека с $x_i(t)$ бележим броя заразени хора, а с $y_j(t)$ - заразени комари. При направените допускания, в момент $t$, в местообитание $j$ съотношението на заразени към всички хора е:
\begin{equation}
  \frac{\sum_{i=1}^n p_{ij} x_i(t)}{\sum_{i=1}^n p_{ij} N_i}
\end{equation}
Ако $b_j$ е броят на ухапвания за човек за единица време, $a_j$ са ухапванията за комар за единица време, то като представим по два начина броя ухавпания в местообитание $j$:
\begin{equation}
  a_j M_j = b_j \sum_{i=1}^n p_{ij} N_i \iff b_j = \frac{a_j M_j}{\sum_{i=1}^n p_{ij} N_i}
\end{equation}
Mодел за разпространението на заразата е следния:
\begin{enumerate}
  \item В момент $t$ заразените хора $x_i$ се увеличават от ухапване на незаразен човек от заразени комари в различните местообитания, а намаляват пропорционално на броя си с коефициента на оздравяване. Заразяването моделираме по закона за масите, като коефициентът ще бъде $b_j$. Тогава може да се изрази $\dot{x}_i(t) = \sum_{j=1}^{m} \beta_{vh} b_j p_{ij} (N_i - x_i(t)) \frac{I_j}{M_j} - \gamma_i x_i(t)$.
  \item В момент $t$ заразените комари $y_j$ се увеличават от ухапване на заразен човек от незаразен комар в местообитание $j$, а намаляват пропорционално на броя си с коефициента на смъртност. Заразяването моделираме по закона за масите, като коефициентът ще бъде $a_j$. Достига се до $\dot{y}_j(t) = \beta_{hv} a_j (M_j - y(t)) \frac{\sum_{i=1}^n p_{ij} x_i(t)}{\sum_{i=1}^n p_{ij} N_i} - \mu_j y_j(t)$.
\end{enumerate}
След като се вземе предвид оценката за $b_j$, то системата има вида:
\begin{equation}
  \label{eq:MigrationProblem}
  \begin{split}
    &\dot{x}_i(t) = \beta_{vh} (N_i - x_i(t)) \sum_{j=1}^{m} \frac{p_{ij} a_j I_j}{\sum_{k=1}^n p_{kj} N_k} - \gamma_i x_i(t), \quad i=\overline{1, n} \\
    &\dot{y}_j(t) = \beta_{hv} a_j (M_j - y(t)) \frac{\sum_{i=1}^n p_{ij} x_i(t)}{\sum_{i=1}^n p_{ij} N_i} - \mu_j y_j(t), \quad j=\overline{1, m}
  \end{split}
  \end{equation}
В \cite{Bichara2016} с помощта на  \color{Red} ЦИТАТ СМИТ СТАТИЯ И УЧЕБНИК!!!
\color{Black} се показва, че за системата е изпълнено точно едно от:
\begin{itemize}
  \item $\mathscr{R}_0 \leq 1$ и $\mathbf{0}$ е единствената равновесна точка и е глобално асимптотично устойчива.
  \item $\mathscr{R}_0 > 1$ и $\mathbf{0}$ е неустойчива равновесна точка, като ако системата е неразложима, има единствена глобално асимптотично устойчива точка вътрешна за $\bigtimes_{i=1}^{n} [0, N_i] \times \bigtimes_{j=1}^{m} [0, M_j]$ (тоест маларията има ендемичен характер).
\end{itemize}

Тъй като $\mathscr{R}_0$ не може да бъде получено в явен вид аналитично, останалата част от статията \cite{Bichara2016} разглежда различни аналитични оценки за $\mathscr{R}_0$ и няколко симулации.

\subsection{Модел на Ross-Macdonald с използване на репелент}
Разглежда се модела от \cite{Rashkov2019}. По същността си уравненията на модела са като на $Ross-Macdonald$, но с усложнението, че може с помощта на репеленти \color{Red} ЦИТАТ РЕПЕЛЕНТИ!!!
\color{Black} да се намали честотата на ухапвания, тоест има множител $(1 - \kappa u(t))$ в закона за действие на масите, където $\kappa$ е ефективността на репелента, а пък $u(t)$ функция управление, задаващо пропорцията на хора предпазени с помощтта на репелента.
Разглежда се следния казус - възможно ли е всички заразени да бъдат хоспитализирани? Ако приемем, че има някакъв праг на заразените $\bar{I}$ и търсим такова управление $u(t)$, че $\forall t>0 (x(t) < \bar{I})$.

\begin{equation}
  \label{eq:RepelentProblem}
  \begin{split}
    &\dot{x}(t) = \beta_{vh} (N-x(t)) e^{-\mu \tau} a (1-\kappa u(t)) y(t) - \gamma x(t) \\
    &\dot{y}(t) = \beta_{hv} (M-y(t)) a (1-\kappa u(t)) x(t) - \mu y(t) \\
  \end{split}
\end{equation}

\color{Red} Да се разпише едномерния модел с репелент!!!
\color{Black} \\

\subsection{Кооперативни(квазимонотонни) системи}
Кооперативните системи са
\cite{Capasso2008}
