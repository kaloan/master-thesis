\section{Въведение}
\subsection{Малария}
\subsection{SIS модел на Ross-Macdonald}
\color{Red} Допусканията от DeLara!!!
\color{Black} \\
Разглеждат се само женските комари в популяцията, понеже те са хапещите.
Комарите нямат имунна система и не оздравяват от маларийния плазмодий, откъдето заразени комари се отстраняват от популацията само чрез смъртност.
\color{Red} Да се разкаже накрратко от A short history of mathematical population dynamics
\color{Black} \\

\subsection{Модел на Ross-Macdonald с мобилност}
Разглежда се леко опростена форма на модела, предложен от \cite{Bichara2016}. Дадени са $m$ местообитания с популации на комари и $n$ популации с хора, като всяка от тях е с постоянен размер. Всяка от популациите си има своите съответни $\mu_j$ смъртности (комари) и $\gamma_i$ скорости на оздравяване (хора). Комарите се приема, че не мигрират (което е разумно предположение с оглед \color{Red} ЦИТАТ ДВИЖЕНИЕ КОМАРИ!!!
\color{Black}). Предполага се, че индивидите от всяка от популациите хора, пребивават в местообитанията на комарите за $p_{ij}$ част от времето, $\sum_{j=1}^m p_{ij} = 1$. \\
Нека с $x_i(t)$ бележим броя заразени хора, а с $y_j(t)$ - заразени комари. При направените допускания, в момент $t$, в местообитание $j$ съотношението на заразени към всички хора е:
\begin{equation}
  \frac{\sum_{i=1}^n p_{ij} x_i(t)}{\sum_{i=1}^n p_{ij} N_i}
\end{equation}
Ако $b_j$ е броят на ухапвания за човек за единица време, $a_j$ са ухапванията за комар за единица време, то като представим по два начина броя ухавпания в местообитание $j$:
\begin{equation}
  a_j M_j = b_j \sum_{i=1}^n p_{ij} N_i \iff b_j = \frac{a_j M_j}{\sum_{i=1}^n p_{ij} N_i}
\end{equation}
Mодел за разпространението на заразата е следния:
\begin{enumerate}
  \item В момент $t$ заразените хора $x_i$ се увеличават от ухапване на незаразен човек от заразени комари в различните местообитания, а намаляват пропорционално на броя си с коефициента на оздравяване. Заразяването моделираме по закона за масите, като коефициентът ще бъде $b_j$. Тогава може да се изрази $\dot{x}_i(t) = \sum_{j=1}^{m} \beta_{vh} b_j p_{ij} (N_i - x_i(t)) \frac{I_j}{M_j} - \gamma_i x_i(t)$.
  \item В момент $t$ заразените комари $y_j$ се увеличават от ухапване на заразен човек от незаразен комар в местообитание $j$, а намаляват пропорционално на броя си с коефициента на смъртност. Заразяването моделираме по закона за масите, като коефициентът ще бъде $a_j$. Достига се до $\dot{y}_j(t) = \beta_{hv} a_j (M_j - y(t)) \frac{\sum_{i=1}^n p_{ij} x_i(t)}{\sum_{i=1}^n p_{ij} N_i} - \mu_j y_j(t)$.
\end{enumerate}
След като се вземе предвид оценката за $b_j$, то системата има вида:
\begin{align*}
  &\dot{x}_i(t) =  \beta_{vh} (N_i - x_i(t)) \sum_{j=1}^{m} \frac{p_{ij} a_j I_j}{\sum_{k=1}^n p_{kj} N_k} - \gamma_i x_i(t), \quad i=\overline{1, n} \\
  &\dot{y}_j(t) = \beta_{hv} a_j (M_j - y(t)) \frac{\sum_{i=1}^n p_{ij} x_i(t)}{\sum_{i=1}^n p_{ij} N_i} - \mu_j y_j(t), \quad j=\overline{1, m}
\end{align*}
В \cite{Bichara2016} с помощта на  \color{Red} ЦИТАТ СМИТ СТАТИЯ И УЧЕБНИК!!!
\color{Black} се показва, че за системата е изпълнено точно едно от:
\begin{itemize}
  \item $R_0 \leq 1$ и $\mathbf{0}$ е единствената равновесна точка и е глобално асимптотично устойчива.
  \item $R_0 > 1$ и $\mathbf{0}$ е неустойчива равновесна точка, като ако системата е неразложима, има единствена глобално асимптотично устойчива точка вътрешна за $\bigtimes_{i=1}^{n} [0, N_i] \times \bigtimes_{j=1}^{m} [0, M_j]$ (тоест маларията има ендемичен характер).
\end{itemize}

Тъй като $R_0$ не може да бъде получено в явен вид аналитично, останалата част от статията \cite{Bichara2016} разглежда различни аналитични оценки за $R_0$ и няколко симулации.

\subsection{Модел на Ross-Macdonald с използване на репелент}
Разглежда се модела от \cite{Rashkov2019}. По същността си уравненията на модела са като на $Ross-Macdonald$, но с усложнението, че може с помощта на репеленти \color{Red} ЦИТАТ РЕПЕЛЕНТИ!!!
\color{Black} да се намали честотата на ухапвания, тоест има множител $(1 - \kappa u(t))$ в закона за действие на масите, където $\kappa$ е ефективността на репелента, а пък $u(t)$ функция управление, задаващо пропорцията на хора предпазени с помощтта на репелента.
Разглежда се следния казус - възможно ли е всички заразени да бъдат хоспитализирани? Ако приемем, че има някакъв праг на заразените $\bar{I}$ и търсим такова управление $u(t)$, че $\forall t>0 (x(t) < \bar{I})$.
\color{Red} Да се разпише едномерния модел с репелент!!!
\color{Black} \\

\subsection{Кооперативни(квазимонотонни) системи}
Кооперативните системи са
\cite{Capasso2008}
