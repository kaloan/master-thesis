\section{Съществуване на решение и основни свойства}
Първо, отбелязваме, че ако е в сила $z, z' < C_z$ и $s, s' < C_s$, то е изпълнено:
\begin{align*}
  &|(C_z - z) s - (C_z - z') s'| =
  |C_z s - z s - C_z s' + z' s' + z s' - z s'| =
  |C_z (s - s') - z (s - s') - s' (z - z')| \leq \\
  &|C_z| |s - s'| + |z| |s - s'| + |s'| |z - z'| \leq
  2 |C_z| |s - s'|  + |C_s| |z - z'| \leq
  \max\{2 |C_z|, |C_s|\} (|s-s'| + |z - z'|)
\end{align*}

\subsection{Липшицовост по фазови променливи}
Ще използваме това твърдение при дозателството на Липшицовата непрекъснатост на дясната страна по фазовите променливи $x_1, y_1, x_2, y_2$. Взимаме произволни допустими двойки, тоест $(x_1, y_1, x_2, y_2), (x'_1, y'_1, x'_2, y'_2) \in \Omega$ и $(u_1, u_2), (u'_1, u'_2) \in [0, \bar{u}_1] \cross [0, \bar{u}_2]$.
Първо от неравенството на триъгълника имаме, че:
\begin{align*}
  &\|\mathbf{F}(x_1, y_1, x_2, y_2, u_1, u_2) - \mathbf{F}(x'_1, y'_1, x'_2, y'_2, u'_1, u'_2)\| \leq \\
  &|f_1(x_1, y_1, x_2, y_2, u_1, u_2) - f_1(x'_1, y'_1, x'_2, y'_2, u'_1, u'_2)| + |g_2(x_1, y_1, x_2, y_2, u_1, u_2) - g_2(x'_1, y'_1, x'_2, y'_2, u'_1, u'_2)| + \\
  &|g_1(x_1, y_1, x_2, y_2, u_1, u_2) - g_1(x'_1, y'_1, x'_2, y'_2, u'_1, u'_2)| + |f_2(x_1, y_1, x_2, y_2, u_1, u_2) - f_2(x'_1, y'_1, x'_2, y'_2, u'_1, u'_2)|
\end{align*}
Сега може неколкократно да ползваме горната оценка за $f_1$:
\begin{align*}
  &\bigg|\beta_{vh} (N_1-x_1) \left(\frac{a_1 p_{11} e^{-\mu_1 \tau} (1-\kappa u_1) y_1}{p_{11} N_1 + p_{21} N_2} + \frac{a_2 p_{12} e^{-\mu_2 \tau} (1-\kappa u_1) y_2}{p_{12} N_1 + p_{22} N_2 }\right) - \gamma_1 x_1 - \\
  &\beta_{vh} (N_1-x'_1) \left(\frac{a_1 p_{11} e^{-\mu_1 \tau} (1-\kappa u'_1) y'_1}{p_{11} N_1 + p_{21} N_2} + \frac{a_2 p_{12} e^{-\mu_2 \tau} (1-\kappa u'_1) y'_2}{p_{12} N_1 + p_{22} N_2 }\right) + \gamma_1 x'_1\bigg| \leq \\
  &\frac{\beta_{vh} a_1 p_{11} e^{-\mu_1 \tau}}{p_{11} N_1 + p_{21} N_2} \left|(N_1-x_1) [(1-\kappa u_1) y_1] - (N_1-x'_1) [(1-\kappa u'_1) y'_1]\right| + \\
  &\frac{\beta_{vh} a_2 p_{12} e^{-\mu_2 \tau}}{p_{12} N_1 + p_{22} N_2} \left|(N_1-x_1) [(1-\kappa u_1) y_2] - (N_1-x'_1) [(1-\kappa u'_1) y'_2]\right| + \\
  &\gamma_1 |x_1-x'_1|
\end{align*}
Имаме, че $x_1, x'_1 \leq N_1, \quad (1-\kappa u_1)y_1, (1-\kappa u_1) y'_1 \leq M_1, \quad (1-\kappa u_1)y_2, (1-\kappa u_1) y'_2 \leq M_2$:
\begin{align*}
  &\left|(N_1-x_1) [(1-\kappa u_1) y_1] - (N_1-x'_1) [(1-\kappa u'_1) y'_1]\right| \leq 2 N_1 |(1-\kappa u_1) y_1 - (1-\kappa u'_1) y'_1| + M_1 |x_1 - x'_1| \leq \\
  &2 N_1 (2|y_1 - y'_1| + M_1 \kappa |u_1 - u'_1|) + M_1 |x_1 - x'_1| \\
  &\left|(N_1-x_1) [(1-\kappa u_1) y_2] - (N_1-x'_1) [(1-\kappa u'_1) y'_2]\right| \leq 2 N_1 |(1-\kappa u_1) y_2 - (1-\kappa u'_1) y'_2| + M_2 |x_1 - x'_1| \leq \\
  &2 N_1 (2|y_2 - y'_2| + M_2 \kappa |u_1 - u'_1|) + M_2 |x_1 - x'_1|
\end{align*}
Тук също ползвахме $1-\kappa u_1, 1-\kappa u'_1 \leq 1, \quad y_1, y'_1 \leq M_1, \quad y_2, y'_2 \leq M_2$. Така получихе оценка отгоре за първото събираемо \\
Тъй като видът на $f_2$ е същият с точност до индекси, то директно получаваме и оценка за третото събираемо. \\

Сега да разгледаме за $g_1$:
\begin{align*}
  & \bigg|\beta_{hv} a_1 (M_1-y_1) \frac{p_{11} (1-\kappa u_1) x_1 + p_{21} (1-\kappa u_2) x_2}{p_{11} N_1 + p_{21} N_2} - \mu_1 y_1 - \\
  &\beta_{hv} a_1 (M_1-y'_1) \frac{p_{11} (1-\kappa u'_1) x'_1 + p_{21} (1-\kappa u'_2) x'_2}{p_{11} N_1 + p_{21} N_2} + \mu_1 y'_1\bigg| \leq \\
  & \frac{\beta_{hv} a_1 p_{11}}{p_{11} N_1 + p_{21} N_2} \left|(M_1-y_1) [(1-\kappa u_1) x_1] - (M_1-y'_1) [(1-\kappa u'_1) x'_1]\right| + \\
  &\frac{\beta_{hv} a_1 p_{21}}{p_{11} N_1 + p_{21} N_2} \left|(M_1-y_1) [(1-\kappa u_2) x_2] - (M_1-y'_1) [(1-\kappa u'_2) x'_2]\right| + \\
  &\mu_1 |y_1 - y'_1|
\end{align*}
Ограниченията са $y_1, y'_1 \leq M_1, \quad (1-\kappa u_1)x_1, (1-\kappa u'_1)x'_1 \leq N_1, \quad (1-\kappa u_2)x_2, (1-\kappa u'_2)x'_2 \leq N_2$:
\begin{align*}
  &\left|(M_1-y_1) [(1-\kappa u_1) x_1] - (M_1-y'_1) [(1-\kappa u'_1) x'_1]\right| \leq 2 M_1 |(1-\kappa u_1) x_1 - (1-\kappa u'_1) x'_1| + N_1 |y_1 - y'_1| \leq \\
  & 2 M_1 (2|x_1 - x'_1| + N_1 \kappa |u_1 - u'_1|) + N_1 |y_1 - y'_1| \\
  &\left|(M_1-y_1) [(1-\kappa u_2) x_2] - (M_1-y'_1) [(1-\kappa u'_2) x'_2]\right| \leq 2 M_1 |(1-\kappa u_2) x_2 - (1-\kappa u'_2) x'_2| + N_2 |y_1 - y'_1| \\
  & 2 M_1 (2|x_2 - x'_2| + N_2 \kappa |u_2 - u'_2|) + N_2 |y_1 - y'_1|
\end{align*}
Тук също ползвахме $1-\kappa u_1, 1-\kappa u'_1, 1-\kappa u_2, 1-\kappa u'_2 \leq 1, \quad x_1, x'_1 \leq M_1, \quad x_2, x'_2 \leq M_2$. Така получихе оценка отгоре за второто събираемо \\
Тъй като видът на $g_2$ е същият с точност до индекси, то директно получаваме и оценка за четвъртото събираемо. \\
За да проверим липшицовостта по фазовите променливи, то заместваме с $u_1 = 1'_1, u_2 = u'_2$ всичко и за цялата дясна страна е в сила:
% \begin{align*}
% &\|\mathbf{F}(x_1, y_1, x_2, y_2, u_1, u_2) - \mathbf{F}(x'_1, y'_1, x'_2, y'_2, u'_1, u'_2)\| \leq \\
% &\frac{\beta_{vh} a_1 p_{11} e^{-\mu_1 \tau}}{p_{11} N_1 + p_{21} N_2} (2 N_1 (2|y_1 - y'_1| + M_1 \kappa |u_1 - u'_1|) + M_1 |x_1 - x'_1|) + \\
% &\frac{\beta_{vh} a_2 p_{12} e^{-\mu_2 \tau}}{p_{12} N_1 + p_{22} N_2}(2 N_1 (2|y_2 - y'_2| + M_2 \kappa |u_1 - u'_1|) + M_1 |x_1 - x'_1|) + \gamma_1 |x_1-x'_1| + \\
% &\frac{\beta_{hv} a_1 p_{11}}{p_{11} N_1 + p_{21} N_2} (2 M_1 (2|x_1 - x'_1| + N_1 \kappa |u_1 - u'_1|) + N_1 |y_1 - y'_1|) + \\
% &\frac{\beta_{hv} a_1 p_{21}}{p_{11} N_1 + p_{21} N_2} (2 M_1 (2|x_2 - x'_2| + N_2 \kappa |u_2 - u'_2|) + N_2 |y_1 - y'_1|) + \mu_1 |y_1 - y'_1| + \\
% &\frac{\beta_{vh} a_1 p_{21} e^{-\mu_1 \tau}}{p_{11} N_1 + p_{21} N_2} (2 N_1 (2|y_1 - y'_1| + M_1 \kappa |u_1 - u'_1|) + M_1 |x_1 - x'_1|) + \\
% &\frac{\beta_{vh} a_2 p_{22} e^{-\mu_2 \tau}}{p_{12} N_1 + p_{22} N_2}(2 N_1 (2|y_2 - y'_2| + M_2 \kappa |u_1 - u'_1|) + M_1 |x_1 - x'_1|) + \gamma_2 |x_2-x'_2| + \\
% &\frac{\beta_{hv} a_2 p_{12}}{p_{12} N_1 + p_{22} N_2} (2 M_2 (2|x_1 - x'_1| + N_1 \kappa |u_1 - u'_1|) + N_1 |y_2 - y'_2|) + \\
% &\frac{\beta_{hv} a_2 p_{22}}{p_{12} N_1 + p_{22} N_2} (2 M_2 (2|x_2 - x'_2| + N_2 \kappa |u_2 - u'_2|) + N_2 |y_2 - y'_2|) + \mu_2 |y_2 - y'_2| \leq
% \end{align*}
\begin{align*}
  &\|\mathbf{F}(x_1, y_1, x_2, y_2, u'_1, u'_2) - \mathbf{F}(x'_1, y'_1, x'_2, y'_2, u'_1, u'_2)\| \leq \\
  &\frac{\beta_{vh} a_1 p_{11} e^{-\mu_1 \tau}}{p_{11} N_1 + p_{21} N_2} (4 N_1 |y_1 - y'_1| + M_1 |x_1 - x'_1|) +
  \frac{\beta_{vh} a_2 p_{12} e^{-\mu_2 \tau}}{p_{12} N_1 + p_{22} N_2}(4 N_1 |y_2 - y'_2| + M_1 |x_1 - x'_1|) + \gamma_1 |x_1-x'_1| + \\
  &\frac{\beta_{hv} a_1 p_{11}}{p_{11} N_1 + p_{21} N_2} (4 M_1 |x_1 - x'_1| + N_1 |y_1 - y'_1|) +
  \frac{\beta_{hv} a_1 p_{21}}{p_{11} N_1 + p_{21} N_2} (4 M_1 |x_2 - x'_2| + N_2 |y_1 - y'_1|) + \mu_1 |y_1 - y'_1| + \\
  &\frac{\beta_{vh} a_1 p_{21} e^{-\mu_1 \tau}}{p_{11} N_1 + p_{21} N_2} (4 N_1 |y_1 - y'_1| + M_1 |x_1 - x'_1|) +
  \frac{\beta_{vh} a_2 p_{22} e^{-\mu_2 \tau}}{p_{12} N_1 + p_{22} N_2}(4 N_1|y_2 - y'_2| + M_1 |x_1 - x'_1|) + \gamma_2 |x_2-x'_2| + \\
  &\frac{\beta_{hv} a_2 p_{12}}{p_{12} N_1 + p_{22} N_2} (4 M_2 |x_1 - x'_1| + N_1 |y_2 - y'_2|) +
  \frac{\beta_{hv} a_2 p_{22}}{p_{12} N_1 + p_{22} N_2} (4 M_2 |x_2 - x'_2| + N_2 |y_2 - y'_2|) + \mu_2 |y_2 - y'_2| \leq \\
  & C \|(x_1, y_1, x_2, y_2) - (x'_1, y'_1, x'_2, y'_2)\|
\end{align*}
Накрая се използват неравенства от вида $|x_1-x'_1| \leq \|(x_1, y_1, x_2, y_2) - (x'_1, y'_1, x'_2, y'_2)\|$. Тогава, спрямо общата теория на диференциалните решения с управление, съществува единствено решение на $\eqref{eq:TheProblem}$ за произволни $t$.

\subsection{Ограниченост на решението}
Да разгледаме множеството
\begin{equation}
  \Omega = \{0 \leq x_1 \leq N_1, 0 \leq y_1 \leq M_1, 0 \leq x_2 \leq N_2, 0 \leq y_2 \leq M_2\}
\end{equation}
Началното условие е някъде в това множество, тъй като популациите са неотрицателни, а заразените индивиди не са над общата популация за съответната категория. Лесно може да се види, че е в сила:
\begin{align}
  (x_1(0), y_1(0), x_2(0), y_2(0)) \in \Omega \implies \forall{t>0}\left((x_1(0), y_1(0), x_2(0), y_2(0)) \in \Omega\right)
\end{align}
Трябва да се покаже, че $\mathbf{F}$ сочи към вътрешността на $\Omega$, ако решението се намира по границата $\partial \Omega$. Но това наистина е така, от:
\begin{align*}
  &\dot{x}_1(t)\vert_{\Omega \cap \{x_1(t)=0\}} = \beta_{vh} N_1(t) \left(\frac{p_{11} e^{-\mu_1 \tau} a_1 (1-\kappa u_1(t)) y_1(t)}{p_{11} N_1 + p_{21} N_2} + \frac{p_{12} e^{-\mu_2 \tau} a_2 (1-\kappa u_1(t)) y_2(t)}{p_{12} N_1 + p_{22} N_2 }\right) \geq 0 \\
  &\dot{x}_1(t)\vert_{\Omega \cap \{x_1(t)=N_1\}} = - \gamma_1 N_1 < 0 \\
  &\dot{y}_1(t)\vert_{\Omega \cap \{y_1(t)=0\}} = \beta_{hv} a_1 M_1 \frac{p_{11} (1-\kappa u_1(t)) x_1(t) + p_{21} (1-\kappa u_2(t)) x_2(t)}{p_{11} N_1 + p_{21} N_2} \geq 0 \\
  &\dot{y}_1(t)\vert_{\Omega \cap \{y_1(t)=M_1\}} = - \mu_1 M_1 < 0 \\
  &\dot{x}_2(t)\vert_{\Omega \cap \{x_2(t)=0\}} = \beta_{vh} N_2 \left(\frac{p_{21} e^{-\mu_1 \tau} a_1 (1-\kappa u_2(t)) y_1(t)}{p_{11} N_1 + p_{21} N_2} + \frac{p_{22} e^{-\mu_2 \tau} a_2 (1-\kappa u_2(t)) y_2(t)}{p_{12} N_1 + p_{22} N_2}\right) \geq 0 \\
  &\dot{x}_2(t)\vert_{\Omega \cap \{x_2(t)=N_2\}} = - \gamma_2 N_2 < 0 \\
  &\dot{y}_2(t)\vert_{\Omega \cap \{y_2(t)=0\}} = \beta_{hv} a_2 M_2 \frac{p_{12} (1-\kappa u_1(t)) x_1(t) + p_{22} (1-\kappa u_2(t)) x_2(t)}{p_{12} N_1 + p_{22} N_2} \geq 0 \\
  &\dot{y}_2(t)\vert_{\Omega \cap \{y_2(t)=M_2\}} = - \mu_2 M_2 < 0
\end{align*}

\subsection{Кооперативност (квазимонотонност)}
Доказваме квазимонотонността по дефиницията \color{Red} ДЕФИНИЦИЯ!!!
\color{Black}
Матрицата на Якоби $\mathrm{D} \mathbf{F}(x_1, y_1, x_2, y_2)(t)$ ще ни трябав и натам, затова нека я изведем изцяло:
\begin{align*}
  &\mathrm{D} \mathbf{F}(x_1, y_1, x_2, y_2)(t) =
  \begin{pmatrix}
    \pdv{\mathbf{F}_{x_1}}{x_1} && \pdv{\mathbf{F}_{x_1}}{y_1} && \pdv{\mathbf{F}_{x_1}}{x_2} && \pdv{\mathbf{F}_{x_1}}{y_2} \\
    \pdv{\mathbf{F}_{y_1}}{x_1} && \pdv{\mathbf{F}_{y_1}}{y_1} && \pdv{\mathbf{F}_{y_1}}{x_2} && \pdv{\mathbf{F}_{y_1}}{y_2} \\
    \pdv{\mathbf{F}_{x_2}}{x_1} && \pdv{\mathbf{F}_{x_2}}{y_1} && \pdv{\mathbf{F}_{x_2}}{x_2} && \pdv{\mathbf{F}_{x_2}}{y_2} \\
    \pdv{\mathbf{F}_{y_2}}{x_1} && \pdv{\mathbf{F}_{y_2}}{y_1} && \pdv{\mathbf{F}_{y_2}}{x_2} && \pdv{\mathbf{F}_{y_2}}{y_2}
  \end{pmatrix} \\
  &\pdv{\mathbf{F}_{x_1}}{x_1} = \pdv{\dot{x}_1}{x_1} = -\beta_{vh} \left(\frac{p_{11} e^{-\mu_1 \tau} a_1 (1-\kappa u_1(t)) y_1(t)}{p_{11} N_1 + p_{21} N_2} + \frac{p_{12} e^{-\mu_2 \tau} a_2 (1-\kappa u_1(t)) y_2(t)}{p_{12} N_1 + p_{22} N_2}\right) - \gamma_1 < 0 \\
  &\pdv{\mathbf{F}_{x_1}}{y_1} = \pdv{\dot{x}_1}{y_1} = \beta_{vh} (N_1-x_1(t)) \frac{p_{11} e^{-\mu_1 \tau} a_1 (1-\kappa u_1(t))}{p_{11} N_1 + p_{21} N_2} \geq 0 \\
  &\pdv{\mathbf{F}_{x_1}}{x_2} = \pdv{\dot{x}_1}{x_2} = 0 \\
  &\pdv{\mathbf{F}_{x_1}}{y_2} = \pdv{\dot{x}_1}{y_2} = \beta_{vh} (N_1-x_1(t))\frac{p_{12} e^{-\mu_2 \tau} a_2 (1-\kappa u_1(t))}{p_{12} N_1 + p_{22} N_2} \geq 0 \\
  %
  &\pdv{\mathbf{F}_{y_1}}{x_1} = \pdv{\dot{y}_1}{x_1} = \beta_{hv} a_1 (M_1-y_1(t)) \frac{p_{11} (1-\kappa u_1(t))}{p_{11} N_1 + p_{21} N_2} \geq 0 \\
  &\pdv{\mathbf{F}_{y_1}}{y_1} = \pdv{\dot{y}_1}{y_1} = -\beta_{hv} a_1 \frac{p_{11} (1-\kappa u_1(t)) x_1(t) + p_{21} (1-\kappa u_2(t)) x_2(t)}{p_{11} N_1 + p_{21} N_2} - \mu_1 < 0 \\
  &\pdv{\mathbf{F}_{y_1}}{x_2} = \pdv{\dot{y}_1}{x_2} = \beta_{hv} a_1 (M_1-y_1(t)) \frac{p_{21} (1-\kappa u_2(t))}{p_{11} N_1 + p_{21} N_2} \geq 0 \\
  &\pdv{\mathbf{F}_{y_1}}{y_2} = \pdv{\dot{y}_1}{y_2} = 0 \\
  %
  &\pdv{\mathbf{F}_{x_2}}{x_1} = \pdv{\dot{x}_2}{x_1} = 0 \\
  &\pdv{\mathbf{F}_{x_2}}{y_1} = \pdv{\dot{x}_2}{y_1} = \beta_{vh} (N_2-x_2(t)) \frac{p_{21} e^{-\mu_1 \tau} a_1 (1-\kappa u_2(t))}{p_{11} N_1 + p_{21} N_2} \geq 0 \\
  &\pdv{\mathbf{F}_{x_2}}{x_2} = \pdv{\dot{x}_2}{x_2} = -\beta_{vh} \left(\frac{p_{21} e^{-\mu_1 \tau} a_1 (1-\kappa u_2(t)) y_1(t)}{p_{11} N_1 + p_{21} N_2} + \frac{p_{22} e^{-\mu_2 \tau} a_2 (1-\kappa u_2(t)) y_2(t)}{p_{12} N_1 + p_{22} N_2}\right) - \gamma_2 < 0 \\
  &\pdv{\mathbf{F}_{x_2}}{y_2} = \pdv{\dot{x}_2}{y_2} = \beta_{vh} (N_2-x_2(t)) \frac{p_{22} e^{-\mu_2 \tau} a_2 (1-\kappa u_2(t))}{p_{12} N_1 + p_{22} N_2} \geq 0 \\
  %
  &\pdv{\mathbf{F}_{y_2}}{x_1} = \pdv{\dot{y}_2}{x_1} = \beta_{hv} a_2 (M_2-y_2(t)) \frac{p_{12} (1-\kappa u_1(t))}{p_{12} N_1 + p_{22} N_2} \geq 0 \\
  &\pdv{\mathbf{F}_{y_12}}{y_1} = \pdv{\dot{y}_2}{y_1} = 0 \\
  &\pdv{\mathbf{F}_{y_2}}{x_2} = \pdv{\dot{y}_2}{x_2} = \beta_{hv} a_2 (M_2-y_2(t)) \frac{p_{22} (1-\kappa u_2(t))}{p_{12} N_1 + p_{22} N_2} \geq 0 \\
  &\pdv{\mathbf{F}_{y_2}}{y_2} = \pdv{\dot{y}_2}{y_2} = -\beta_{hv} a_2 \frac{p_{12} (1-\kappa u_1(t)) x_1(t) + p_{22} (1-\kappa u_2(t)) x_2(t)}{p_{12} N_1 + p_{22} N_2} - \mu_2 < 0
\end{align*}

Извън главния диагонал има само неотрицателни елементи, тогава системата е кооперативна.

% \subsection{Кооперативност (квазимонотонност)}
% Доказваме квазимонотонността по дефиницията \color{Red} ДЕФИНИЦИЯ!!!
% \color{Black}
% \begin{align*}
%   &\pdv{\dot{x}_1}{y_1} = \beta_{vh} (N_1-x_1(t)) \frac{p_{11} e^{-\mu_1 \tau} a_1 (1-\kappa u_1(t))}{p_{11} N_1 + p_{21} N_2} \geq 0 \\
%   &\pdv{\dot{x}_1}{x_2} = 0 \\
%   &\pdv{\dot{x}_1}{y_2} = \beta_{vh} (N_1-x_1(t))\frac{p_{12} e^{-\mu_2 \tau} a_2 (1-\kappa u_1(t))}{p_{12} N_1 + p_{22} N_2} \geq 0 \\
%   &\pdv{\dot{y}_1}{x_1} = \beta_{hv} a_1 (M_1-y_1(t)) \frac{p_{11} (1-\kappa u_1(t))}{p_{11} N_1 + p_{21} N_2} \geq 0 \\
%   &\pdv{\dot{y}_1}{x_2} = \beta_{hv} a_1 (M_1-y_1(t)) \frac{p_{21} (1-\kappa u_2(t))}{p_{11} N_1 + p_{21} N_2} \geq 0 \\
%   &\pdv{\dot{y}_1}{y_2} = 0 \\
%   &\pdv{\dot{x}_2}{x_1} = 0 \\
%   &\pdv{\dot{x}_2}{y_1} = \beta_{vh} (N_2-x_2(t)) \frac{p_{21} e^{-\mu_1 \tau} a_1 (1-\kappa u_2(t))}{p_{11} N_1 + p_{21} N_2} \geq 0 \\
%   &\pdv{\dot{x}_2}{y_2} = \beta_{vh} (N_2-x_2(t)) \frac{p_{22} e^{-\mu_2 \tau} a_2 (1-\kappa u_2(t))}{p_{12} N_1 + p_{22} N_2} \geq 0 \\
%   &\pdv{\dot{y}_2}{x_1} = \beta_{hv} a_2 (M_2-y_2(t)) \frac{p_{12} (1-\kappa u_1(t))}{p_{12} N_1 + p_{22} N_2} \geq 0 \\
%   &\pdv{\dot{y}_2}{y_1} = 0 \\
%   &\pdv{\dot{y}_2}{x_2} = \beta_{hv} a_2 (M_2-y_2(t)) \frac{p_{22} (1-\kappa u_2(t))}{p_{12} N_1 + p_{22} N_2} \geq 0 \\
% \end{align*}

\subsection{Неразложимост}
Използваме теорема 3.2.1 от \cite{Brualdi1991}, която гласи:
\begin{theorem}
  \label{theorem:ConnectedIrreducability}
  Матрица $A=(a_{ij})$ е неразложима точно когато ориентираният граф $G=(V,E)$, с върхове $V=\{1, \cdots, n\}$ и ребра $E=\{(i,j) \vert a_{ij} \neq 0 \}$, е силно свързан.
\end{theorem}

\begin{proposition}
  Якобианът на системата $\ref{eq:TheProblem}$ е неразложим.
\end{proposition}
\begin{proof}
  Заместваме ненулевите елементи на $\mathrm{D} \mathbf{F}$ с 1. Така получаваме графа с матрица на съседство $\mathrm{D}$:
  \begin{equation}
    \mathrm{D} =
    \begin{pmatrix}
      1 && 1 && 0 && 1 \\
      1 && 1 && 1 && 0 \\
      0 && 1 && 1 && 1 \\
      1 && 0 && 1 && 1 \\
    \end{pmatrix}
    \implies
    \mathrm{D}^3 =
    \begin{pmatrix}
      7 && 7 && 6 && 7 \\
      7 && 7 && 7 && 6 \\
      6 && 7 && 7 && 7 \\
      7 && 6 && 7 && 7 \\
    \end{pmatrix}
    >
    \mathscr{O}
  \end{equation}
  Тъй графа има 4 върха, с матрицата на съседство повдигната на 3-та степен виждаме кои върхове са свързани помежду си и кои не (тъй като ще получим информация за свързаните компоненти, от факта че всеки прост път е с дължина не по-голяма от 3).
  Понеже има единствена свързана компонента, то графът е силно свързан.
  Спрямо $\ref{theorem:ConnectedIrreducability}$ откъдето $\mathrm{D}\mathbf{F}$ е неразложима.
  \end{proof}

\subsection{Силна вдлъбнатост}
\begin{proposition}
  Системата е силно вдлъбната, т.е. $\mathbf{0} < \mathbf{z}_1 < \mathbf{z}_2 \implies \mathrm{D}\mathbf{F}(\mathbf{z}_2) < \mathrm{D}\mathbf{F}(\mathbf{z}_1)$
\end{proposition}
\begin{proof}

  Достатъчно условие за това е всяка компонента на $\mathrm{D}\mathbf{F}$ да е нерастяща функция по всички променливи, като за поне една от тях да е намаляваща. Това може да проверим с производни по различните променливи:
{\allowdisplaybreaks
  \begin{align*}
    &\pdv{\mathbf{F}_{x_1}}{x_1}{x_1} = 0, \quad
    \pdv{\mathbf{F}_{x_1}}{x_1}{y_1} = -\beta_{vh} \frac{p_{11} e^{-\mu_1 \tau} a_1 (1-\kappa u_1(t))}{p_{11} N_1 + p_{21} N_2} < 0, \\
    &\pdv{\mathbf{F}_{x_1}}{x_1}{x_2} = 0, \quad
    \pdv{\mathbf{F}_{x_1}}{x_1}{y_2} = -\beta_{vh} \frac{p_{12} e^{-\mu_2 \tau} a_2 (1-\kappa u_1(t))}{p_{12} N_1 + p_{22} N_2} < 0 \\
    %
    &\pdv{\mathbf{F}_{x_1}}{y_1}{x_1} = -\beta_{vh} \frac{p_{11} e^{-\mu_1 \tau} a_1 (1-\kappa u_1(t))}{p_{11} N_1 + p_{21} N_2}  < 0, \quad
    \pdv{\mathbf{F}_{x_1}}{y_1}{y_1} = 0, \\
    &\pdv{\mathbf{F}_{x_1}}{y_1}{x_2} = 0, \quad
    \pdv{\mathbf{F}_{x_1}}{y_1}{y_2} = 0 \\
    %
    &\pdv{\mathbf{F}_{x_1}}{x_2}{x_1} = 0, \quad
    \pdv{\mathbf{F}_{x_1}}{x_2}{y_1} = 0, \\
    &\pdv{\mathbf{F}_{x_1}}{x_2}{x_2} = 0, \quad
    \pdv{\mathbf{F}_{x_1}}{x_2}{y_2} = 0 \\
    %
    &\pdv{\mathbf{F}_{x_1}}{y_2}{x_1} = -\beta_{vh} \frac{p_{12} e^{-\mu_2 \tau} a_2 (1-\kappa u_1(t))}{p_{12} N_1 + p_{22} N_2} < 0, \quad
    \pdv{\mathbf{F}_{x_1}}{y_2}{y_1} = 0, \\
    &\pdv{\mathbf{F}_{x_1}}{y_2}{x_2} = 0, \quad
    \pdv{\mathbf{F}_{x_1}}{y_2}{y_2} = 0 \\
    %
    %
    %
    &\pdv{\mathbf{F}_{y_1}}{x_1}{x_1} = 0, \quad
    \pdv{\mathbf{F}_{y_1}}{x_1}{y_1} = -\beta_{hv} a_1 \frac{p_{11} (1-\kappa u_1(t))}{p_{11} N_1 + p_{21} N_2} < 0, \\
    &\pdv{\mathbf{F}_{y_1}}{x_1}{x_2} = 0, \quad
    \pdv{\mathbf{F}_{y_1}}{x_1}{y_2} = 0 \\
    %
    &\pdv{\mathbf{F}_{y_1}}{y_1}{x_1} = -\beta_{hv} a_1 \frac{p_{11} (1-\kappa u_1(t))}{p_{11} N_1 + p_{21} N_2} < 0, \quad
    \pdv{\mathbf{F}_{y_1}}{y_1}{y_1} = 0, \\
    &\pdv{\mathbf{F}_{y_1}}{y_1}{x_2} = -\beta_{hv} a_1 \frac{p_{21} (1-\kappa u_2(t)) x_2(t)}{p_{11} N_1 + p_{21} N_2}, \quad
    \pdv{\mathbf{F}_{y_1}}{y_1}{y_2} = 0 \\
    %
    &\pdv{\mathbf{F}_{y_1}}{x_2}{x_1} = 0, \quad
    \pdv{\mathbf{F}_{y_1}}{x_2}{y_1} = -\beta_{hv} a_1 \frac{p_{21} (1-\kappa u_2(t))}{p_{11} N_1 + p_{21} N_2} < 0, \\
    &\pdv{\mathbf{F}_{y_1}}{x_2}{x_2} = 0, \quad
    \pdv{\mathbf{F}_{y_1}}{x_2}{y_2} = 0 \\
    %
    &\pdv{\mathbf{F}_{y_1}}{y_2}{x_1} = 0, \quad
    \pdv{\mathbf{F}_{y_1}}{y_2}{y_1} = 0, \\
    &\pdv{\mathbf{F}_{y_1}}{y_2}{x_2} = 0, \quad
    \pdv{\mathbf{F}_{y_1}}{y_2}{y_2} = 0 \\
    %
    %
    %
    &\pdv{\mathbf{F}_{x_2}}{x_1}{x_1} = 0, \quad
    \pdv{\mathbf{F}_{x_2}}{x_1}{y_1} = 0, \\
    &\pdv{\mathbf{F}_{x_2}}{x_1}{x_2} = 0, \quad
    \pdv{\mathbf{F}_{x_2}}{x_1}{y_2} = 0 \\
    %
    &\pdv{\mathbf{F}_{x_2}}{y_1}{x_1} = 0, \quad
    \pdv{\mathbf{F}_{x_2}}{y_1}{y_1} = 0, \\
    &\pdv{\mathbf{F}_{x_2}}{y_1}{x_2} = -\beta_{vh} \frac{p_{21} e^{-\mu_1 \tau} a_1 (1-\kappa u_2(t))}{p_{11} N_1 + p_{21} N_2} < 0, \quad
    \pdv{\mathbf{F}_{x_2}}{y_1}{y_2} = 0 \\
    %
    &\pdv{\mathbf{F}_{x_2}}{x_2}{x_1} = 0, \quad
    \pdv{\mathbf{F}_{x_2}}{x_2}{y_1} = -\beta_{vh} \frac{p_{21} e^{-\mu_1 \tau} a_1 (1-\kappa u_2(t))}{p_{11} N_1 + p_{21} N_2} < 0, \\
    &\pdv{\mathbf{F}_{x_2}}{x_2}{x_2} = 0, \quad
    \pdv{\mathbf{F}_{x_2}}{x_2}{y_2} = -\beta_{vh} \frac{p_{22} e^{-\mu_2 \tau} a_2 (1-\kappa u_2(t))}{p_{12} N_1 + p_{22} N_2} < 0 \\
    %
    &\pdv{\mathbf{F}_{x_2}}{y_2}{x_1} = 0, \quad
    \pdv{\mathbf{F}_{x_2}}{y_2}{y_1} = 0, \\
    &\pdv{\mathbf{F}_{x_2}}{y_2}{x_2} = -\beta_{vh} \frac{p_{22} e^{-\mu_2 \tau} a_2 (1-\kappa u_2(t))}{p_{12} N_1 + p_{22} N_2} < 0, \quad
    \pdv{\mathbf{F}_{x_2}}{y_2}{y_2} = 0 \\
    %
    %
    %
    &\pdv{\mathbf{F}_{y_2}}{x_1}{x_1} = 0, \quad
    \pdv{\mathbf{F}_{y_2}}{x_1}{y_1} = 0, \\
    &\pdv{\mathbf{F}_{y_2}}{x_1}{x_2} = 0, \quad
    \pdv{\mathbf{F}_{y_2}}{x_1}{y_2} = -\beta_{hv} a_2 \frac{p_{12} (1-\kappa u_1(t))}{p_{12} N_1 + p_{22} N_2} < 0 \\
    %
    &\pdv{\mathbf{F}_{y_2}}{y_1}{x_1} = 0, \quad
    \pdv{\mathbf{F}_{y_2}}{y_1}{y_1} = 0, \\
    &\pdv{\mathbf{F}_{y_2}}{y_1}{x_2} = 0, \quad
    \pdv{\mathbf{F}_{y_2}}{y_1}{y_2} = 0 \\
    %
    &\pdv{\mathbf{F}_{y_2}}{x_2}{x_1} = 0, \quad
    \pdv{\mathbf{F}_{y_2}}{x_2}{y_1} = 0, \\
    &\pdv{\mathbf{F}_{y_2}}{x_2}{x_2} = 0, \quad
    \pdv{\mathbf{F}_{y_2}}{x_2}{y_2} = -\beta_{hv} a_2 \frac{p_{22} (1-\kappa u_2(t))}{p_{12} N_1 + p_{22} N_2} < 0 \\
    %
    &\pdv{\mathbf{F}_{y_2}}{y_2}{x_1} = -\beta_{hv} a_2 \frac{p_{12} (1-\kappa u_1(t))}{p_{12} N_1 + p_{22} N_2} < 0, \quad
    \pdv{\mathbf{F}_{y_2}}{y_2}{y_1} = 0, \\
    &\pdv{\mathbf{F}_{y_2}}{y_2}{x_2} = -\beta_{hv} a_2 \frac{p_{22} (1-\kappa u_2(t))}{p_{12} N_1 + p_{22} N_2} < 0, \quad
    \pdv{\mathbf{F}_{y_2}}{y_2}{y_2} = 0
  \end{align*}
}
Така достатъчното условие е изпълнено и системата притежава силна вдлъбнатост.
\end{proof}


\subsection{Неподвижни точки}
Системата е силно нелинейна и с голяма размерност, откъдето не е възможно да бъдат изведени аналитични изрази за равновесните точки, различни от тривиалната ($\mathbf{0}$). С помощта на теорията на кооперативните системи може да получим техния брой.

Да разгледаме константно управление $\mathbf{u}(t)=\mathbf{u}=const$. Веднага се вижда, че $\mathbf{F}(\mathbf{0}, \mathbf{u}) = \mathbf{0}$, т.е. $\mathbf{0}$ е тривиалната неподвижна точка на системата.
Сега разглеждаме за фиксирано управление $\mathbf{u}(t) \equiv \mathbf{u}$.

Ще използваме теоремата на Smith за автономни системи. Видя се, че системата е кооперативна, с неразложима матрица на Якоби и е силно вдлъбната. Тогава имаме всички условия от
\color{Red} ЦИТАТ Smith!!!
\color{Black}
.
При $R_0(\mathbf{u}) \leq 1$, $\mathbf{0}$ е единствена устойчива неподвижна точка, а при $R_0(\mathbf{u}) > 1$, то $\mathbf{0}$ е неустойчива неподвижна точка и съществува точно една друга устойчива, намираща се във вътрешността на $\Omega$.

Във случая $\mathbf{u}(t)=\bar{\mathbf{u}}$, то ако имаме ендемична точка $\mathbf{E}^*$ и $E_1^* > \bar{I}_1 \lor E_3^* > \bar{I}_2$, то търсената от нас задача няма решение. Наистина, всяка друга система $\mathbf{u}(t)$ мяжорира тази, а тук поне от някъде нататък траекторията злиза извън желаното множество. Но тогава и за всяка друга система траекторията ще излезе от него, тоест не можем да намерим каквото и да било управление, за което във всеки момент заразените и в двете области хора да са под желаните прагове.

Обратната посока не е ясна. В случая $\mathbf{u}(t)=\mathbf{0}$, то ако липсва ендемична точка, решението ще клони към $\mathbf{0}$, но не е ясно дали винаги се намира в желаното множество, или по някакъв начин се нгъва и клони излизайки от него.

% Да изведем матрицата на Якоби $\mathrm{D} \mathbf{F}(x_1, y_1, x_2, y_2)(t)$:
% \begin{align*}
%   &\mathrm{D} \mathbf{F}(x_1, y_1, x_2, y_2)(t) = \\
%   &
%   \begin{tiny}
%     \begin{pmatrix}
%       - \beta_{vh} \left(\frac{p_{11} e^{-\mu_1 \tau} a_1 (1-\kappa u_1) y_1(t)}{p_{11} N_1 + p_{21} N_2} + \frac{p_{12} e^{-\mu_2 \tau} a_2 (1-\kappa u_1) y_2(t)}{p_{12} N_1 + p_{22} N_2}\right) - \gamma_1 && \beta_{vh} (N_1-x_1(t)) \frac{p_{11} e^{-\mu_1 \tau} a_1 (1-\kappa u_1)}{p_{11} N_1 + p_{21} N_2} && 0 && \beta_{vh} (N_1-x_1(t)) \frac{p_{12} e^{-\mu_2 \tau} a_2 (1-\kappa u_1)}{p_{12} N_1 + p_{22} N_2} \\
%       \beta_{hv} a_1 (M_1-y_1(t)) \frac{p_{11} (1-\kappa u_1)}{p_{11} N_1 + p_{21} N_2} && -\beta_{hv} a_1 \frac{p_{11} (1-\kappa u_1) x_1(t) + p_{21} (1-\kappa u_2) x_2(t)}{p_{11} N_1 + p_{21} N_2} - \mu_1 && \beta_{hv} a_1 (M_1-y_1(t)) \frac{p_{21} (1-\kappa u_2)}{p_{11} N_1 + p_{21} N_2}&& 0 \\
%       0 && \beta_{vh} (N_2-x_2(t)) \frac{p_{21} e^{-\mu_1 \tau} a_1 (1-\kappa u_2)}{p_{11} N_1 + p_{21} N_2} && -\beta_{vh} \left(\frac{p_{21} e^{-\mu_1 \tau} a_1 (1-\kappa u_2) y_1(t)}{p_{11} N_1 + p_{21} N_2} + \frac{p_{22} e^{-\mu_2 \tau} a_2 (1-\kappa u_2) y_2(t)}{p_{12} N_1 + p_{22} N_2}\right) - \gamma_2 && \beta_{vh} (N_2-x_2(t)) \frac{p_{22} e^{-\mu_2 \tau} a_2 (1-\kappa u_2) y_2(t)}{p_{12} N_1 + p_{22} N_2} \\
%       \beta_{hv} a_2 (M_2-y_2(t)) \frac{p_{12} (1-\kappa u_1)}{p_{12} N_1 + p_{22} N_2} && 0 && \beta_{hv} a_2 (M_2-y_2(t)) \frac{p_{22} (1-\kappa u_2) x_2(t)}{p_{12} N_1 + p_{22} N_2} && -\beta_{hv} a_2 \frac{p_{12} (1-\kappa u_1) x_1(t) + p_{22} (1-\kappa u_2) x_2(t)}{p_{12} N_1 + p_{22} N_2} - \mu_2 \\
%     \end{pmatrix}
%   \end{tiny}
% \end{align*}

% % Да разгледаме $A_{ij}=\frac{a_j b_j p_ij e^{-\mu_j \tau_j}}{H_j}$, $B_{ij}=\frac{a_j c_j \delta_{ij}}{H_j}$
% \color{Red} ДА СЕ ОПРАВЯТ ЗНАМЕНАТЕЛИТЕ!!!
% \color{Black}
% Да означим $A_{ij}=\frac{a_j b_j p_ij e^{-\mu_j \tau_j}}{H_j}$, $B_{ij}=\frac{a_j c_j p_ji}{H_j}$. Тогава системата ни добива точно вида $(2.4)$ от \color{Red} ЦИТАТ Cosner, обаче с оправените Bichara?!!!
% \color{Black}
% и ползвайки Теорема 1 от същата статия с $\cap{A}=(\frac{A_{ij} N_i}{\gamma_i})$, то ако зададем $R_0^2=\rho(\cap{А}\cap{B})$,
При $R_0 \leq 1$, $\mathbf{0}$ е единствена устойчива неподвижна точка, а при $R_0 > 1$, то $\mathbf{0}$ е неустойчива неподвижна точка и съществува точно една друга устойчива, намираща се във вътрешността на $\Omega$.

Ендемичната точка (когато съществува) може да бъде намерена приблизително по два начина. П
ървият е да се пусне числена симулация на системата и когато решението вече не се мени значително, то знаем, че сме в околност на епидемичната точка, откъдето може да я преближим с решението в съответния момент.
Другият начин е да се реши числено нелинейната система, получена когато се занулят левите страни на $\ref{eq:TheProblem}$. Полученото решение ще е равновесна точка (но може да получим и $\mathbf{0}$). Варирайки първоначалното приближение, ще получим и приближение на ендемичната точка.
