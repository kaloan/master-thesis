\section{Съществуване на решение и основни свойства}
\begin{lemma}
  \label{lemma:Modulus}
  Нека $z, z', s, s', C_z, C_s \in \mathbb{R}$, за които $z, z' < C_z$ и $s, s' < C_s$. Тогава след полагането $C = \max\{2 |C_z|, |C_s|\}$ е в сила $|(C_z - z) s - (C_z - z') s'| \leq C (|s-s'| + |z - z'|)$.
\end{lemma}

\begin{proof}
  \begin{equation}
    \begin{split}
      &|(C_z - z) s - (C_z - z') s'| =
      |C_z s - z s - C_z s' + z' s' + z s' - z s'| =
      |C_z (s - s') - z (s - s') - s' (z - z')| \leq \\
      &|C_z| |s - s'| + |z| |s - s'| + |s'| |z - z'| \leq
      2 |C_z| |s - s'|  + |C_s| |z - z'| \leq
      \max\{2 |C_z|, |C_s|\} (|s-s'| + |z - z'|)
    \end{split}
    \end{equation}
    \end{proof}
\subsection{Съществуване на решение}
\begin{proposition}
  Системата \ref{eq:TheDimensionlessProblem} има единствено решение.
\end{proposition}

\begin{proof}
  Трябва да покажем, че $\mathbf{f}$ е липшицова по фазовите применливи за всички възможни управления. Нека първо $\mathbf{z}, \mathbf{z}' \in \Omega, \mathbf{u} \in U$.
  Първо от неравенството на триъгълника имаме, че:
  \begin{equation}
    \|\mathbf{f}(\mathbf{z}, \mathbf{u}) - \mathbf{f}(\mathbf{z}', \mathbf{u}')\| \leq
    |F_{x_1}(\mathbf{z}, \mathbf{u})| + |F_{x_2}(\mathbf{z}, \mathbf{u})| + |F_{y_1}(\mathbf{z}, \mathbf{u})| + |F_{y_2}(\mathbf{z}, \mathbf{u})|
  \end{equation}
  Сега може да ползваме неколкократно \ref{lemma:Modulus} за $F_{x_1}$:
  \begin{multline}
    |(1-x_1) (1-\kappa u_1) \left(b_{11} y_1 + b_{12} y_2\right) - \gamma_1 x_1 - (1-x'_1) (1-\kappa u'_1) \left(b_{11} y'_1 + b_{12} y'_2\right) - \gamma_1 x'_1| \leq \\
    b_{11} \left|(1-x_1)[(1 - \kappa u_1) y_1] -  (1-x'_1)[(1 - \kappa u'_1) y'_1]\right| + \\
    b_{12} \left|(1-x_1)[(1 - \kappa u_1) y_2] -  (1-x'_1)[(1 - \kappa u'_2) y'_2]\right| + \gamma |x_1 - x'_1|
  \end{multline}

  Имаме, че $x_1, x'_1 \leq 1, \quad (1-\kappa u_1)y_1, (1-\kappa u_1) y'_1 \leq 1, \quad (1-\kappa u_1)y_2, (1-\kappa u_1) y'_2 \leq 1$:
  \begin{multline}
    \left|(1-x_1) [(1-\kappa u_1) y_1] - (1-x'_1) [(1-\kappa u'_1) y'_1]\right| \leq \\
    2 |(1-\kappa u_1) y_1 - (1-\kappa u'_1) y'_1| + |x_1 - x'_1| \leq \\
    2 (2|y_1 - y'_1| + \kappa |u_1 - u'_1|) + |x_1 - x'_1|
  \end{multline}
  \begin{multline}
    \left|(1-x_1) [(1-\kappa u_1) y_2] - (1-x'_1) [(1-\kappa u'_1) y'_2]\right| \leq \\
    2 |(1-\kappa u_1) y_2 - (1-\kappa u'_1) y'_2| + |x_1 - x'_1| \leq \\
    2 (2|y_2 - y'_2| + \kappa |u_1 - u'_1|) + |x_1 - x'_1|
  \end{multline}

  Тук също ползвахме $1-\kappa u_1, 1-\kappa u'_1 \leq 1, \quad y_1, y'_1 \leq 1, \quad y_2, y'_2 \leq 1$. Така получихе оценка отгоре за първото събираемо \\
  Тъй като видът на $F_{x_2}$ е същият с точност до индекси, то директно получаваме и оценка за второто събираемо.

  Сега да разгледаме $F_{y_1}$:
  \begin{multline}
    |(1-y_1) \left(c_{11}(1-\kappa u_1) x_1 + c_{12}(1-\kappa u_2) x_2\right) - \mu_1 y_1 - (1-y'_1) \left(c_{11}(1-\kappa u'_1) x'_1 + c_{12}(1-\kappa u'_2) x'_2\right) - \mu_1 y'_1| \leq \\
    c_{11} \left|(1-y_1)[(1-\kappa u_1) x_1] -  (1-y'_1)[(1 - \kappa u'_1) x'_1]\right| + \\
    c_{12} \left|(1-y_1)[(1-\kappa u_2) x_2] -  (1-y'_1)[(1 - \kappa u'_2) x'_2]\right| + \mu |y_1 - y'_1|
  \end{multline}
  Ограниченията са $y_1, y'_1 \leq 1, \quad (1-\kappa u_1)x_1, (1-\kappa u'_1)x'_1 \leq 1, \quad (1-\kappa u_2)x_2, (1-\kappa u'_2)x'_2 \leq 1$:
  \begin{multline}
    \left|(1-y_1)[(1-\kappa u_1) x_1] -  (1-y'_1)[(1-\kappa u'_1) x'_1]\right| \leq \\
    2 |(1-\kappa u_1) x_1 - (1-\kappa u'_1) x'_1| + |y_1 - y'_1| \leq \\
    2 (2|x_1 - x'_1| + \kappa |u_1 - u'_1|) + |y_1 - y'_1|
  \end{multline}
  \begin{multline}
    \left|(1-y_1)[(1 -\kappa u_2) x_2] -  (1-y'_1)[(1 -\kappa u'_2) x'_2]\right| \leq \\
    2 |(1-\kappa u_2) x_2 - (1-\kappa u'_2) x'_2| + |y_1 - y'_1| \leq \\
    2 (2|x_2 - x'_2| + \kappa |u_2 - u'_2|) + |y_1 - y'_1|
  \end{multline}
  Тук също ползвахме $1-\kappa u_1, 1-\kappa u'_1, 1-\kappa u_2, 1-\kappa u'_2 \leq 1, \quad x_1, x'_1 \leq M_1, \quad x_2, x'_2 \leq M_2$. Така получихе оценка отгоре за второто събираемо \\
  Тъй като видът на $F_{y_2}$ е същият с точност до индекси, то директно получаваме и оценка за четвъртото събираемо. \\

  За да проверим липшицовостта по фазовите променливи, то заместваме с $u_1 = 1'_1, u_2 = u'_2$ всичко и за цялата дясна страна е в сила:
  \begin{multline}
    \|\mathbf{f}(\mathbf{z}, \mathbf{u}) - \mathbf{f}(\mathbf{z}', \mathbf{u}')\| \leq \\
    b_{11} (4 |y_1 - y'_1| + |x_1 - x'_1|) +
    b_{12} (4 |y_2 - y'_2| + |x_1 - x'_1|) + \gamma_1 |x_1-x'_1| + \\
    b_{21} (4 |y_1 - y'_1| + |x_1 - x'_1|) +
    b_{22} (4 |y_2 - y'_2| + |x_1 - x'_1|) + \gamma_2 |x_2-x'_2| + \\
    c_{11} (4 |x_1 - x'_1| + |y_1 - y'_1|) +
    c_{22} (4 |x_2 - x'_2| + |y_1 - y'_1|) + \mu_1 |y_1 - y'_1| + \\
    c_{21} (4 |x_1 - x'_1| + |y_2 - y'_2|) +
    c_{22} (4 |x_2 - x'_2| + |y_2 - y'_2|) + \mu_2 |y_2 - y'_2| \leq \\
    C \|\mathbf{z} - \mathbf{z}'\|
  \end{multline}
  Накрая се използват неравенства от вида $|x_1-x'_1| \leq \|(x_1, x_2, y_1, y_2) - (x'_1, x'_2, y'_1, y'_2)\| = \|\mathbf{z} - \mathbf{z}'\|$.
  Тогава, спрямо общата теория на диференциалните решения с управление, съществува единствено решение на $\eqref{eq:TheProblem}$ за произволни $t>0$.
  \end{proof}

\subsection{Ограниченост на решението}
Началното условие е някъде в $\Omega$, тъй като популациите са неотрицателни, а заразените индивиди не са над общата популация за съответната категория. Лесно може да се види, че е в сила:
\begin{align}
  (x_1(0), x_2(0), y_1(0), y_2(0))^T \in \Omega \implies \forall{t>0}\left((x_1(t), x_2(t), y_1(t), y_2(t))^T \in \Omega\right)
\end{align}
Трябва да се покаже, че $\mathbf{f}$ сочи към вътрешността на $\Omega$, ако решението се намира по границата $\partial \Omega$. Но това наистина е така, от:
% \begin{align*}
% &\dot{x}_1(t)\vert_{\Omega \cap \{x_1(t)=0\}} = \beta_{vh} N_1(t) \left(\frac{p_{11} e^{-\mu_1 \tau} a_1 (1-\kappa u_1(t)) y_1(t)}{p_{11} N_1 + p_{21} N_2} + \frac{p_{12} e^{-\mu_2 \tau} a_2 (1-\kappa u_1(t)) y_2(t)}{p_{12} N_1 + p_{22} N_2 }\right) \geq 0 \\
% &\dot{x}_1(t)\vert_{\Omega \cap \{x_1(t)=N_1\}} = - \gamma_1 N_1 < 0 \\
% &\dot{y}_1(t)\vert_{\Omega \cap \{y_1(t)=0\}} = \beta_{hv} a_1 M_1 \frac{p_{11} (1-\kappa u_1(t)) x_1(t) + p_{21} (1-\kappa u_2(t)) x_2(t)}{p_{11} N_1 + p_{21} N_2} \geq 0 \\
% &\dot{y}_1(t)\vert_{\Omega \cap \{y_1(t)=M_1\}} = - \mu_1 M_1 < 0 \\
% &\dot{x}_2(t)\vert_{\Omega \cap \{x_2(t)=0\}} = \beta_{vh} N_2 \left(\frac{p_{21} e^{-\mu_1 \tau} a_1 (1-\kappa u_2(t)) y_1(t)}{p_{11} N_1 + p_{21} N_2} + \frac{p_{22} e^{-\mu_2 \tau} a_2 (1-\kappa u_2(t)) y_2(t)}{p_{12} N_1 + p_{22} N_2}\right) \geq 0 \\
% &\dot{x}_2(t)\vert_{\Omega \cap \{x_2(t)=N_2\}} = - \gamma_2 N_2 < 0 \\
% &\dot{y}_2(t)\vert_{\Omega \cap \{y_2(t)=0\}} = \beta_{hv} a_2 M_2 \frac{p_{12} (1-\kappa u_1(t)) x_1(t) + p_{22} (1-\kappa u_2(t)) x_2(t)}{p_{12} N_1 + p_{22} N_2} \geq 0 \\
% &\dot{y}_2(t)\vert_{\Omega \cap \{y_2(t)=M_2\}} = - \mu_2 M_2 < 0
% \end{align*}

\begin{equation}
  \begin{split}
    &\dot{x}_1(t)\vert_{\Omega \cap \{x_1(t)=0\}} = (1-\kappa u_1(t))(b_{11} y_1(t) + b_{12} y_2(t)) \geq 0 \\
    &\dot{x}_1(t)\vert_{\Omega \cap \{x_1(t)=1\}} = - \gamma_1 < 0 \\
    &\dot{x}_2(t)\vert_{\Omega \cap \{x_2(t)=0\}} = (1-\kappa u_2(t))(b_{21} y_1(t) + b_{22} y_2(t)) \geq 0 \\
    &\dot{x}_2(t)\vert_{\Omega \cap \{x_2(t)=1\}} = - \gamma_2 < 0 \\
    &\dot{y}_1(t)\vert_{\Omega \cap \{y_1(t)=0\}} = c_{11}(1-\kappa u_1(t)) x_1(t) + c_{12}(1-\kappa u_2(t)) x_2(t) \geq 0 \\
    &\dot{y}_1(t)\vert_{\Omega \cap \{y_1(t)=1\}} = - \mu_1 < 0 \\
    &\dot{y}_2(t)\vert_{\Omega \cap \{y_2(t)=0\}} = c_{21}(1-\kappa u_1(t)) x_1(t) + c_{22}(1-\kappa u_2(t)) x_2(t) \geq 0 \\
    &\dot{y}_2(t)\vert_{\Omega \cap \{y_2(t)=1\}} = - \mu_2 < 0
  \end{split}
\end{equation}

\subsection{Кооперативност (квазимонотонност)}
% Доказваме квазимонотонността по дефиницията \ref{def:Cooperative}.
% Матрицата на Якоби $\mathrm{D} \mathbf{f}(x_1, x_2, y_1, y_2)(t)$ ще ни трябав и натам, затова нека я изведем изцяло:
% \begin{align*}
% &\mathrm{D} \mathbf{f}(x_1, x_2, y_1, y_2)(t) =
% \begin{pmatrix}
%   \pdv{F_{x_1}}{x_1} && \pdv{F_{x_1}}{x_2} && \pdv{F_{x_1}}{y_1} && \pdv{F_{x_1}}{y_2} \\
%   \pdv{F_{x_2}}{x_1} && \pdv{F_{x_2}}{x_2} && \pdv{F_{x_2}}{y_1} && \pdv{F_{x_2}}{y_2} \\
%   \pdv{F_{y_1}}{x_1} && \pdv{F_{y_1}}{x_2} && \pdv{F_{y_1}}{y_1} && \pdv{F_{y_1}}{y_2} \\
%   \pdv{F_{y_2}}{x_1} && \pdv{F_{y_2}}{x_2} && \pdv{F_{y_2}}{y_1} && \pdv{F_{y_2}}{y_2}
% \end{pmatrix} \\
% &\pdv{F_{x_1}}{x_1} = \pdv{\dot{x}_1}{x_1} = -\beta_{vh} \left(\frac{p_{11} e^{-\mu_1 \tau} a_1 (1-\kappa u_1(t)) y_1(t)}{p_{11} N_1 + p_{21} N_2} + \frac{p_{12} e^{-\mu_2 \tau} a_2 (1-\kappa u_1(t)) y_2(t)}{p_{12} N_1 + p_{22} N_2}\right) - \gamma_1 < 0 \\
% &\pdv{F_{x_1}}{y_1} = \pdv{\dot{x}_1}{y_1} = \beta_{vh} (N_1-x_1(t)) \frac{p_{11} e^{-\mu_1 \tau} a_1 (1-\kappa u_1(t))}{p_{11} N_1 + p_{21} N_2} \geq 0 \\
% &\pdv{F_{x_1}}{x_2} = \pdv{\dot{x}_1}{x_2} = 0 \\
% &\pdv{F_{x_1}}{y_2} = \pdv{\dot{x}_1}{y_2} = \beta_{vh} (N_1-x_1(t))\frac{p_{12} e^{-\mu_2 \tau} a_2 (1-\kappa u_1(t))}{p_{12} N_1 + p_{22} N_2} \geq 0 \\
% %
% &\pdv{F_{y_1}}{x_1} = \pdv{\dot{y}_1}{x_1} = \beta_{hv} a_1 (M_1-y_1(t)) \frac{p_{11} (1-\kappa u_1(t))}{p_{11} N_1 + p_{21} N_2} \geq 0 \\
% &\pdv{F_{y_1}}{y_1} = \pdv{\dot{y}_1}{y_1} = -\beta_{hv} a_1 \frac{p_{11} (1-\kappa u_1(t)) x_1(t) + p_{21} (1-\kappa u_2(t)) x_2(t)}{p_{11} N_1 + p_{21} N_2} - \mu_1 < 0 \\
% &\pdv{F_{y_1}}{x_2} = \pdv{\dot{y}_1}{x_2} = \beta_{hv} a_1 (M_1-y_1(t)) \frac{p_{21} (1-\kappa u_2(t))}{p_{11} N_1 + p_{21} N_2} \geq 0 \\
% &\pdv{F_{y_1}}{y_2} = \pdv{\dot{y}_1}{y_2} = 0 \\
% %
% &\pdv{F_{x_2}}{x_1} = \pdv{\dot{x}_2}{x_1} = 0 \\
% &\pdv{F_{x_2}}{y_1} = \pdv{\dot{x}_2}{y_1} = \beta_{vh} (N_2-x_2(t)) \frac{p_{21} e^{-\mu_1 \tau} a_1 (1-\kappa u_2(t))}{p_{11} N_1 + p_{21} N_2} \geq 0 \\
% &\pdv{F_{x_2}}{x_2} = \pdv{\dot{x}_2}{x_2} = -\beta_{vh} \left(\frac{p_{21} e^{-\mu_1 \tau} a_1 (1-\kappa u_2(t)) y_1(t)}{p_{11} N_1 + p_{21} N_2} + \frac{p_{22} e^{-\mu_2 \tau} a_2 (1-\kappa u_2(t)) y_2(t)}{p_{12} N_1 + p_{22} N_2}\right) - \gamma_2 < 0 \\
% &\pdv{F_{x_2}}{y_2} = \pdv{\dot{x}_2}{y_2} = \beta_{vh} (N_2-x_2(t)) \frac{p_{22} e^{-\mu_2 \tau} a_2 (1-\kappa u_2(t))}{p_{12} N_1 + p_{22} N_2} \geq 0 \\
% %
% &\pdv{F_{y_2}}{x_1} = \pdv{\dot{y}_2}{x_1} = \beta_{hv} a_2 (M_2-y_2(t)) \frac{p_{12} (1-\kappa u_1(t))}{p_{12} N_1 + p_{22} N_2} \geq 0 \\
% &\pdv{F_{y_12}}{y_1} = \pdv{\dot{y}_2}{y_1} = 0 \\
% &\pdv{F_{y_2}}{x_2} = \pdv{\dot{y}_2}{x_2} = \beta_{hv} a_2 (M_2-y_2(t)) \frac{p_{22} (1-\kappa u_2(t))}{p_{12} N_1 + p_{22} N_2} \geq 0 \\
% &\pdv{F_{y_2}}{y_2} = \pdv{\dot{y}_2}{y_2} = -\beta_{hv} a_2 \frac{p_{12} (1-\kappa u_1(t)) x_1(t) + p_{22} (1-\kappa u_2(t)) x_2(t)}{p_{12} N_1 + p_{22} N_2} - \mu_2 < 0
% \end{align*}

Якобианът за системата \ref{eq:TheDimensionlessProblem} може да се представи във вида:
\begin{equation}
  \mathrm{D} \mathbf{f}(x_1, x_2, y_1, y_2)(t) =
  \begin{pmatrix}
    \pdv{F_{x_1}}{x_1} && \pdv{F_{x_1}}{x_2} && \pdv{F_{x_1}}{y_1} && \pdv{F_{x_1}}{y_2} \\
    \pdv{F_{x_2}}{x_1} && \pdv{F_{x_2}}{x_2} && \pdv{F_{x_2}}{y_1} && \pdv{F_{x_2}}{y_2} \\
    \pdv{F_{y_1}}{x_1} && \pdv{F_{y_1}}{x_2} && \pdv{F_{y_1}}{y_1} && \pdv{F_{y_1}}{y_2} \\
    \pdv{F_{y_2}}{x_1} && \pdv{F_{y_2}}{x_2} && \pdv{F_{y_2}}{y_1} && \pdv{F_{y_2}}{y_2}
  \end{pmatrix}
  \end{equation}

\begin{proposition}
  Система \ref{eq:TheDimensionlessProblem} е кооперативна.
\end{proposition}

\begin{proof}
  \label{eq:JacobianElements}
  \begin{equation}
    \begin{split}
      &\pdv{F_{x_1}}{x_1} = \pdv{\dot{x}_1}{x_1} = -(1-\kappa u_1(t)) \left(b_{11} y_1(t) + b_{12} y_2(t)\right) - \gamma_1 < 0 \\
      &\pdv{F_{x_1}}{x_2} = \pdv{\dot{x}_1}{x_2} = 0 \\
      &\pdv{F_{x_1}}{y_1} = \pdv{\dot{x}_1}{y_1} = (1-x_1(t)) (1-\kappa u_1(t)) b_{11} \geq 0 \\
      &\pdv{F_{x_1}}{y_2} = \pdv{\dot{x}_1}{y_2} = (1-x_1(t)) (1-\kappa u_1(t)) b_{12} \geq 0 \\
      %
      &\pdv{F_{x_2}}{x_1} = \pdv{\dot{x}_2}{x_1} = 0 \\
      &\pdv{F_{x_2}}{x_2} = \pdv{\dot{x}_2}{x_2} = -(1-\kappa u_2(t)) \left(b_{21} y_1(t) + b_{22} y_2(t)\right) - \gamma_2 < 0 \\
      &\pdv{F_{x_2}}{y_1} = \pdv{\dot{x}_2}{y_1} = (1-x_2(t)) (1-\kappa u_2(t)) b_{21} \geq 0 \\
      &\pdv{F_{x_2}}{y_2} = \pdv{\dot{x}_2}{y_2} = (1-x_2(t)) (1-\kappa u_2(t)) b_{22} \geq 0 \\
      %
      &\pdv{F_{y_1}}{x_1} = \pdv{\dot{y}_1}{x_1} = (1-y_1(t)) c_{11}(1-\kappa u_1(t)) \geq 0 \\
      &\pdv{F_{y_1}}{x_2} = \pdv{\dot{y}_1}{x_2} = (1-y_1(t)) c_{12}(1-\kappa u_2(t)) \geq 0 \\
      &\pdv{F_{y_1}}{y_1} = \pdv{\dot{y}_1}{y_1} = -\left(c_{11}(1-\kappa u_1(t)) x_1(t) + c_{12}(1-\kappa u_2(t)) x_2(t)\right) - \mu_1 < 0 \\
      &\pdv{F_{y_1}}{y_2} = \pdv{\dot{y}_1}{y_2} = 0 \\
      %
      &\pdv{F_{y_2}}{x_1} = \pdv{\dot{y}_2}{x_1} = (1-y_2(t)) c_{21}(1-\kappa u_1(t)) \geq 0 \\
      &\pdv{F_{y_2}}{x_2} = \pdv{\dot{y}_2}{x_2} = (1-y_2(t)) c_{22}(1-\kappa u_2(t)) \geq 0 \\
      &\pdv{F_{y_12}}{y_1} = \pdv{\dot{y}_2}{y_1} = 0 \\
      &\pdv{F_{y_2}}{y_2} = \pdv{\dot{y}_2}{y_2} = -\left(c_{21}(1-\kappa u_1(t)) x_1(t) + c_{22}(1-\kappa u_2(t)) x_2(t)\right) - \mu_1 < 0
    \end{split}
    \end{equation}
  Извън главния диагонал има само неотрицателни елементи и така системата е кооперативна.
\end{proof}


% \subsection{Кооперативност (квазимонотонност)}
% Доказваме квазимонотонността по дефиницията \color{Red} ДЕФИНИЦИЯ!!!
% \color{Black}
% \begin{align*}
%   &\pdv{\dot{x}_1}{y_1} = \beta_{vh} (N_1-x_1(t)) \frac{p_{11} e^{-\mu_1 \tau} a_1 (1-\kappa u_1(t))}{p_{11} N_1 + p_{21} N_2} \geq 0 \\
%   &\pdv{\dot{x}_1}{x_2} = 0 \\
%   &\pdv{\dot{x}_1}{y_2} = \beta_{vh} (N_1-x_1(t))\frac{p_{12} e^{-\mu_2 \tau} a_2 (1-\kappa u_1(t))}{p_{12} N_1 + p_{22} N_2} \geq 0 \\
%   &\pdv{\dot{y}_1}{x_1} = \beta_{hv} a_1 (M_1-y_1(t)) \frac{p_{11} (1-\kappa u_1(t))}{p_{11} N_1 + p_{21} N_2} \geq 0 \\
%   &\pdv{\dot{y}_1}{x_2} = \beta_{hv} a_1 (M_1-y_1(t)) \frac{p_{21} (1-\kappa u_2(t))}{p_{11} N_1 + p_{21} N_2} \geq 0 \\
%   &\pdv{\dot{y}_1}{y_2} = 0 \\
%   &\pdv{\dot{x}_2}{x_1} = 0 \\
%   &\pdv{\dot{x}_2}{y_1} = \beta_{vh} (N_2-x_2(t)) \frac{p_{21} e^{-\mu_1 \tau} a_1 (1-\kappa u_2(t))}{p_{11} N_1 + p_{21} N_2} \geq 0 \\
%   &\pdv{\dot{x}_2}{y_2} = \beta_{vh} (N_2-x_2(t)) \frac{p_{22} e^{-\mu_2 \tau} a_2 (1-\kappa u_2(t))}{p_{12} N_1 + p_{22} N_2} \geq 0 \\
%   &\pdv{\dot{y}_2}{x_1} = \beta_{hv} a_2 (M_2-y_2(t)) \frac{p_{12} (1-\kappa u_1(t))}{p_{12} N_1 + p_{22} N_2} \geq 0 \\
%   &\pdv{\dot{y}_2}{y_1} = 0 \\
%   &\pdv{\dot{y}_2}{x_2} = \beta_{hv} a_2 (M_2-y_2(t)) \frac{p_{22} (1-\kappa u_2(t))}{p_{12} N_1 + p_{22} N_2} \geq 0 \\
% \end{align*}

\subsection{Неразложимост}
Използваме теорема 3.2.1 от \cite{Brualdi1991}, която гласи:
\begin{theorem}
  \label{theorem:ConnectedIrreducability}
  Матрица $A=(a_{ij})$ е неразложима точно когато ориентираният граф $G=(V,E)$, с върхове $V=\{1, \cdots, n\}$ и ребра $E=\{(i,j) \vert a_{ij} \neq 0 \}$, е силно свързан.
\end{theorem}

\begin{proposition}
  Якобианът на системата $\ref{eq:TheProblem}$ е неразложим.
\end{proposition}
\begin{proof}
  Заместваме ненулевите елементи на $\mathrm{D} \mathbf{f}$ с 1 (тях знаем от \ref{eq:JacobianElements}). Така получаваме графа с матрица на съседство $\mathrm{D}$:
  \begin{equation}
    \mathrm{D} =
    \begin{pmatrix}
      1 && 1 && 0 && 1 \\
      1 && 1 && 1 && 0 \\
      0 && 1 && 1 && 1 \\
      1 && 0 && 1 && 1 \\
    \end{pmatrix}
    \implies
    \mathrm{D}^3 =
    \begin{pmatrix}
      7 && 7 && 6 && 7 \\
      7 && 7 && 7 && 6 \\
      6 && 7 && 7 && 7 \\
      7 && 6 && 7 && 7 \\
    \end{pmatrix}
    >
    \mathscr{O}
  \end{equation}
  Тъй графа има 4 върха, с матрицата на съседство повдигната на 3-та степен виждаме кои върхове са свързани помежду си и кои не (тъй като ще получим информация за свързаните компоненти, от факта че всеки прост път е с дължина не по-голяма от 3).
  Понеже има единствена свързана компонента, то графът е силно свързан.
  Спрямо $\ref{theorem:ConnectedIrreducability}$ откъдето $\mathrm{D}\mathbf{f}$ е неразложима.
  \end{proof}

\subsection{Силна вдлъбнатост}
\begin{proposition}
  Системата е силно вдлъбната, т.е. $\mathbf{0} < \mathbf{z}_1 < \mathbf{z}_2 \implies \mathrm{D}\mathbf{f}(\mathbf{z}_2) < \mathrm{D}\mathbf{f}(\mathbf{z}_1)$
\end{proposition}

\begin{proof}
Достатъчно условие за това е всяка компонента на $\mathrm{D}\mathbf{f}$ да е нерастяща функция по всички променливи, като за поне една от тях да е намаляваща. Това може да проверим с производни по различните променливи:
{\allowdisplaybreaks
  \begin{align*}
    &\pdv{F_{x_1}}{x_1}{x_1} = 0, \quad
    \pdv{F_{x_1}}{x_1}{x_2} = 0, \quad
    \pdv{F_{x_1}}{x_1}{y_1} = -(1-\kappa u_1(t)) b_{11} < 0, \quad
    \pdv{F_{x_1}}{x_1}{y_2} = -(1-\kappa u_1(t)) b_{12} < 0 \\
    %
    &\pdv{F_{x_1}}{x_2}{x_1} = 0, \quad
    \pdv{F_{x_1}}{x_2}{x_2} = 0, \quad
    \pdv{F_{x_1}}{x_2}{y_1} = 0, \quad
    \pdv{F_{x_1}}{x_2}{y_2} = 0 \\
    %
    &\pdv{F_{x_1}}{y_1}{x_1} = -(1-\kappa u_1(t)) b_{11} < 0, \quad
    \pdv{F_{x_1}}{y_1}{x_2} = 0, \quad
    \pdv{F_{x_1}}{y_1}{y_1} = 0, \quad
    \pdv{F_{x_1}}{y_1}{y_2} = 0 \\
    %
    &\pdv{F_{x_1}}{y_2}{x_1} = -(1-\kappa u_1(t)) b_{12} < 0, \quad
    \pdv{F_{x_1}}{y_2}{x_2} = 0, \quad
    \pdv{F_{x_1}}{y_2}{y_1} = 0, \quad
    \pdv{F_{x_1}}{y_2}{y_2} = 0 \\
    %
    %
    %
    &\pdv{F_{x_2}}{x_1}{x_1} = 0, \quad
    \pdv{F_{x_2}}{x_1}{x_2} = 0, \quad
    \pdv{F_{x_2}}{x_1}{y_1} = 0, \quad
    \pdv{F_{x_2}}{x_1}{y_2} = 0 \\
    %
    &\pdv{F_{x_2}}{x_2}{x_1} = 0, \quad
    \pdv{F_{x_2}}{x_2}{y_1} = -(1-\kappa u_2(t)) b_{21}, \quad
    \pdv{F_{x_2}}{x_2}{x_2} = 0, \quad
    \pdv{F_{x_2}}{x_2}{y_2} = -(1-\kappa u_2(t)) b_{22} < 0 \\
    %
    &\pdv{F_{x_2}}{y_1}{x_1} = 0, \quad
    \pdv{F_{x_2}}{y_1}{x_2} = -(1-\kappa u_2(t)) b_{21}, \quad
    \pdv{F_{x_2}}{y_1}{y_1} = 0, \quad
    \pdv{F_{x_2}}{y_1}{y_2} = 0 \\
    %
    &\pdv{F_{x_2}}{y_2}{x_1} = 0, \quad
    \pdv{F_{x_2}}{y_2}{x_2} = -(1-\kappa u_2(t)) b_{22} < 0, \quad
    \pdv{F_{x_2}}{y_2}{y_1} = 0, \quad
    \pdv{F_{x_2}}{y_2}{y_2} = 0 \\
    %
    %
    %
    &\pdv{F_{y_1}}{x_1}{x_1} = 0, \quad
    \pdv{F_{y_1}}{x_1}{x_2} = 0, \quad
    \pdv{F_{y_1}}{x_1}{y_1} = -(1-\kappa u_1(t))c_{11} < 0, \quad
    \pdv{F_{y_1}}{x_1}{y_2} = 0 \\
    %
    &\pdv{F_{y_1}}{x_2}{x_1} = 0, \quad
    \pdv{F_{y_1}}{x_2}{y_1} = -(1-\kappa u_1(t))c_{12} < 0, \quad
    \pdv{F_{y_1}}{x_2}{x_2} = 0, \quad
    \pdv{F_{y_1}}{x_2}{y_2} = 0 \\
    %
    &\pdv{F_{y_1}}{y_1}{x_1} = -(1-\kappa u_1(t))c_{11} < 0, \quad
    \pdv{F_{y_1}}{y_1}{x_2} = -(1-\kappa u_1(t))c_{12}, \quad
    \pdv{F_{y_1}}{y_1}{y_1} = 0, \quad
    \pdv{F_{y_1}}{y_1}{y_2} = 0 \\
    %
    &\pdv{F_{y_1}}{y_2}{x_1} = 0, \quad
    \pdv{F_{y_1}}{y_2}{x_2} = 0, \quad
    \pdv{F_{y_1}}{y_2}{y_1} = 0, \quad
    \pdv{F_{y_1}}{y_2}{y_2} = 0 \\
    %
    %
    %
    &\pdv{F_{y_2}}{x_1}{x_1} = 0, \quad
    \pdv{F_{y_2}}{x_1}{x_2} = 0, \quad
    \pdv{F_{y_2}}{x_1}{y_1} = 0, \quad
    \pdv{F_{y_2}}{x_1}{y_2} = -(1-\kappa u_1(t)) c_{21} \\
    %
    &\pdv{F_{y_2}}{x_2}{x_1} = 0, \quad
    \pdv{F_{y_2}}{x_2}{x_2} = 0, \quad
    \pdv{F_{y_2}}{x_2}{y_1} = 0, \quad
    \pdv{F_{y_2}}{x_2}{y_2} = -(1-\kappa u_2(t)) c_{22} < 0 \\
    %
    &\pdv{F_{y_2}}{y_1}{x_1} = 0, \quad
    \pdv{F_{y_2}}{y_1}{x_2} = 0, \quad
    \pdv{F_{y_2}}{y_1}{y_1} = 0, \quad
    \pdv{F_{y_2}}{y_1}{y_2} = 0 \\
    %
    &\pdv{F_{y_2}}{y_2}{x_1} = -(1-\kappa u_1(t)) c_{21} < 0, \quad
    \pdv{F_{y_2}}{y_2}{x_2} = -(1-\kappa u_2(t)) c_{22} < 0, \quad
    \pdv{F_{y_2}}{y_2}{y_1} = 0, \quad
    \pdv{F_{y_2}}{y_2}{y_2} = 0
  \end{align*}
}
Така достатъчното условие е изпълнено и системата притежава силна вдлъбнатост.
\end{proof}


\subsection{Неподвижни точки}
Веднага се вижда, че $\mathbf{f}(\mathbf{0}, \mathbf{u}) = \mathbf{0}$, т.е. $\mathbf{0}$ е тривиалната неподвижна точка на системата.

Тъй като системата е с управление, не може в общия случай да говорим за равновесни точки, понеже промени по него водят до промени по дясната страна. Да предположим, че сме фиксирали константно управление. Тогава системата става автономна, но е силно нелинейна и с голяма размерност, откъдето не е възможно да бъдат изведени аналитични изрази за координатите на равновесните точки, различни от  $\mathbf{0}$. С помощта на теорията на кооперативните системи може обаче да получим техния брой.

\begin{proposition}
  Нека $\mathbf{u}(t)\equiv \mathbf{u}=const$. Тогава системата \ref{eq:TheDimensionlessProblem} има най-много една нетривиална равновесна точка (ендемична точка).
\end{proposition}

\begin{proof}
  Видя се, че системата е кооперативна, с неразложима матрица на Якоби и е силно вдлъбната. Тогава имаме всички условия от Следствие 3.2 от статията \cite{Smith1986} на Smith.

  Нека $\mathscr{R}_0(\mathbf{u}) \leq 1$. Тогава от \cite{Driessche2002} $\mathbf{0}$ е единствена устойчива неподвижна точка.

  Нека $\mathscr{R}_0(\mathbf{u}) > 1$. Тогава от \cite{Driessche2002}, $\mathbf{0}$ е неустойчива неподвижна точка и сме във втория случай от следствието на Smith. Вече доказахме, че решението е ограничено. Тогава е изпълнен подслучай (b) и съществува точно една друга устойчива равновесна точка $\mathbf{E}^*$.

\end{proof}

% Да изведем матрицата на Якоби $\mathrm{D} \mathbf{f}(x_1, y_1, x_2, y_2)(t)$:
% \begin{align*}
%   &\mathrm{D} \mathbf{f}(x_1, y_1, x_2, y_2)(t) = \\
%   &
%   \begin{tiny}
%     \begin{pmatrix}
%       - \beta_{vh} \left(\frac{p_{11} e^{-\mu_1 \tau} a_1 (1-\kappa u_1) y_1(t)}{p_{11} N_1 + p_{21} N_2} + \frac{p_{12} e^{-\mu_2 \tau} a_2 (1-\kappa u_1) y_2(t)}{p_{12} N_1 + p_{22} N_2}\right) - \gamma_1 && \beta_{vh} (N_1-x_1(t)) \frac{p_{11} e^{-\mu_1 \tau} a_1 (1-\kappa u_1)}{p_{11} N_1 + p_{21} N_2} && 0 && \beta_{vh} (N_1-x_1(t)) \frac{p_{12} e^{-\mu_2 \tau} a_2 (1-\kappa u_1)}{p_{12} N_1 + p_{22} N_2} \\
%       \beta_{hv} a_1 (M_1-y_1(t)) \frac{p_{11} (1-\kappa u_1)}{p_{11} N_1 + p_{21} N_2} && -\beta_{hv} a_1 \frac{p_{11} (1-\kappa u_1) x_1(t) + p_{21} (1-\kappa u_2) x_2(t)}{p_{11} N_1 + p_{21} N_2} - \mu_1 && \beta_{hv} a_1 (M_1-y_1(t)) \frac{p_{21} (1-\kappa u_2)}{p_{11} N_1 + p_{21} N_2}&& 0 \\
%       0 && \beta_{vh} (N_2-x_2(t)) \frac{p_{21} e^{-\mu_1 \tau} a_1 (1-\kappa u_2)}{p_{11} N_1 + p_{21} N_2} && -\beta_{vh} \left(\frac{p_{21} e^{-\mu_1 \tau} a_1 (1-\kappa u_2) y_1(t)}{p_{11} N_1 + p_{21} N_2} + \frac{p_{22} e^{-\mu_2 \tau} a_2 (1-\kappa u_2) y_2(t)}{p_{12} N_1 + p_{22} N_2}\right) - \gamma_2 && \beta_{vh} (N_2-x_2(t)) \frac{p_{22} e^{-\mu_2 \tau} a_2 (1-\kappa u_2) y_2(t)}{p_{12} N_1 + p_{22} N_2} \\
%       \beta_{hv} a_2 (M_2-y_2(t)) \frac{p_{12} (1-\kappa u_1)}{p_{12} N_1 + p_{22} N_2} && 0 && \beta_{hv} a_2 (M_2-y_2(t)) \frac{p_{22} (1-\kappa u_2) x_2(t)}{p_{12} N_1 + p_{22} N_2} && -\beta_{hv} a_2 \frac{p_{12} (1-\kappa u_1) x_1(t) + p_{22} (1-\kappa u_2) x_2(t)}{p_{12} N_1 + p_{22} N_2} - \mu_2 \\
%     \end{pmatrix}
%   \end{tiny}
% \end{align*}

% % Да разгледаме $A_{ij}=\frac{a_j b_j p_ij e^{-\mu_j \tau_j}}{H_j}$, $B_{ij}=\frac{a_j c_j \delta_{ij}}{H_j}$
% \color{Red} ДА СЕ ОПРАВЯТ ЗНАМЕНАТЕЛИТЕ!!!
% \color{Black}
% Да означим $A_{ij}=\frac{a_j b_j p_ij e^{-\mu_j \tau_j}}{H_j}$, $B_{ij}=\frac{a_j c_j p_ji}{H_j}$. Тогава системата ни добива точно вида $(2.4)$ от \color{Red} ЦИТАТ Cosner, обаче с оправените Bichara?!!!
% \color{Black}
% и ползвайки Теорема 1 от същата статия с $\cap{A}=(\frac{A_{ij} N_i}{\gamma_i})$, то ако зададем $R_0^2=\rho(\cap{А}\cap{B})$,

Ендемичната точка (когато съществува) може да бъде намерена приблизително по два начина.
Първият е да се пусне числена симулация на системата и когато решението вече не се мени значително, то знаем, че сме в околност на епидемичната точка, откъдето може да я преближим с решението в съответния момент.
Другият начин е да се реши числено нелинейната система, получена когато се занулят левите страни на $\ref{eq:TheProblem}$.
Полученото решение ще е равновесна точка (но може да получим и $\mathbf{0}$).
Варирайки първоначалното приближение, ще получим и приближение на ендемичната точка.
