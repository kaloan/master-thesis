\section{Съществуване на решение и основни свойства}
Първо, отбелязваме, че ако е в сила $z, z' < C_z$ и $s, s' < C_s$, то е изпълнено:
\begin{align*}
  &|(C_z - z) s - (C_z - z') s'| =
  |C_z s - z s - C_z s' + z' s' + z s' - z s'| =
  |C_z (s - s') - z (s - s') - s' (z - z')| \leq \\
  &|C_z| |s - s'| + |z| |s - s'| + |s'| |z - z'| \leq
  2 |C_z| |s - s'|  + |C_s| |z - z'| \leq
  \max\{2 |C_z|, |C_s|\} (|s-s'| + |z - z'|)
\end{align*}

\subsection{Липшицовост по фазови променливи}
Ще използваме това твърдение при дозателството на Липшицовата непрекъснатост на дясната страна по фазовите променливи $x_1, y_1, x_2, y_2$. Взимаме произволни допустими двойки, тоест $(x_1, y_1, x_2, y_2), (x'_1, y'_1, x'_2, y'_2) \in \Omega$ и $(u_1, u_2), (u'_1, u'_2) \in [0, \bar{u}_1] \cross [0, \bar{u}_2]$.
Първо от неравенството на триъгълника имаме, че:
\begin{align*}
  &\|\mathbf{F}(x_1, y_1, x_2, y_2, u_1, u_2) - \mathbf{F}(x'_1, y'_1, x'_2, y'_2, u'_1, u'_2)\| \leq \\
  &|f_1(x_1, y_1, x_2, y_2, u_1, u_2) - f_1(x'_1, y'_1, x'_2, y'_2, u'_1, u'_2)| + |g_2(x_1, y_1, x_2, y_2, u_1, u_2) - g_2(x'_1, y'_1, x'_2, y'_2, u'_1, u'_2)| + \\
  &|g_1(x_1, y_1, x_2, y_2, u_1, u_2) - g_1(x'_1, y'_1, x'_2, y'_2, u'_1, u'_2)| + |f_2(x_1, y_1, x_2, y_2, u_1, u_2) - f_2(x'_1, y'_1, x'_2, y'_2, u'_1, u'_2)|
\end{align*}
Сега може неколкократно да ползваме горната оценка за $f_1$:
\begin{align*}
  &\bigg|\beta_{vh} (N_1-x_1) \left(\frac{a_1 p_{11} e^{-\mu_1 \tau} (1-\kappa u_1) y_1}{p_{11} N_1 + p_{21} N_2} + \frac{a_2 p_{12} e^{-\mu_2 \tau} (1-\kappa u_1) y_2}{p_{12} N_1 + p_{22} N_2 }\right) - \gamma_1 x_1 - \\
  &\beta_{vh} (N_1-x'_1) \left(\frac{a_1 p_{11} e^{-\mu_1 \tau} (1-\kappa u'_1) y'_1}{p_{11} N_1 + p_{21} N_2} + \frac{a_2 p_{12} e^{-\mu_2 \tau} (1-\kappa u'_1) y'_2}{p_{12} N_1 + p_{22} N_2 }\right) + \gamma_1 x'_1\bigg| \leq \\
  &\frac{\beta_{vh} a_1 p_{11} e^{-\mu_1 \tau}}{p_{11} N_1 + p_{21} N_2} \left|(N_1-x_1) [(1-\kappa u_1) y_1] - (N_1-x'_1) [(1-\kappa u'_1) y'_1]\right| + \\
  &\frac{\beta_{vh} a_2 p_{12} e^{-\mu_2 \tau}}{p_{12} N_1 + p_{22} N_2} \left|(N_1-x_1) [(1-\kappa u_1) y_2] - (N_1-x'_1) [(1-\kappa u'_1) y'_2]\right| + \\
  &\gamma_1 |x_1-x'_1|
\end{align*}
Имаме, че $x_1, x'_1 \leq N_1, \quad (1-\kappa u_1)y_1, (1-\kappa u_1) y'_1 \leq M_1, \quad (1-\kappa u_1)y_2, (1-\kappa u_1) y'_2 \leq M_2$:
\begin{align*}
  &\left|(N_1-x_1) [(1-\kappa u_1) y_1] - (N_1-x'_1) [(1-\kappa u'_1) y'_1]\right| \leq 2 N_1 |(1-\kappa u_1) y_1 - (1-\kappa u'_1) y'_1| + M_1 |x_1 - x'_1| \leq \\
  &2 N_1 (2|y_1 - y'_1| + M_1 \kappa |u_1 - u'_1|) + M_1 |x_1 - x'_1| \\
  &\left|(N_1-x_1) [(1-\kappa u_1) y_2] - (N_1-x'_1) [(1-\kappa u'_1) y'_2]\right| \leq 2 N_1 |(1-\kappa u_1) y_2 - (1-\kappa u'_1) y'_2| + M_2 |x_1 - x'_1| \leq \\
  &2 N_1 (2|y_2 - y'_2| + M_2 \kappa |u_1 - u'_1|) + M_2 |x_1 - x'_1|
\end{align*}
Тук също ползвахме $1-\kappa u_1, 1-\kappa u'_1 \leq 1, \quad y_1, y'_1 \leq M_1, \quad y_2, y'_2 \leq M_2$. Така получихе оценка отгоре за първото събираемо \\
Тъй като видът на $f_2$ е същият с точност до индекси, то директно получаваме и оценка за третото събираемо. \\

Сега да разгледаме за $g_1$:
\begin{align*}
  & \bigg|\beta_{hv} a_1 (M_1-y_1) \frac{p_{11} (1-\kappa u_1) x_1 + p_{21} (1-\kappa u_2) x_2}{p_{11} N_1 + p_{21} N_2} - \mu_1 y_1 - \\
  &\beta_{hv} a_1 (M_1-y'_1) \frac{p_{11} (1-\kappa u'_1) x'_1 + p_{21} (1-\kappa u'_2) x'_2}{p_{11} N_1 + p_{21} N_2} + \mu_1 y'_1\bigg| \leq \\
  & \frac{\beta_{hv} a_1 p_{11}}{p_{11} N_1 + p_{21} N_2} \left|(M_1-y_1) [(1-\kappa u_1) x_1] - (M_1-y'_1) [(1-\kappa u'_1) x'_1]\right| + \\
  &\frac{\beta_{hv} a_1 p_{21}}{p_{11} N_1 + p_{21} N_2} \left|(M_1-y_1) [(1-\kappa u_2) x_2] - (M_1-y'_1) [(1-\kappa u'_2) x'_2]\right| + \\
  &\mu_1 |y_1 - y'_1|
\end{align*}
Ограниченията са $y_1, y'_1 \leq M_1, \quad (1-\kappa u_1)x_1, (1-\kappa u'_1)x'_1 \leq N_1, \quad (1-\kappa u_2)x_2, (1-\kappa u'_2)x'_2 \leq N_2$:
\begin{align*}
  &\left|(M_1-y_1) [(1-\kappa u_1) x_1] - (M_1-y'_1) [(1-\kappa u'_1) x'_1]\right| \leq 2 M_1 |(1-\kappa u_1) x_1 - (1-\kappa u'_1) x'_1| + N_1 |y_1 - y'_1| \leq \\
  & 2 M_1 (2|x_1 - x'_1| + N_1 \kappa |u_1 - u'_1|) + N_1 |y_1 - y'_1| \\
  &\left|(M_1-y_1) [(1-\kappa u_2) x_2] - (M_1-y'_1) [(1-\kappa u'_2) x'_2]\right| \leq 2 M_1 |(1-\kappa u_2) x_2 - (1-\kappa u'_2) x'_2| + N_2 |y_1 - y'_1| \\
  & 2 M_1 (2|x_2 - x'_2| + N_2 \kappa |u_2 - u'_2|) + N_2 |y_1 - y'_1|
\end{align*}
Тук също ползвахме $1-\kappa u_1, 1-\kappa u'_1, 1-\kappa u_2, 1-\kappa u'_2 \leq 1, \quad x_1, x'_1 \leq M_1, \quad x_2, x'_2 \leq M_2$. Така получихе оценка отгоре за второто събираемо \\
Тъй като видът на $g_2$ е същият с точност до индекси, то директно получаваме и оценка за четвъртото събираемо. \\
За да проверим липшицовостта по фазовите променливи, то заместваме с $u_1 = 1'_1, u_2 = u'_2$ всичко и за цялата дясна страна е в сила:
% \begin{align*}
% &\|\mathbf{F}(x_1, y_1, x_2, y_2, u_1, u_2) - \mathbf{F}(x'_1, y'_1, x'_2, y'_2, u'_1, u'_2)\| \leq \\
% &\frac{\beta_{vh} a_1 p_{11} e^{-\mu_1 \tau}}{p_{11} N_1 + p_{21} N_2} (2 N_1 (2|y_1 - y'_1| + M_1 \kappa |u_1 - u'_1|) + M_1 |x_1 - x'_1|) + \\
% &\frac{\beta_{vh} a_2 p_{12} e^{-\mu_2 \tau}}{p_{12} N_1 + p_{22} N_2}(2 N_1 (2|y_2 - y'_2| + M_2 \kappa |u_1 - u'_1|) + M_1 |x_1 - x'_1|) + \gamma_1 |x_1-x'_1| + \\
% &\frac{\beta_{hv} a_1 p_{11}}{p_{11} N_1 + p_{21} N_2} (2 M_1 (2|x_1 - x'_1| + N_1 \kappa |u_1 - u'_1|) + N_1 |y_1 - y'_1|) + \\
% &\frac{\beta_{hv} a_1 p_{21}}{p_{11} N_1 + p_{21} N_2} (2 M_1 (2|x_2 - x'_2| + N_2 \kappa |u_2 - u'_2|) + N_2 |y_1 - y'_1|) + \mu_1 |y_1 - y'_1| + \\
% &\frac{\beta_{vh} a_1 p_{21} e^{-\mu_1 \tau}}{p_{11} N_1 + p_{21} N_2} (2 N_1 (2|y_1 - y'_1| + M_1 \kappa |u_1 - u'_1|) + M_1 |x_1 - x'_1|) + \\
% &\frac{\beta_{vh} a_2 p_{22} e^{-\mu_2 \tau}}{p_{12} N_1 + p_{22} N_2}(2 N_1 (2|y_2 - y'_2| + M_2 \kappa |u_1 - u'_1|) + M_1 |x_1 - x'_1|) + \gamma_2 |x_2-x'_2| + \\
% &\frac{\beta_{hv} a_2 p_{12}}{p_{12} N_1 + p_{22} N_2} (2 M_2 (2|x_1 - x'_1| + N_1 \kappa |u_1 - u'_1|) + N_1 |y_2 - y'_2|) + \\
% &\frac{\beta_{hv} a_2 p_{22}}{p_{12} N_1 + p_{22} N_2} (2 M_2 (2|x_2 - x'_2| + N_2 \kappa |u_2 - u'_2|) + N_2 |y_2 - y'_2|) + \mu_2 |y_2 - y'_2| \leq
% \end{align*}
\begin{align*}
  &\|\mathbf{F}(x_1, y_1, x_2, y_2, u'_1, u'_2) - \mathbf{F}(x'_1, y'_1, x'_2, y'_2, u'_1, u'_2)\| \leq \\
  &\frac{\beta_{vh} a_1 p_{11} e^{-\mu_1 \tau}}{p_{11} N_1 + p_{21} N_2} (4 N_1 |y_1 - y'_1| + M_1 |x_1 - x'_1|) +
  \frac{\beta_{vh} a_2 p_{12} e^{-\mu_2 \tau}}{p_{12} N_1 + p_{22} N_2}(4 N_1 |y_2 - y'_2| + M_1 |x_1 - x'_1|) + \gamma_1 |x_1-x'_1| + \\
  &\frac{\beta_{hv} a_1 p_{11}}{p_{11} N_1 + p_{21} N_2} (4 M_1 |x_1 - x'_1| + N_1 |y_1 - y'_1|) +
  \frac{\beta_{hv} a_1 p_{21}}{p_{11} N_1 + p_{21} N_2} (4 M_1 |x_2 - x'_2| + N_2 |y_1 - y'_1|) + \mu_1 |y_1 - y'_1| + \\
  &\frac{\beta_{vh} a_1 p_{21} e^{-\mu_1 \tau}}{p_{11} N_1 + p_{21} N_2} (4 N_1 |y_1 - y'_1| + M_1 |x_1 - x'_1|) +
  \frac{\beta_{vh} a_2 p_{22} e^{-\mu_2 \tau}}{p_{12} N_1 + p_{22} N_2}(4 N_1|y_2 - y'_2| + M_1 |x_1 - x'_1|) + \gamma_2 |x_2-x'_2| + \\
  &\frac{\beta_{hv} a_2 p_{12}}{p_{12} N_1 + p_{22} N_2} (4 M_2 |x_1 - x'_1| + N_1 |y_2 - y'_2|) +
  \frac{\beta_{hv} a_2 p_{22}}{p_{12} N_1 + p_{22} N_2} (4 M_2 |x_2 - x'_2| + N_2 |y_2 - y'_2|) + \mu_2 |y_2 - y'_2| \leq \\
  & C \|(x_1, y_1, x_2, y_2) - (x'_1, y'_1, x'_2, y'_2)\|
\end{align*}
Накрая се използват неравенства от вида $|x_1-x'_1| \leq \|(x_1, y_1, x_2, y_2) - (x'_1, y'_1, x'_2, y'_2)\|$. Тогава, спрямо общата теория на диференциалните решения с управление, съществува единствено решение на $\eqref{eq:TheProblem}$ за произволни $t$.

\subsection{Ограниченост на решението}
Да разгледаме множеството
\begin{equation}
  X = \{0 \leq x_1 \leq N_1, 0 \leq y_1 \leq M_1, 0 \leq x_2 \leq N_2, 0 \leq y_2 \leq M_2\}
\end{equation}
Началното условие е някъде в това множество, тъй като популациите са неотрицателни, а заразените индивиди не са над общата популация за съответната категория. Лесно може да се види, че е в сила:
\begin{align}
  (x_1(0), y_1(0), x_2(0), y_2(0)) \in X \implies \forall{t>0}\left((x_1(0), y_1(0), x_2(0), y_2(0)) \in X\right)
\end{align}
Трябва да се покаже, че $\mathbf{F}$ сочи към вътрешността на $X$, ако решението се намира по границата $\partial X$. Но това наистина е така, от:
\begin{align*}
  &\dot{x}_1(t)\vert_{X \cap \{x_1(t)=0\}} = \beta_{vh} N_1(t) \left(\frac{p_{11} e^{-\mu_1 \tau} a_1 (1-\kappa u_1(t)) y_1(t)}{p_{11} N_1 + p_{21} N_2} + \frac{p_{12} e^{-\mu_2 \tau} a_2 (1-\kappa u_1(t)) y_2(t)}{p_{12} N_1 + p_{22} N_2 }\right) \geq 0 \\
  &\dot{x}_1(t)\vert_{X \cap \{x_1(t)=N_1\}} = - \gamma_1 N_1 < 0 \\
  &\dot{y}_1(t)\vert_{X \cap \{y_1(t)=0\}} = \beta_{hv} a_1 M_1 \frac{p_{11} (1-\kappa u_1(t)) x_1(t) + p_{21} (1-\kappa u_2(t)) x_2(t)}{p_{11} N_1 + p_{21} N_2} \geq 0 \\
  &\dot{y}_1(t)\vert_{X \cap \{y_1(t)=M_1\}} = - \mu_1 M_1 < 0 \\
  &\dot{x}_2(t)\vert_{X \cap \{x_2(t)=0\}} = \beta_{vh} N_2 \left(\frac{p_{21} e^{-\mu_1 \tau} a_1 (1-\kappa u_2(t)) y_1(t)}{p_{11} N_1 + p_{21} N_2} + \frac{p_{22} e^{-\mu_2 \tau} a_2 (1-\kappa u_2(t)) y_2(t)}{p_{12} N_1 + p_{22} N_2}\right) \geq 0 \\
  &\dot{x}_2(t)\vert_{X \cap \{x_2(t)=N_2\}} = - \gamma_2 N_2 < 0 \\
  &\dot{y}_2(t)\vert_{X \cap \{y_2(t)=0\}} = \beta_{hv} a_2 M_2 \frac{p_{12} (1-\kappa u_1(t)) x_1(t) + p_{22} (1-\kappa u_2(t)) x_2(t)}{p_{12} N_1 + p_{22} N_2} \geq 0 \\
  &\dot{y}_2(t)\vert_{X \cap \{y_2(t)=M_2\}} = - \mu_2 M_2 < 0
\end{align*}

\subsection{Кооперативност (квазимонотонност)}
Доказваме квазимонотонността по дефиницията \color{Red} ДЕФИНИЦИЯ!!!
\color{Black}
Матрицата на Якоби $\mathrm{D} \mathbf{F}(x_1, y_1, x_2, y_2)(t)$ ще ни трябав и натам, затова нека я изведем изцяло:
\begin{align*}
  &\mathrm{D} \mathbf{F}(x_1, y_1, x_2, y_2)(t) =
  \begin{pmatrix}
    \pdv{\mathbf{F}_{x_1}}{x_1} && \pdv{\mathbf{F}_{x_1}}{y_1} && \pdv{\mathbf{F}_{x_1}}{x_2} && \pdv{\mathbf{F}_{x_1}}{y_2} \\
    \pdv{\mathbf{F}_{y_1}}{x_1} && \pdv{\mathbf{F}_{y_1}}{y_1} && \pdv{\mathbf{F}_{y_1}}{x_2} && \pdv{\mathbf{F}_{y_1}}{y_2} \\
    \pdv{\mathbf{F}_{x_2}}{x_1} && \pdv{\mathbf{F}_{x_2}}{y_1} && \pdv{\mathbf{F}_{x_2}}{x_2} && \pdv{\mathbf{F}_{x_2}}{y_2} \\
    \pdv{\mathbf{F}_{y_2}}{x_1} && \pdv{\mathbf{F}_{y_2}}{y_1} && \pdv{\mathbf{F}_{y_2}}{x_2} && \pdv{\mathbf{F}_{y_2}}{y_2}
  \end{pmatrix} \\
  &\pdv{\mathbf{F}_{x_1}}{x_1} = \pdv{\dot{x}_1}{x_1} = -\beta_{vh} \left(\frac{p_{11} e^{-\mu_1 \tau} a_1 (1-\kappa u_1) y_1(t)}{p_{11} N_1 + p_{21} N_2} + \frac{p_{12} e^{-\mu_2 \tau} a_2 (1-\kappa u_1) y_2(t)}{p_{12} N_1 + p_{22} N_2}\right) - \gamma_1 < 0 \\
  &\pdv{\mathbf{F}_{x_1}}{y_1} = \pdv{\dot{x}_1}{y_1} = \beta_{vh} (N_1-x_1(t)) \frac{p_{11} e^{-\mu_1 \tau} a_1 (1-\kappa u_1(t))}{p_{11} N_1 + p_{21} N_2} \geq 0 \\
  &\pdv{\mathbf{F}_{x_1}}{x_2} = \pdv{\dot{x}_1}{x_2} = 0 \\
  &\pdv{\mathbf{F}_{x_1}}{y_2} = \pdv{\dot{x}_1}{y_2} = \beta_{vh} (N_1-x_1(t))\frac{p_{12} e^{-\mu_2 \tau} a_2 (1-\kappa u_1(t))}{p_{12} N_1 + p_{22} N_2} \geq 0 \\
  &\pdv{\mathbf{F}_{y_1}}{y_1} = \pdv{\dot{y}_1}{x_1} = \beta_{hv} a_1 (M_1-y_1(t)) \frac{p_{11} (1-\kappa u_1(t))}{p_{11} N_1 + p_{21} N_2} \geq 0 \\
  &\pdv{\mathbf{F}_{y_1}}{y_1} = \pdv{\dot{y}_1}{y_1} = -\beta_{hv} a_1 \frac{p_{11} (1-\kappa u_1) x_1(t) + p_{21} (1-\kappa u_2) x_2(t)}{p_{11} N_1 + p_{21} N_2} - \mu_1 < 0 \\
  &\pdv{\mathbf{F}_{y_1}}{x_2} = \pdv{\dot{y}_1}{x_2} = \beta_{hv} a_1 (M_1-y_1(t)) \frac{p_{21} (1-\kappa u_2(t))}{p_{11} N_1 + p_{21} N_2} \geq 0 \\
  &\pdv{\mathbf{F}_{y_1}}{y_2} = \pdv{\dot{y}_1}{y_2} = 0 \\
  &\pdv{\mathbf{F}_{x_2}}{x_1} = \pdv{\dot{x}_2}{x_1} = 0 \\
  &\pdv{\mathbf{F}_{x_2}}{y_1} = \pdv{\dot{x}_2}{y_1} = \beta_{vh} (N_2-x_2(t)) \frac{p_{21} e^{-\mu_1 \tau} a_1 (1-\kappa u_2(t))}{p_{11} N_1 + p_{21} N_2} \geq 0 \\
  &\pdv{\mathbf{F}_{x_2}}{x_2} = \pdv{\dot{x}_2}{x_2} = -\beta_{vh} \left(\frac{p_{21} e^{-\mu_1 \tau} a_1 (1-\kappa u_2) y_1(t)}{p_{11} N_1 + p_{21} N_2} + \frac{p_{22} e^{-\mu_2 \tau} a_2 (1-\kappa u_2) y_2(t)}{p_{12} N_1 + p_{22} N_2}\right) - \gamma_2 < 0 \\
  &\pdv{\mathbf{F}_{x_2}}{y_2} = \pdv{\dot{x}_2}{y_2} = \beta_{vh} (N_2-x_2(t)) \frac{p_{22} e^{-\mu_2 \tau} a_2 (1-\kappa u_2(t))}{p_{12} N_1 + p_{22} N_2} \geq 0 \\
  &\pdv{\mathbf{F}_{y_2}}{x_1} = \pdv{\dot{y}_2}{x_1} = \beta_{hv} a_2 (M_2-y_2(t)) \frac{p_{12} (1-\kappa u_1(t))}{p_{12} N_1 + p_{22} N_2} \geq 0 \\
  &\pdv{\mathbf{F}_{y_12}}{y_1} = \pdv{\dot{y}_2}{y_1} = 0 \\
  &\pdv{\mathbf{F}_{y_2}}{x_2} = \pdv{\dot{y}_2}{x_2} = \beta_{hv} a_2 (M_2-y_2(t)) \frac{p_{22} (1-\kappa u_2(t))}{p_{12} N_1 + p_{22} N_2} \geq 0 \\
  &\pdv{\mathbf{F}_{y_2}}{y_2} = \pdv{\dot{y}_2}{y_2} = -\beta_{hv} a_2 \frac{p_{12} (1-\kappa u_1) x_1(t) + p_{22} (1-\kappa u_2) x_2(t)}{p_{12} N_1 + p_{22} N_2} - \mu_2 < 0
\end{align*}

Извън главния диагонал има само неотрицателни елементи, тогава системата е кооперативна.

% \subsection{Кооперативност (квазимонотонност)}
% Доказваме квазимонотонността по дефиницията \color{Red} ДЕФИНИЦИЯ!!!
% \color{Black}
% \begin{align*}
%   &\pdv{\dot{x}_1}{y_1} = \beta_{vh} (N_1-x_1(t)) \frac{p_{11} e^{-\mu_1 \tau} a_1 (1-\kappa u_1(t))}{p_{11} N_1 + p_{21} N_2} \geq 0 \\
%   &\pdv{\dot{x}_1}{x_2} = 0 \\
%   &\pdv{\dot{x}_1}{y_2} = \beta_{vh} (N_1-x_1(t))\frac{p_{12} e^{-\mu_2 \tau} a_2 (1-\kappa u_1(t))}{p_{12} N_1 + p_{22} N_2} \geq 0 \\
%   &\pdv{\dot{y}_1}{x_1} = \beta_{hv} a_1 (M_1-y_1(t)) \frac{p_{11} (1-\kappa u_1(t))}{p_{11} N_1 + p_{21} N_2} \geq 0 \\
%   &\pdv{\dot{y}_1}{x_2} = \beta_{hv} a_1 (M_1-y_1(t)) \frac{p_{21} (1-\kappa u_2(t))}{p_{11} N_1 + p_{21} N_2} \geq 0 \\
%   &\pdv{\dot{y}_1}{y_2} = 0 \\
%   &\pdv{\dot{x}_2}{x_1} = 0 \\
%   &\pdv{\dot{x}_2}{y_1} = \beta_{vh} (N_2-x_2(t)) \frac{p_{21} e^{-\mu_1 \tau} a_1 (1-\kappa u_2(t))}{p_{11} N_1 + p_{21} N_2} \geq 0 \\
%   &\pdv{\dot{x}_2}{y_2} = \beta_{vh} (N_2-x_2(t)) \frac{p_{22} e^{-\mu_2 \tau} a_2 (1-\kappa u_2(t))}{p_{12} N_1 + p_{22} N_2} \geq 0 \\
%   &\pdv{\dot{y}_2}{x_1} = \beta_{hv} a_2 (M_2-y_2(t)) \frac{p_{12} (1-\kappa u_1(t))}{p_{12} N_1 + p_{22} N_2} \geq 0 \\
%   &\pdv{\dot{y}_2}{y_1} = 0 \\
%   &\pdv{\dot{y}_2}{x_2} = \beta_{hv} a_2 (M_2-y_2(t)) \frac{p_{22} (1-\kappa u_2(t))}{p_{12} N_1 + p_{22} N_2} \geq 0 \\
% \end{align*}

\subsection{Неподвижни точки}
Да разгледаме $\mathbf{u}(t)=\bar{\mathbf{u}}$. Веднага се вижда, че $\mathbf{F}(\mathbf{0}, \mathbf{u}) = \mathbf{0}$, т.е. $\mathbf{0}$ е тривиалната неподвижна точка на системата.
Сега разглеждаме за фиксирано укравление $\mathbf{u}(t) \equiv \mathbf{u}$.

Ще използваме теоремата на Смит за автономни системи.
% Да изведем матрицата на Якоби $\mathrm{D} \mathbf{F}(x_1, y_1, x_2, y_2)(t)$:
% \begin{align*}
%   &\mathrm{D} \mathbf{F}(x_1, y_1, x_2, y_2)(t) = \\
%   &
%   \begin{tiny}
%     \begin{pmatrix}
%       - \beta_{vh} \left(\frac{p_{11} e^{-\mu_1 \tau} a_1 (1-\kappa u_1) y_1(t)}{p_{11} N_1 + p_{21} N_2} + \frac{p_{12} e^{-\mu_2 \tau} a_2 (1-\kappa u_1) y_2(t)}{p_{12} N_1 + p_{22} N_2}\right) - \gamma_1 && \beta_{vh} (N_1-x_1(t)) \frac{p_{11} e^{-\mu_1 \tau} a_1 (1-\kappa u_1)}{p_{11} N_1 + p_{21} N_2} && 0 && \beta_{vh} (N_1-x_1(t)) \frac{p_{12} e^{-\mu_2 \tau} a_2 (1-\kappa u_1)}{p_{12} N_1 + p_{22} N_2} \\
%       \beta_{hv} a_1 (M_1-y_1(t)) \frac{p_{11} (1-\kappa u_1)}{p_{11} N_1 + p_{21} N_2} && -\beta_{hv} a_1 \frac{p_{11} (1-\kappa u_1) x_1(t) + p_{21} (1-\kappa u_2) x_2(t)}{p_{11} N_1 + p_{21} N_2} - \mu_1 && \beta_{hv} a_1 (M_1-y_1(t)) \frac{p_{21} (1-\kappa u_2)}{p_{11} N_1 + p_{21} N_2}&& 0 \\
%       0 && \beta_{vh} (N_2-x_2(t)) \frac{p_{21} e^{-\mu_1 \tau} a_1 (1-\kappa u_2)}{p_{11} N_1 + p_{21} N_2} && -\beta_{vh} \left(\frac{p_{21} e^{-\mu_1 \tau} a_1 (1-\kappa u_2) y_1(t)}{p_{11} N_1 + p_{21} N_2} + \frac{p_{22} e^{-\mu_2 \tau} a_2 (1-\kappa u_2) y_2(t)}{p_{12} N_1 + p_{22} N_2}\right) - \gamma_2 && \beta_{vh} (N_2-x_2(t)) \frac{p_{22} e^{-\mu_2 \tau} a_2 (1-\kappa u_2) y_2(t)}{p_{12} N_1 + p_{22} N_2} \\
%       \beta_{hv} a_2 (M_2-y_2(t)) \frac{p_{12} (1-\kappa u_1)}{p_{12} N_1 + p_{22} N_2} && 0 && \beta_{hv} a_2 (M_2-y_2(t)) \frac{p_{22} (1-\kappa u_2) x_2(t)}{p_{12} N_1 + p_{22} N_2} && -\beta_{hv} a_2 \frac{p_{12} (1-\kappa u_1) x_1(t) + p_{22} (1-\kappa u_2) x_2(t)}{p_{12} N_1 + p_{22} N_2} - \mu_2 \\
%     \end{pmatrix}
%   \end{tiny}
% \end{align*}

% % Да разгледаме $A_{ij}=\frac{a_j b_j p_ij e^{-\mu_j \tau_j}}{H_j}$, $B_{ij}=\frac{a_j c_j \delta_{ij}}{H_j}$
% \color{Red} ДА СЕ ОПРАВЯТ ЗНАМЕНАТЕЛИТЕ!!!
% \color{Black}
% Да означим $A_{ij}=\frac{a_j b_j p_ij e^{-\mu_j \tau_j}}{H_j}$, $B_{ij}=\frac{a_j c_j p_ji}{H_j}$. Тогава системата ни добива точно вида $(2.4)$ от \color{Red} ЦИТАТ Cosner, обаче с оправените Bichara?!!!
% \color{Black}
% и ползвайки Теорема 1 от същата статия с $\cap{A}=(\frac{A_{ij} N_i}{\gamma_i})$, то ако зададем $R_0^2=\rho(\cap{А}\cap{B})$,
При $R_0 \leq 1$, $\mathbf{0}$ е единствена устойчива неподвижна точка, а при $R_0 > 1$, то $\mathbf{0}$ е неустойчива неподвижна точка и съществува точно една друга устойчива, намираща се във вътрешността на $X$.

\subsection{Система на Marchaud/Peano}
За да има търсената задача решение, то трябва ядрото на допустимост $\eqref{eq:ViabilityKernel}$ да съществува и да не е празно. От \color{Red} ЦИТАТ!!!
\color{Black} за съществуването е необходимо да покажем, че $\mathscr{F}(\mathbf{z}(t))=\mathbf(F)(\mathbf{z}(t), U)$ е изображение на Marchaud/Peano.

$\mathscr{F}$ e изображение на Marchaud/Peano, ако е нетривиално, отгоре полунепрекъснато, с компактни изпъкнали образи и с линейно нарастване.

Стига параметрите ни да не са всички нулеви, то $\mathscr{F}$ е нетривиално. \\
От факта, че $\mathbf{F}$ е непрекъснато по всяка компонента (а даже и диференцируемо), то е и отгоре полунепрекъснато, откъдето и $\mathscr{F}$ е. \\
$X$ е затворено и ограничено, а е крайномерно, съответно е компактно. Аналогично за $U$. Тогава и $X \cross U$ е компактно. Директно от дефинициите на $X$, $U$ те също така са изпъкнали, тоест $X \cross U$ е изпъкнало. Вече видяхме, че $X$ положително инвариантно за системата. Тогава образите на $\mathscr{F}$ ще са в $X \cross U$, с други думи са компактни и изпъкнали. \\

\subsubsection{Линейно нарастване}
За да покажем линейното нарастване, да забележим, че $\mathbf{F}(\mathbf{0}) = \mathbf{0}$. Тогава може да запишем:
\begin{align*}
  &\|\mathbf{F}(x_1, y_1, x_2, y_2, u_1, u_2)\| = \|\mathbf{F}(x_1, y_1, x_2, y_2, u_1, u_2) - \mathbf{0}\| = \|\mathbf{F}(x_1, y_1, x_2, y_2, u_1, u_2) - \mathbf{F}(\mathbf{0})\| = \\
  &\frac{\beta_{vh} a_1 p_{11} e^{-\mu_1 \tau}}{p_{11} N_1 + p_{21} N_2} (2 N_1 (2 |y_1| + M_1 \kappa |u_1|) + M_1 |x_1|) + \\
  &\frac{\beta_{vh} a_2 p_{12} e^{-\mu_2 \tau}}{p_{12} N_1 + p_{22} N_2}(2 N_1 (2 |y_2| + M_2 \kappa |u_1|) + M_1 |x_1|) + \gamma_1 |x_1| + \\
  &\frac{\beta_{hv} a_1 p_{11}}{p_{11} N_1 + p_{21} N_2} (2 M_1 (2|x_1| + N_1 \kappa |u_1|) + N_1 |y_1|) + \\
  &\frac{\beta_{hv} a_1 p_{21}}{p_{11} N_1 + p_{21} N_2} (2 M_1 (2|x_2| + N_2 \kappa |u_2|) + N_2 |y_1|) + \mu_1 |y_1| + \\
  &\frac{\beta_{vh} a_1 p_{21} e^{-\mu_1 \tau}}{p_{11} N_1 + p_{21} N_2} (2 N_1 (2|y_1| + M_1 \kappa |u_1|) + M_1 |x_1|) + \\
  &\frac{\beta_{vh} a_2 p_{22} e^{-\mu_2 \tau}}{p_{12} N_1 + p_{22} N_2}(2 N_1 (2|y_2| + M_2 \kappa |u_1|) + M_1 |x_1|) + \gamma_2 |x_2| + \\
  &\frac{\beta_{hv} a_2 p_{12}}{p_{12} N_1 + p_{22} N_2} (2 M_2 (2|x_1| + N_1 \kappa |u_1|) + N_1 |y_2|) + \\
  &\frac{\beta_{hv} a_2 p_{22}}{p_{12} N_1 + p_{22} N_2} (2 M_2 (2|x_2| + N_2 \kappa |u_2|) + N_2 |y_2|) + \mu_2 |y_2| \leq \\
  &\tilde{C}_1|u_1| + \tilde{C}_2|u_2| + \tilde{C}_3 \|(x_1,y_1,x_2,y_2)\| \leq \tilde{C}_1 + \tilde{C}_2 + \tilde{C}_3 \|(x_1,y_1,x_2,y_2)\| \leq \tilde{C}(1 + \|(x_1,y_1,x_2,y_2)\|)
\end{align*}

Така $V(\bar{\mathbf{I}}, \bar{\mathbf{u}})$ съществува. Веднага може да видим, че $V(\bar{\mathbf{I}}, \bar{\mathbf{u}}) \neq \emptyset$, понеже $\mathbf{0}$ е неподвижна точка за кое да е управление и следователно $\mathbf{0} \in V(\bar{\mathbf{I}}, \bar{\mathbf{u}})$.
