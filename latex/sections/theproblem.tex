\section{Модел на Ross-Macdonald с две местообитания и репелент}
Задачата която се изследва в дипломната работа е:
\begin{align*}
  \label{eq:TheProblem} \tag{\text{Система} 1}
  &\dot{x}_1(t) = \beta_{vh} (N_1-x_1(t)) \left(\frac{p_{11} e^{-\mu_1 \tau} a_1 (1-\kappa u_1(t)) y_1(t)}{p_{11} N_1 + p_{21} N_2} + \frac{p_{12} e^{-\mu_2 \tau} a_2 (1-\kappa u_1(t)) y_2(t)}{p_{12} N_1 + p_{22} N_2}\right) - \gamma_1 x_1(t) \\
  &\dot{y}_1(t) = \beta_{hv} a_1 (M_1-y_1(t)) \frac{p_{11} (1-\kappa u_1(t)) x_1(t) + p_{21} (1-\kappa u_2(t)) x_2(t)}{p_{11} N_1 + p_{21} N_2} - \mu_1 y_1(t) \\
  &\dot{x}_2(t) = \beta_{vh} (N_2-x_2(t)) \left(\frac{p_{21} e^{-\mu_1 \tau} a_1 (1-\kappa u_2(t)) y_1(t)}{p_{11} N_1 + p_{21} N_2} + \frac{p_{22} e^{-\mu_2 \tau} a_2 (1-\kappa u_2(t)) y_2(t)}{p_{12} N_1 + p_{22} N_2}\right) - \gamma_2 x_2(t) \\
  &\dot{y}_2(t) = \beta_{hv} a_2 (M_2-y_2(t)) \frac{p_{12} (1-\kappa u_1(t)) x_1(t) + p_{22} (1-\kappa u_2(t)) x_2(t)}{p_{12} N_1 + p_{22} N_2} - \mu_2 y_2(t)
\end{align*}

Това е обезразмерен модел, където са смесини моделите за миграция и за репелент против комари.
$t$ е времето, като ще разглеждаме само $t \in [0, \inf)$. \\
$x_1, x_2 \in [0, 1]$ са частта от заразени хора, а $y_1, y_2 \in [0, 1]$ - частта на заразените комари в локации 1 и 2 съответно. \\
$u_1 \in [0, \bar{u}_1], u_2 \in [0, \bar{u}_2]$ са управленията отговарящи за това каква част от хората от съответното местообитание са предпазени от репелента, като $ \bar{u}_1, \bar{u}_2 \leq 1 $ отговарят за максималната предпазена част от населението, вследствие от производствената способност. Надолу се бележи $\mathcal{U} = [0, \bar{u}_1] \cross [0, \bar{u}_2]$ \\
$\kappa \in [0, 1]$ е константа, която представя ефективността на репелента (т.е. предполагаме че едно и също вещество/метод се използва и на двете места). \\
$p_11, p_12, p_21, p_22 \in [0, 0.5]$, $p_11 + p_12 = 1, p_21 + p_22 = 1$ са константи, отговарящи за миграцията, като $p_ij$ е частта от хора от $i$, които пребивават временно в $j$. \\
$\gamma_1, \gamma_2$ са смъртностите на хората, а $\mu_1, \mu_2$ - на комарите, които приемаме за константи. \\
$\tau$ е константа на средното време, за което комарите стават заразни. \\
$\alpha_1, \alpha_2$ са константи, които бележат колко ухапвания на комари има за единица време. \\
$\beta_{vh}$ е констатнтана заразност, когато заразен комар ухапва човек, а $\beta_{hv}$ е констатнтана заразност, комар ухапва заразен човек.
$\bar{I}_1, \bar{I}_2 \in [0, 1]$ са константи, отговарящи за максималната част от населението в съответното местообитание, което може да получи адекватна здравна помощ при заразяване с малария.

\color{Red} ИЗВЕЖДАНЕ НА МОДЕЛА!!!
\color{Black}

Питаме се има ли такива управления $u_1, u_2$, за които във всеки момент всички заразени да имат възможност да получат помощ от здравната система, т.е. :
\begin{equation}
  \label{eq:AllHospitalised}
  \forall t>0 (x_1(t) \leq \bar{I}_1 \wedge x_2(t) \leq \bar{I}_2)
\end{equation}
Тъй като първоначалният брой заразени хора и комари влияят на развитието на системата ще търсим:
\begin{equation}
  \label{eq:ViabilityKernel}
  V(\bar{\mathbf{I}}, \bar{\mathbf{u}}) = \{(x_1^0, y_1^0, x_2^0, y_2^0) | \exists \mathbf{u} (\eqref{eq:TheProblem} \text{ има решение} \wedge \eqref{eq:AllHospitalised} \text{ е изпълнено})\}
\end{equation}
$V(\bar{\mathbf{I}}, \bar{\mathbf{u}})$ се нарича ядро на допустимост.
