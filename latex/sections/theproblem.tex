\section{Модел на Ross-Macdonald с две местообитания и репелент}
Задачата която се изследва в дипломната работа е:
\begin{equation}
  \label{eq:TheProblem}
  \begin{split}
    &\dot{x}_1(t) = \beta_{vh} (N_1-x_1(t)) \left(\frac{p_{11} e^{-\mu_1 \tau} a_1 (1-\kappa u_1(t)) y_1(t)}{p_{11} N_1 + p_{21} N_2} + \frac{p_{12} e^{-\mu_2 \tau} a_2 (1-\kappa u_1(t)) y_2(t)}{p_{12} N_1 + p_{22} N_2}\right) - \gamma_1 x_1(t) \\
    &\dot{y}_1(t) = \beta_{hv} a_1 (M_1-y_1(t)) \frac{p_{11} (1-\kappa u_1(t)) x_1(t) + p_{21} (1-\kappa u_2(t)) x_2(t)}{p_{11} N_1 + p_{21} N_2} - \mu_1 y_1(t) \\
    &\dot{x}_2(t) = \beta_{vh} (N_2-x_2(t)) \left(\frac{p_{21} e^{-\mu_1 \tau} a_1 (1-\kappa u_2(t)) y_1(t)}{p_{11} N_1 + p_{21} N_2} + \frac{p_{22} e^{-\mu_2 \tau} a_2 (1-\kappa u_2(t)) y_2(t)}{p_{12} N_1 + p_{22} N_2}\right) - \gamma_2 x_2(t) \\
    &\dot{y}_2(t) = \beta_{hv} a_2 (M_2-y_2(t)) \frac{p_{12} (1-\kappa u_1(t)) x_1(t) + p_{22} (1-\kappa u_2(t)) x_2(t)}{p_{12} N_1 + p_{22} N_2} - \mu_2 y_2(t)
  \end{split}
  \end{equation}

Това е модел обендинение на моделите за мобилност и за репелент против комари.
$t$ е времето, като ще разглеждаме само $t \in [0, \infty)$. \\
$x_1, x_2 \in [0, N_i]$ са броят заразени хора, а $y_1, y_2 \in [0, M_i]$ - броят заразени комари в локации 1 и 2 съответно. \\
$u_1 \in [0, \bar{u}_1], u_2 \in [0, \bar{u}_2]$ са управленията отговарящи за това каква част от хората от съответното местообитание са предпазени от репелента, като $ \bar{u}_1, \bar{u}_2 \leq 1 $ отговарят за максималната предпазена част от населението, вследствие от производствената способност. Надолу се бележи $\mathcal{U} = [0, \bar{u}_1] \cross [0, \bar{u}_2]$ и $\bar{\mathbf{u}} = (\bar{u}_1, \bar{u}_2)$ \\
$\kappa \in [0, 1]$ е константа, която представя ефективността на репелента (т.е. предполагаме че едно и също вещество/метод се използва и на двете места). \\
$p_{11}, p_{12}, p_{21}, p_{22} \in [0, 0.5]$, $p_{11} + p_{12} = 1, p_{21} + p_{22} = 1$ са константи, отговарящи за мобилността, като $p_{ij}$ е частта от хора от $i$, които пребивават временно в $j$. \\
$\gamma_1, \gamma_2$ са скоростите на оздравяване на хората, а $\mu_1, \mu_2$ - скоростите на смъртност на комарите, които приемаме за константи. \\
$\tau$ е константа на средното време, за което комарите стават заразни. \\
$\alpha_1, \alpha_2$ са константи, които бележат колко ухапвания на комари има за единица време. \\
$\beta_{vh}$ е константна вероятност за заразяване на здрав човек с патогена, когато бъде ухапан от заразен комар, а $\beta_{hv}$ е константна вероятност за заразяване на здрав комар с патогена, когато ухапе заразен човек.

\begin{table}[h!]
  \centering
  \caption{Таблица с параметри на модела}
  \begin{tabular}{ |c c c|  }
    \hline
    Параметър & Описание & Мерни единици\\
    \hline
    $\beta_{hv}$ & Вероятност на прехвърляне на патогена от човек на комар & безразмерен\\
    $\beta_{vh}$ & Вероятност на прехвърляне на патогена от комар на човек & безразмерен\\
    $a_i$ & Честота на ухапвания & $\text{ден}^{-1}$\\
    $M_i$ & Популация на женски комари & комар\\
    $\mu_i$ & Смъртност на комари & $\text{ден}^{-1}$\\
    $\tau$ & Инкубационен период при комарите & ден\\
    $N_i$ & Човешка популация & човек\\
    $\gamma_i$ & Скорост на оздравяване на хора & $\text{ден}^{-1}$\\
    $p_{ij}$ & Мобилност на хора от местообитание i в j & безразмерна\\
    $\kappa$ & Ефективност на репелент & безразмерен\\
    $\bar{u}_i$ & Максимална възможна предпазена част хора с репелента & безразмерен\\
    $\bar{I}_i$ & Максимална част на заразени хора & безразмерен\\
    \hline
  \end{tabular}
  \label{table:1}
\end{table}

Моделът подлежи на скалиране на променливите чрез смяната $(x_1, y_1, x_2, y_2) \rightarrow (\frac{x_1}{N_1}, \frac{y_1}{M_1}, \frac{x_2}{N_2}, \frac{y_2}{M_2})$. След това достигаме до:
\begin{equation}
  \label{eq:TheDimensionlessProblemFull}
  \begin{split}
    &\dot{x}_1(t) = \beta_{vh} (1-x_1(t)) \left(\frac{p_{11} e^{-\mu_1 \tau} a_1 (1-\kappa u_1(t)) M_1 y_1(t)}{p_{11} N_1 + p_{21} N_2} + \frac{p_{12} e^{-\mu_2 \tau} a_2 (1-\kappa u_1(t)) M_2 y_2(t)}{p_{12} N_1 + p_{22} N_2}\right) - \gamma_1 x_1(t) \\
    &\dot{y}_1(t) = \beta_{hv} a_1 (1-y_1(t)) \frac{p_{11} (1-\kappa u_1(t)) N_1 x_1(t) + p_{21} (1-\kappa u_2(t)) N_2 x_2(t)}{p_{11} N_1 + p_{21} N_2} - \mu_1 y_1(t) \\
    &\dot{x}_2(t) = \beta_{vh} (1-x_2(t)) \left(\frac{p_{21} e^{-\mu_1 \tau} a_1 (1-\kappa u_2(t)) M_1 y_1(t)}{p_{11} N_1 + p_{21} N_2} + \frac{p_{22} e^{-\mu_2 \tau} a_2 (1-\kappa u_2(t)) M_2 y_2(t)}{p_{12} N_1 + p_{22} N_2}\right) - \gamma_2 x_2(t) \\
    &\dot{y}_2(t) = \beta_{hv} a_2 (1-y_2(t)) \frac{p_{12} (1-\kappa u_1(t)) N_1 x_1(t) + p_{22} (1-\kappa u_2(t)) N_2 x_2(t)}{p_{12} N_1 + p_{22} N_2} - \mu_2 y_2(t)
  \end{split}
\end{equation}

В този модел за улеснение на по-нататъшни изрази може да направим полагането:
\begin{equation}
  \begin{split}
    &b_{11} = \beta_{vh} \frac{p_{11} e^{-\mu_1 \tau} a_1 M_1}{p_{11} N_1 + p_{21} N_2} \\
    &b_{12} = \beta_{vh} \frac{p_{12} e^{-\mu_2 \tau} a_2 M_2}{p_{12} N_1 + p_{22} N_2} \\
    &b_{21} = \beta_{vh} \frac{p_{21} e^{-\mu_1 \tau} a_1 (1-\kappa u_2(t)) M_1}{p_{11} N_1 + p_{21} N_2} \\
    &b_{22} = \beta_{vh} \frac{p_{22} e^{-\mu_2 \tau} a_2 (1-\kappa u_2(t)) M_2}{p_{12} N_1 + p_{22} N_2} \\
    &c_{11} = \beta_{hv} a_1 \frac{p_{11} N_1}{p_{11} N_1 + p_{21} N_2} \\
    &c_{11} = \beta_{hv} a_1 \frac{p_{21} N_2}{p_{11} N_1 + p_{21} N_2} \\
    &c_{21} = \beta_{hv} a_2 \frac{p_{12} N_1}{p_{11} N_1 + p_{21} N_2} \\
    &c_{22} = \beta_{hv} a_2 \frac{p_{22} N_2}{p_{12} N_1 + p_{22} N_2} \\
  \end{split}
\end{equation}

Крайният вид на обезразмерения модел е:
\begin{equation}
  \label{eq:TheDimensionlessProblem}
  \begin{split}
    &\dot{x}_1(t) = \beta_{vh} (1-x_1(t)) (1-\kappa u_1(t)) \left(b_{11} y_1(t) + b_{12} y_2(t)\right) - \gamma_1 x_1(t) \\
    &\dot{y}_1(t) = (1-y_1(t)) \left(c_{11}(1-\kappa u_1(t)) x_1(t) + c_{12}(1-\kappa u_2(t)) x_2(t)\right) - \mu_1 y_1(t) \\
    &\dot{x}_2(t) = \beta_{vh} (1-x_2(t)) (1-\kappa u_2(t))\left(b_{21} y_1(t) + b_{22} y_2(t)\right) - \gamma_2 x_2(t) \\
    &\dot{y}_2(t) = (1-y_2(t)) \left(c_{21}(1-\kappa u_1(t)) x_1(t) + c_{22} (1-\kappa u_2(t)) x_2(t)\right) - \mu_2 y_2(t)
  \end{split}
\end{equation}

Нека $\bar{I}_1, \bar{I}_2 \in [0, 1]$ са константи, отговарящи за максималната част от населението в съответното местообитание, което може да получи адекватна здравна помощ при заразяване с малария.
Питаме се има ли такива управления $u_1(t), u_2(t)$, за които във всеки момент всички заразени да имат възможност да получат помощ от здравната система, т.е. :
\begin{equation}
  \label{eq:AllHospitalised}
  \forall t>0 (x_1(t) \leq \bar{I}_1 \wedge x_2(t) \leq \bar{I}_2)
\end{equation}
Тъй като първоначалният брой заразени хора и комари влияят на развитието на системата ще търсим:
\begin{equation}
  \label{eq:ViabilityKernel}
  V(\bar{\mathbf{I}}, \bar{\mathbf{u}}) = \{(x_1^0, y_1^0, x_2^0, y_2^0) | \exists \mathbf{u} (\eqref{eq:TheProblem} \text{ има решение} \wedge \eqref{eq:AllHospitalised} \text{ е изпълнено})\}
\end{equation}
$V(\bar{\mathbf{I}}, \bar{\mathbf{u}})$ се нарича ядро на слаба инвариантност на Белман.
