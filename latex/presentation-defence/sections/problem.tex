\section{\hspace{1em}Модел}

\begin{frame}[t]{Схема на модела}
  \begin{figure}{}
    \centering
    % \hspace{-1em}
    \begin{tikzpicture}[node distance=15mm]
      % \begin{pgfinterruptboundingbox}
      %   \begin{axis}[ ymin=0, xlabel = variáveis aleatórias, ylabel = frequência]

      \node[label={[xshift=-9mm]\small\it хора}] (sh1) [hostp,text width=18mm] {податливи};

      \node (ih1) [hostp, below of=sh1, text width=18mm] {заразени ($X_1$)};

      \node[label={[xshift=-5mm] \small\it комари}] (sv1) [vectorp, left of=sh1, xshift=-10mm, text width=18mm] {податливи};

      \node (iv1) [vectorp, below of=sv1, text width=18mm] {заразени ($Y_1$)};

      \draw [arrow2] (sh1) -- (ih1);

      \draw [arrow2] (ih1) -- (sh1);

      \draw [arrow] (iv1) -- (sh1) ;

      \draw [arrow] (ih1) -- (sv1) ;

      \draw [arrow2] (sv1) -- (iv1);

      \node[label={[label distance=3mm]-90: местообитание 1},draw=black, fit=(sh1)(iv1),inner sep=5mm, rounded corners](FIt1) {};

      % patch 2

      \node[label={[xshift=-7mm]\small\it хора}] (sh2) [hostp, right of=sh1,xshift=17mm,text width=18mm] {податливи};

      % right
      \node (ih2) [hostp, below of=sh2, text width=18mm] {заразени ($X_2$)};

      \node[label={[xshift=-5mm] \small\it комари}] (sv2) [vectorp, right of=sh2, xshift=10mm, text width=18mm] {податливи};

      \node (iv2) [vectorp, below of=sv2, text width=18mm] {заразени ($Y_2$)};

      \draw [arrow2] (sh2) -- (ih2);

      \draw [arrow2] (ih2) -- (sh2);

      \draw [arrow] (iv2) -- (sh2) ;

      \draw [arrow] (ih2) -- (sv2) ;

      \draw [arrow2] (sv2) -- (iv2);

      \node[label={[label distance=3mm]-90: местообитание 2},draw=black, fit=(sh2)(iv2),inner sep=5mm, rounded corners](FIt2) {};

      % arrows between patches
      \draw [arrow3] (sh1) to (sh2) ;
      \draw [arrow3] (ih1) to (ih2) ;

      % \end{axis}
      % \end{pgfinterruptboundingbox}
      % \path[use as bounding box] ([yshift=-8mm]current axis.south west) rectangle (current axis.north east);
    \end{tikzpicture}
    \centering
    % \captionof{figure}{Диаграма на модела \eqref{eq:MigrationProblem}. \\
    % Черна пунктирана линия: възможен преход на индивид от класа в началото в класа в края.\\
    % Черна непрекъсната линия: индивид от началото може да зарази индивид от края.\\
    % Синя линия: мобилност}
    % Using caption doesn't result in centered text ???
    \caption{Черна пунктирана линия: възможен преход на индивид от класа в началото в класа в края.\\
      Черна непрекъсната линия: индивид от началото може да зарази индивид от края.\\
    Синя линия: мобилност}
    \label{fig:TwoPatchModel}
  \end{figure}
\end{frame}

% \begin{frame}[t]{Допускания}
%   % В дипломната работа се разглежда модел на Ross-Macdonald с две местообитания и репелент. Той носи следните допускания:
%   \begin{small}
%     \begin{enumerate}
%       \item Човешката популация и популацията комари е постоянна, като са разпределени равномерно в местообитанията.
%       \item Смъртността от заразата се пренебрегва, както при хората, така и при комарите.
%       \item Веднъж заразени, комарите не се възстановяват.
%       \item Само податливи се заразяват (няма свръхзаразяване).
%       \item Хората не придобиват никакъв имунитет.
%       \item Смъртността на комарите е независима от възрастта им и съответно продължителността им на живот е експоненциално разпределена.
%       \item Разглежда се само предаването на патогена между хора и комари.
%       \item Латентният период при комарите е константен.
%       \item Мобилността на хората между местообитанията е константна.
%       \item Репелентът има линеен ефект върху честотата на ухапванията върху предпазените с него и не увеличава тази върху непредпазените.
%     \end{enumerate}
%   \end{small}
% \end{frame}

\begin{frame}[t]{Означения}
  \vspace{-1em}
  \begin{table}
    \begin{small}
      \begin{table}[h]
        \centering
        \begin{tabular}{ |c||c| }
          \hline
          Променлива & Описание\\
          \hline
          $t$ & Време [$\text{ден}$]\\
          $X_i(t)$ & Брой заразени жители\\
          $Y_i(t)$ & Брой заразени комари \\
          $u_i(t)$ & Пропорция защитени с репелент жители \\
          \hline
          \hline
          Параметър & Описание\\
          \hline
          $\beta_{vh}$ & Вероятност на прехвърляне на патогена от комар на човек \\
          $\beta_{hv}$ & Вероятност на прехвърляне на патогена от човек на комар \\
          $a_i$ & Честота на ухапвания [$\text{ден}^{-1}$]\\
          $M_i$ & Популация на женски комари\\
          $\mu_i$ & Смъртност на комари [$\text{ден}^{-1}$]\\
          $\tau$ & Инкубационен период при комарите [$\text{ден}$]\\
          $N_i$ & Човешка популация (население) \\
          $\gamma_i$ & Скорост на оздравяване на хора [$\text{ден}^{-1}$]\\
          $p_{ij}$ & Мобилност на хора от местообитание i в j\\
          $\kappa$ & Ефективност на репелент\\
          $\bar{u}_i$ & Максимална възможна предпазена част жители с репелент\\
          $\bar{I}_i$ & Максимална част на заразени хора\\
          \hline
        \end{tabular}
      \end{table}
    \end{small}
    \caption{Таблица с променливи и параметри}
    \label{tbl:Definitions}
  \end{table}
\end{frame}

\begin{frame}[t]{Уравнения на модела}
  \begin{footnotesize}
    \begin{equation}
      \label{eq:TheProblem}
      \begin{split}
        &\dot{X}_1 = \beta_{vh} (N_1-X_1) \mathcolor{red}{(1-\kappa u_1)} \left(\frac{p_{11} e^{-\mu_1 \tau} a_1 Y_1}{p_{11} N_1 + p_{21} N_2} + \frac{p_{12} e^{-\mu_2 \tau} a_2  Y_2}{p_{12} N_1 + p_{22} N_2}\right) - \gamma_1 X_1 \\
        &\dot{X}_2 = \beta_{vh} (N_2-X_2) \mathcolor{red}{(1-\kappa u_2)} \left(\frac{p_{21} e^{-\mu_1 \tau} a_1 Y_1}{p_{11} N_1 + p_{21} N_2} + \frac{p_{22} e^{-\mu_2 \tau} a_2 Y_2}{p_{12} N_1 + p_{22} N_2}\right) - \gamma_2 X_2 \\
        &\dot{Y}_1 = \beta_{hv} a_1 (M_1-Y_1) \frac{p_{11} \mathcolor{red}{(1-\kappa u_1)} X_1 + p_{21} \mathcolor{red}{(1-\kappa u_2)} X_2}{p_{11} N_1 + p_{21} N_2} - \mu_1 Y_1 \\
        &\dot{Y}_2 = \beta_{hv} a_2 (M_2-Y_2) \frac{p_{12} \mathcolor{red}{(1-\kappa u_1)} X_1 + p_{22} \mathcolor{red}{(1-\kappa u_2)} X_2}{p_{12} N_1 + p_{22} N_2} - \mu_2 Y_2 \\
        &u_i \in \mathscr{U}_i = \{u_i:\mathbb{R}_+ \rightarrow [0, \bar{u}_i] \vert u_i \text{- измерима}\}
      \end{split}
    \end{equation}
  \end{footnotesize}
  Моделът се основава на Bichara \cite{Bichara2016} с добавена употреба на репелент \cite{Rashkov2022}.
\end{frame}

\begin{frame}[t]{Скалирана форма на модела}
  Моделът подлежи на скалиране на променливите чрез смяната:
  \begin{equation*}
    (X_1, X_2, Y_1, Y_2)^T \rightarrow \left(\frac{X_1}{N_1}, \frac{X_2}{N_2}, \frac{Y_1}{M_1}, \frac{Y_2}{M_2}\right)^T = (x_1, x_2, y_1, y_2)^T
  \end{equation*}
  След полагания на коефициентите има вида:
  \begin{equation}
    \label{eq:TheDimensionlessProblem}
    \begin{split}
      &\dot{x}_1 = (1-x_1) (1-\kappa u_1) \left(b_{11} y_1 + b_{12} y_2\right) - \gamma_1 x_1 \\
      &\dot{x}_2 = (1-x_2) (1-\kappa u_2)\left(b_{21} y_1 + b_{22} y_2\right) - \gamma_2 x_2 \\
      &\dot{y}_1 = (1-y_1) \left(c_{11}(1-\kappa u_1) x_1 + c_{12}(1-\kappa u_2) x_2\right) - \mu_1 y_1 \\
      &\dot{y}_2 = (1-y_2) \left(c_{21}(1-\kappa u_1) x_1 + c_{22} (1-\kappa u_2) x_2\right) - \mu_2 y_2 \\
    \end{split}
  \end{equation}
\end{frame}

\begin{frame}{Допълнителни означения}
  % \eqref{eq:TheDimensionlessProblem} ще се записва и във векторен вид по следния начин:
  % \begin{equation*}
  %   \begin{pmatrix}
  %     \dot{\boldsymbol{x}} \\
  %     \dot{\boldsymbol{y}}
  %   \end{pmatrix}
  %   =
  %   \boldsymbol{f}(\boldsymbol{x}, \boldsymbol{y}, \boldsymbol{u}), \Hquad
  %   \boldsymbol{x} = (x_1, x_2)^T, \Hquad \boldsymbol{y} = (y_1, y_2)^T, \Hquad \boldsymbol{f}=(f_{x_1}, f_{x_2}, f_{y_1}, f_{y_2})^T
  % \end{equation*}
  \eqref{eq:TheDimensionlessProblem} ще се записва във векторен вид по следния начин:
  \begin{equation*}
    \dot{\boldsymbol{z}} = \boldsymbol{f}(\boldsymbol{z}, \boldsymbol{u}), \Hquad \boldsymbol{z} = (\boldsymbol{x}, \boldsymbol{y})^T, \Hquad \boldsymbol{z}(0) = \boldsymbol{z}_0 = (x_1^0, x_2^0, y_1^0, y_2^0)^T
  \end{equation*}

Задачата се разглежда в:
  \begin{equation*}
    \Omega = \{x_i \in [0, 1], y_i \in [0, 1]\} = \{\boldsymbol{z} \in [0, 1]^4\}
    \end{equation*}

  Означаваме:
  \begin{equation*}
    \begin{split}
      &U = [0, \bar{u}_1] \cross [0, \bar{u}_2] \\
      &\mathscr{U} = \mathscr{U}_1 \cross \mathscr{U}_2
    \end{split}
  \end{equation*}

  % \begin{nota-bene}
  %   Всичките векторни/матрични неравенства се разбират покомпоненто.
  % \end{nota-bene}
\end{frame}

\begin{frame}{Задача за здравна политика}
  $\bar{I}_i \in [0, 1]$ - максималната част от населението в съответното местообитание, което може да получи адекватна здравна помощ при заразяване с малария.
  \begin{equation*}
    \bar{\boldsymbol{I}} = (\bar{I}_1, \bar{I}_2)^T, \quad \mathscr{I} = [0, \bar{I}_1] \times [0, \bar{I}_2] \times [0, 1]^2.
  \end{equation*}

  Питаме се има ли такива управления $\boldsymbol{u}$, за които във всеки момент всички заразени да имат възможност да получат помощ от здравната система, т.е. :
  \begin{equation}
    \label{eq:AllHospitalised}
    \forall t \geq 0 (x_1(t) \leq \bar{I}_1 \wedge x_2(t) \leq \bar{I}_2) \iff \forall t \geq 0 (\boldsymbol{z}(t) \in \mathscr{I})
  \end{equation}
  Тъй като първоначалният брой заразени хора и комари влияят на развитието на системата ще търсим \textbf{ядрото на слаба инвариантност на Белман}:
  \begin{equation}
    \label{eq:ViabilityKernel}
    V(\bar{\boldsymbol{I}}, \bar{\boldsymbol{u}}) = \{\boldsymbol{z}_0  \text{ начално условие} \vert \exists \boldsymbol{u} (\eqref{eq:AllHospitalised} \text{ е изпълнено})\}
  \end{equation}
\end{frame}
