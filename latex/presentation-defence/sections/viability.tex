\section{\hspace{1em}Задача за здравна политика}

\begin{frame}[t]{Екстремални свойства на $V(\bar{\boldsymbol{I}}, \bar{\boldsymbol{u}})$}
  % $\pmb{0}$ е равновесна за \eqref{eq:TheDimensionlessProblem} - за нея винаги е изпълнено \eqref{eq:AllHospitalised}, т.е.
  % $\pmb{0} \in V(\bar{\boldsymbol{I}}, \bar{\boldsymbol{u}})$.
  % Ако $\boldsymbol{z}_0 \notin \mathcal{I}$, то \eqref{eq:AllHospitalised} не е изпълнено и съответно $\boldsymbol{z}_0 \notin V(\bar{\boldsymbol{I}}, \bar{\boldsymbol{u}})$, т.е. $V(\bar{\boldsymbol{I}}, \bar{\boldsymbol{u}}) \subseteq \mathcal{I}$.

  $\pmb{0}$ е равновесна за \eqref{eq:TheDimensionlessProblem} $\implies \pmb{0} \in V(\bar{\boldsymbol{I}}, \bar{\boldsymbol{u}})$, т.е.  $ \{\pmb{0}\} \subseteq V(\bar{\boldsymbol{I}}, \bar{\boldsymbol{u}})$. \\
  $\boldsymbol{z}_0 \notin \mathcal{I} \implies \boldsymbol{z}_0 \notin V(\bar{\boldsymbol{I}}, \bar{\boldsymbol{u}})$, т.е. $V(\bar{\boldsymbol{I}}, \bar{\boldsymbol{u}}) \subseteq \mathcal{I}$.

  \begin{proposition}
    Ако съществува $\boldsymbol{E}^* = (x_1^*, x_2^*, y_1^*, y_2^*)^T$ за $\boldsymbol{u}(t) \equiv \bar{\boldsymbol{u}}$, като $x_1^* > \bar{I}_1$ или $x_2^* > \bar{I}_2$, то $V(\bar{\boldsymbol{I}}, \bar{\boldsymbol{u}}) = \{\pmb{0}\}$.
  \end{proposition}

  % Доказва се с кооперативността и непрекъснатостта на решението на \eqref{eq:TheDimensionlessProblem}.

  % \begin{proposition}
  %   Ако \eqref{eq:AllHospitalised} е изпълнено за решението на система \eqref{eq:TheDimensionlessProblem} с $\boldsymbol{u} \equiv \pmb{0}$ и начално условие $\boldsymbol{z}_0 = (\xi_1, \xi_2, 1, 1)^T$, то $\Xi = [0, \xi_1] \times [0, \xi_2] \times [0, 1]^2 \subseteq V(\bar{\boldsymbol{I}}, \bar{\boldsymbol{u}})$.
  %   Ако е изпълнено за $\xi_i = \bar{I}_i$, то $V(\bar{\boldsymbol{I}}, \bar{\boldsymbol{u}}) = \mathcal{I}$.
  % \end{proposition}
  
  \begin{proposition}
    Ако \eqref{eq:AllHospitalised} е изпълнено за решението на система \eqref{eq:TheDimensionlessProblem} с $\boldsymbol{u} \equiv \pmb{0}$ и начално условие $\boldsymbol{z}_0 = (\bar{I}_1, \bar{I}_2, 1, 1)^T$, то $V(\bar{\boldsymbol{I}}, \bar{\boldsymbol{u}}) = \mathcal{I}$.
  \end{proposition}

  % Доказва се с кооперативността на \eqref{eq:TheDimensionlessProblem}.
\end{frame}

\begin{frame}[c]{Вариационен подход за намиране на $V(\bar{\boldsymbol{I}}, \bar{\boldsymbol{u}})$}
  Дефинираме значна функция на разстоянието $\Gamma$ до $\partial \mathcal{I}$:
  \begin{equation*}
    \Gamma(\boldsymbol{z}) =
    \begin{cases}
      \displaystyle\inf_{\boldsymbol{z}' \in \mathcal{I}} \|\boldsymbol{z}-\boldsymbol{z}'\|, \quad \boldsymbol{z} \in \Omega \setminus \mathcal{I} \\
      -\displaystyle\inf_{\boldsymbol{z}' \in \Omega \setminus \mathcal{I}} \|\boldsymbol{z}-\boldsymbol{z}'\|, \quad \boldsymbol{z} \in \mathcal{I}
    \end{cases}
  \end{equation*}

  Фиксираме $\lambda>L>0$ ($L$ - константата на Липшиц за \eqref{eq:TheDimensionlessProblem}) и въвеждаме \textbf{функция на Белман} $v$:
  \begin{equation*}
    v(\boldsymbol{z}_0) = \inf_{\boldsymbol{u} \in \mathcal{U}} \sup_{t \in (0, +\infty)} e^{-\lambda t} \Gamma(\boldsymbol{z}(t; \boldsymbol{z}_0; \boldsymbol{u}))
  \end{equation*}

\end{frame}

\begin{frame}[c]{Вариационен подход за намиране на $V(\bar{\boldsymbol{I}}, \bar{\boldsymbol{u}})$}
  % Ако започнем с $\boldsymbol{z}_0 \notin V(\bar{\boldsymbol{I}}, \bar{\boldsymbol{u}})$, то $v(\boldsymbol{z}_0) > 0$ и обратното.

  Въвеждаме $v$, защото така намирането на множеството $V(\bar{\boldsymbol{I}}, \bar{\boldsymbol{u}})$ може да се разгледа като задача за намиране на неположителните линии на ниво на функция, понеже:
  \begin{equation*}
    \boldsymbol{z}_0 \in V(\bar{\boldsymbol{I}}, \bar{\boldsymbol{u}}) \iff v(\boldsymbol{z}_0) \leq 0
  \end{equation*}

  С други думи:
  \begin{equation*}
    \begin{split}
      V(\bar{\boldsymbol{I}}, \bar{\boldsymbol{u}}) = \{\boldsymbol{z}_0 \in \Omega \vert v(\boldsymbol{z}_0) \leq 0\}\\ \quad \partial V(\bar{\boldsymbol{I}}, \bar{\boldsymbol{u}}) = \{\boldsymbol{z}_0 \in \Omega \vert v(\boldsymbol{z}_0) = 0\}
    \end{split}
  \end{equation*}
\end{frame}


\begin{frame}[c]{Уравнение на Хамилтон-Якоби-Белман}
  В сила е принцип за динамично програмиране:
  \begin{equation*}
    v(\boldsymbol{z}_0) = \inf_{\boldsymbol{u} \in \mathcal{U}} \max\{e^{-\lambda t} v(\boldsymbol{z}_0), \sup_{s \in (0, t]} e^{-\lambda t} \Gamma(\boldsymbol{z}(s; \boldsymbol{z}_0; \boldsymbol{u}))\}
  \end{equation*}
  $v$ е единственото непрекъснато вискозно решение на \textbf{уравнението от типа на Хамилтон-Якоби–Белман} \myfootcite{Zidani2013}:
  \begin{equation}
    \begin{split}
      \label{eq:HJB}
      &\min\{\lambda v(\boldsymbol{z}) + \mathcal{H}(\boldsymbol{z}, \grad{v}), v(\boldsymbol{z}) - \Gamma(\boldsymbol{z}) \} = 0, \quad \boldsymbol{z} \in \mathbb{R}^4 \\
      &\mathcal{H}(\boldsymbol{z}, \boldsymbol{w}) = \max_{\boldsymbol{u} \in U} \innerproduct{-\boldsymbol{f}(\boldsymbol{z}, \boldsymbol{u})}{\boldsymbol{w}}
    \end{split}
  \end{equation}
\end{frame}

\begin{frame}[c]{Числено решение на уравнението на Х-Я-Б}
  % \vspace{-1em}
  Решението на \eqref{eq:HJB} може да се разгледа като стационарно решение на ЧДУ, където $v$ зависи от времето:
  \begin{equation}
    \label{eq:HJBTime}
    \begin{split}
      &\min\left\{\pdv{v}{t}\left(\boldsymbol{z}, t\right) + \lambda v(\boldsymbol{z}, t) + \mathcal{H}(\boldsymbol{z}, \grad{v}), v(\boldsymbol{z}, t) - \Gamma(\boldsymbol{z})\right\} = 0, \Hquad \boldsymbol{z} \in \mathbb{R}^4,~ t>0 \\
      &v(\boldsymbol{z}, 0) = v_0(\boldsymbol{z}), \quad \boldsymbol{z} \in \mathbb{R}^4 \\
      % &\mathcal{H}(\boldsymbol{z}, \boldsymbol{w}) = \max_{\boldsymbol{u} \in \mathcal{U}} \innerproduct{-\boldsymbol{f}(\boldsymbol{z}, \boldsymbol{u})}{\boldsymbol{w}}
    \end{split}
  \end{equation}
  % Началното условие взимаме $v_0 = \Gamma$.
  % Поставяме ГУ на Нойман.
  \end{frame}

  \begin{frame}[t]{Числено решение на уравнението на Х-Я-Б}

Използваната дискретизация по пространството е равномерна със стъпки $h_{x_1}, h_{x_2}, h_{y_1}, h_{y_2}$.

Чрез метода WENO (Weighted Essentially Non-Oscillatory) се получават по-точни приближения за разлика напред и назад $v_{\eta}^{\pm}$ на производните $\pdv{v}{\eta},~ \eta = x_1, x_2, y_1, y_2$.

  Численият Хамилтониян от вида Lax-Friedrichs $\hat{\mathcal{H}}$ е \myfootcite{Osher2003}:
  \begin{small}
    \begin{equation*}
      \hat{\mathcal{H}} = \mathcal{H}\left(\boldsymbol{z}, \frac{v_{x_1}^++v_{x_1}^-}{2}, \frac{v_{x_2}^++v_{x_2}^-}{2}, \frac{v_{y_1}^++v_{y_1}^-}{2}, \frac{v_{y_2}^++v_{y_2}^-}{2}\right) - \sum_{\eta = x_1, x_2, y_1, y_2} \alpha^{\eta} \frac{v_{\eta}^+-v_{\eta}^-}{2}
    \end{equation*}
  \end{small}
  Множителите $\alpha^{\eta}$ са от вида:
  \begin{small}
    \begin{equation*}
      \begin{split}
        &\alpha^{x_1} = \max_{\boldsymbol{w}} \abs{\pdv{\mathcal{H}}{w_1}\left(\boldsymbol{z}, \boldsymbol{w}\right)},~
        \alpha^{x_2} = \max_{\boldsymbol{w}} \abs{\pdv{\mathcal{H}}{w_2}\left(\boldsymbol{z}, \boldsymbol{w}\right)}, \\
        &\alpha^{y_1} = \max_{\boldsymbol{w}} \abs{\pdv{\mathcal{H}}{w_3}\left(\boldsymbol{z}, \boldsymbol{w}\right)},~
        \alpha^{y_2} = \max_{\boldsymbol{w}} \abs{\pdv{\mathcal{H}}{w_4}\left(\boldsymbol{z}, \boldsymbol{w}\right)}
      \end{split}
    \end{equation*}
  \end{small}
\end{frame}

\begin{frame}[c]{Числено решение на уравнението на Х-Я-Б}
  Използваната дискретизация по времето е равномерна със стъпка $\tau$ и по него се апроксимира с подобрения метод на Ойлер.

  За да може методът да е TVD, трябва да е изпълнено условието на Courant-Friedrichs-Lewy:
  \begin{equation*}
    \tau \max_{\boldsymbol{z}, \boldsymbol{w}} \left(\frac{\abs{\pdv{\mathcal{H}}{w_1}}}{h_{x_1}} + \frac{\abs{\pdv{\mathcal{H}}{w_2}}}{h_{x_2}} + \frac{\abs{\pdv{\mathcal{H}}{w_3}}}{h_{y_1}} + \frac{\abs{\pdv{\mathcal{H}}{w_4}}}{h_{y_2}}\right) < 1
  \end{equation*}
\end{frame}
