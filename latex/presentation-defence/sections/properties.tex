\section{\hspace{1em}Свойства на модела}

% \begin{frame}[t]{Свойства на модела}
%   \begin{proposition}
%     За системата \eqref{eq:TheDimensionlessProblem} са в сила:
%     \begin{enumerate}
%       \item Съществува единствено решение за произволни управления.
%       \item Решение с начално условие в $\Omega$ е ограничено в $\Omega$.
%       \item Системата е кооперативна.
%       \item Системата е силно вдлъбната.
%       \item Системата е неразложима.
%     \end{enumerate}
%   \end{proposition}
% \end{frame}

\begin{frame}[c]{Свойства на модела}
  \begin{proposition}
    За всяко $\boldsymbol{u} \in \mathscr{U}$ задачата на Коши за \eqref{eq:TheDimensionlessProblem} има единствено решение.
  \end{proposition}
  \vspace{2em}

  % Доказва се с Липшицовост на $\boldsymbol{f}$ по $\boldsymbol{z}$.
  % Сметката се основава на свойства на нормата (хомогенност и неравенсто на триъгълника) и  е дълга, затова тук се пропуска.

  \begin{proposition}
    $\Omega$ е положително инвариантно за \eqref{eq:TheDimensionlessProblem}.
  \end{proposition}

  % Доказва се с теоремата на Nagumo, че $\boldsymbol{f} \vert_{\partial \Omega}$ е насочен към $\Omega$.
  % \begin{equation*}
  %   \begin{split}
  %     &f_{x_1}\vert_{\Omega \cap \{x_1=0\}} = (1-\kappa u_1)(b_{11} y_1 + b_{12} y_2) \geq 0, \Hquad f_{x_1}\vert_{\Omega \cap \{x_1=1\}} = - \gamma_1 < 0 \\
  %     &f_{x_2}\vert_{\Omega \cap \{x_2=0\}} = (1-\kappa u_2)(b_{21} y_1 + b_{22} y_2) \geq 0, \Hquad f_{x_2}\vert_{\Omega \cap \{x_2=1\}} = - \gamma_2 < 0 \\
  %     &f_{y_1}\vert_{\Omega \cap \{y_1=0\}} = c_{11}(1-\kappa u_1) x_1 + c_{12}(1-\kappa u_2) x_2 \geq 0 \Hquad f_{y_1}\vert_{\Omega \cap \{y_1=1\}} = - \mu_1 < 0 \\
  %     &f_{y_2}\vert_{\Omega \cap \{y_2=0\}} = c_{21}(1-\kappa u_1) x_1 + c_{22}(1-\kappa u_2) x_2 \geq 0 \Hquad f_{y_2}\vert_{\Omega \cap \{y_2=1\}} = - \mu_2 < 0
  %   \end{split}
  % \end{equation*}
\end{frame}

\begin{frame}[t]{Свойства на модела}
  \begin{proposition}
    Системата \eqref{eq:TheDimensionlessProblem} е \underline{кооперативна}, т.е. Якобианът ѝ има неотрицателни компоненти извън главния диагонал.
  \end{proposition}

  \begin{proposition}
    Системата \eqref{eq:TheDimensionlessProblem} е \underline{силно вдлъбната}, т.е. за Якобиана ѝ $\mathrm{D}\boldsymbol{f}$ е в сила $\pmb{0} < \boldsymbol{z}_1 < \boldsymbol{z}_2 \implies \mathrm{D}\boldsymbol{f}(\boldsymbol{z}_2) < \mathrm{D}\boldsymbol{f}(\boldsymbol{z}_1)$.
    % т.е. Якобианът ѝ е с компоненти, които са намаляващи функции на координатите на точката, в която е взет.
    % т.е. Якобианът ѝ действа като покомпонентно вдлъбнато изображение.
  \end{proposition}

  \begin{proposition}
    Системата \eqref{eq:TheDimensionlessProblem} е \underline{неразложима} при $p_{ij} \notin \{0, 1\}$, т.е. ненулевите компоненти на Якобиана ѝ образуват матрица на съседство на силно свързан ориентиран граф.
  \end{proposition}
\end{frame}

% \begin{frame}[t]{Якобиан на \eqref{eq:TheDimensionlessProblem}}
%   Якобианът за системата \eqref{eq:TheDimensionlessProblem} може да се представи във вида:
%   \begin{equation*}
%     \mathrm{D} \boldsymbol{f}(x_1, x_2, y_1, y_2) =
%     \begin{pmatrix}
%       \pdv{f_{x_1}}{x_1} && \pdv{f_{x_1}}{x_2} && \pdv{f_{x_1}}{y_1} && \pdv{f_{x_1}}{y_2} \\
%       \pdv{f_{x_2}}{x_1} && \pdv{f_{x_2}}{x_2} && \pdv{f_{x_2}}{y_1} && \pdv{f_{x_2}}{y_2} \\
%       \pdv{f_{y_1}}{x_1} && \pdv{f_{y_1}}{x_2} && \pdv{f_{y_1}}{y_1} && \pdv{f_{y_1}}{y_2} \\
%       \pdv{f_{y_2}}{x_1} && \pdv{f_{y_2}}{x_2} && \pdv{f_{y_2}}{y_1} && \pdv{f_{y_2}}{y_2}
%     \end{pmatrix}
%   \end{equation*}
%   \label{eq:JacobianElements}
%   \begin{equation*}
%     \begin{split}
%       &\pdv{f_{x_1}}{x_1} = -(1-\kappa u_1) \left(b_{11} y_1 + b_{12} y_2\right) - \gamma_1 < 0, \Hquad
%       \pdv{f_{x_1}}{x_2} = 0 \\
%       &\pdv{f_{x_1}}{y_1} = (1-x_1) (1-\kappa u_1) b_{11} \geq 0, \Hquad
%       \pdv{f_{x_1}}{y_2} = (1-x_1) (1-\kappa u_1) b_{12} \geq 0\\
%       &\vdots
%     \end{split}
%   \end{equation*}
% \end{frame}

\begin{frame}[t]{Равновесни точки}
  \begin{proposition}
    За система \eqref{eq:TheDimensionlessProblem} при фиксирано $\boldsymbol{u}(t) \equiv \boldsymbol{u} = const$ е в сила точно едно от:
    \begin{enumerate}
      \item $\pmb{0}$ е единствена равновесна точка (глобално асимптотично устойчива).
      \item $\pmb{0}$ е неустойчива равновесна точка и съществува точно една друга ендемична равновесна точка $\boldsymbol{E}^* = (x_1^*, x_2^*, y_1^*, y_2^*)$ (глобално асимптотично устойчива).
    \end{enumerate}
  \end{proposition}
  Твърдението се доказва с помощта на изведените свойства на \eqref{eq:TheDimensionlessProblem} и теорема на Smith \cite{Smith1986}.
\end{frame}
