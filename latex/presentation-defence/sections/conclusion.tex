\section{\hspace{1em}Заключение}
\begin{frame}[c]{Заключения}
  \begin{itemize}
    \item Възможно е размера на $V(\bar{\boldsymbol{I}}, \bar{\boldsymbol{u}})$ да не варира много спрямо мобилността.
    \item При мобилност може да се намали пикът на заразени в едното местообитание, за сметка на това в другото.
    \item Здравна политика на две местообитания може да не е изпълнена при изолираност, но изпълнена при мобилност.
    \item Възможно е мобилността и употребата на репелент да нямат голяма роля.
  \end{itemize}
\end{frame}

\begin{frame}[c]{Основни резултати и приноси}
  \begin{itemize}
    \item Доказателство на свойствата на модела.
    \item Имплементация на численото решаване на уравнението на Хамилтон-Якоби–Белман на C++.
    \item Анализ на модела за 3 набора параметри.
  \end{itemize}
\end{frame}
