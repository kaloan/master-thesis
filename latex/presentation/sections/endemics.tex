\begin{frame}[t]{Ендемично състояние}
  Зараза има ендемичен характер, когато за дълъг период от време, заразените с нея са положително число.

  Възможно е този брой да е приблизително равен във времето, или да се изменя периодично.

  В моделите, които ще изследваме, ендемията съответства на равновесна точка, която е асимптотично устойчива. Това ще рече, че към нея се приближава решението на системата с времето, освен ако не сме започнали в състоянието на липса на зараза.
\end{frame}

\begin{frame}[t]{Репродукционно число $\mathscr{R}_0$}
  $\mathscr{R}_0$ носи смисъла на брой вторични случаи на заразата, причинени от един първичен. За да може болестта да има ендемично състояние, то е необходимо $\mathscr{R}_0 > 1$.
  Наистина, иначе броят заразени веднага щеше да намалее и съответно нямаше да има равновесна точка, различна от $\mathbf{0}$. За модела на Ross е:
  \begin{equation}
    \mathscr{R}_0 = \frac{1}{\gamma} \times \beta_{hv} b \frac{M}{N} \times \frac{1}{\mu} \times \beta_{vh} b = \frac{b^2 \beta_{vh} \beta_{hv} M}{\gamma \mu N}
  \end{equation}
  С други думи Ross е открил сходна по същност до него оценка:
  \begin{equation}
    \mathscr{R}_0 > 1 \iff M > M^* = \frac{\gamma \mu N}{b^2 \beta_{vh} \beta_{hv}}
  \end{equation}
\end{frame}
