\begin{frame}[t]{Ендемично състояние}
  Зараза има ендемичен характер, когато за дълъг период от време, заразените с нея са положително число.

  Възможно е този брой да е приблизително равен във времето, или да се изменя периодично.

  В моделите, които ще изследваме, ендемията съответства на равновесна точка, която е асимптотично устойчива. Това ще рече, че към нея се приближава решението на системата с времето, освен ако не сме започнали в състоянието на липса на зараза.
\end{frame}

\begin{frame}[t]{Репродукционно число $\mathscr{R}_0$}
  $\mathscr{R}_0$ носи смисъла на брой вторични случаи на заразата, причинени от един първичен. За да може болестта да има ендемично състояние, то е необходимо $\mathscr{R}_0 > 1$.
  Наистина, иначе броят заразени веднага щеше да намалее и съответно нямаше да има равновесна точка, различна от $\mathbf{0}$. За модела на Ross е:
  \begin{equation}
    \mathscr{R}_0 = \frac{1}{\gamma} \times \beta_{hv} b \frac{M}{N} \times \frac{1}{\mu} \times \beta_{vh} b = \frac{b^2 \beta_{vh} \beta_{hv} M}{\gamma \mu N}
  \end{equation}
  С други думи Ross е открил сходна по същност до него оценка:
  \begin{equation}
    \mathscr{R}_0 > 1 \iff M > M^* = \frac{\gamma \mu N}{b^2 \beta_{vh} \beta_{hv}}
  \end{equation}
\end{frame}

\begin{frame}[t]{$\mathscr{R}_0$ в многомерни модели}
  Нека имаме няколко категории хора, податливи на заразата, които сме разграничили и това са $\mathbf{z} = (z_1, \cdots, z_n)^T$.

  Нека системата се представя във вида $\dot{\mathbf{z}} = \mathbf{G}{\mathbf{z}} = \mathscr{F}(\mathbf{z}) - \mathscr{V}(\mathbf{z})$.
  $\mathscr{F}$ определя новите заразени, а $\mathscr{V}(\mathbf{z}) = \mathscr{V}^-(\mathbf{z}) - \mathscr{V}^+(\mathbf{z})$ е мобилността, която сме разделили на прииждащи и заминащи за съответните групи.

  Може да се покаже, че е в сила следната теорема
\end{frame}

\begin{frame}[t]{$\mathscr{R}_0$ в многомерни модели}
  \begin{theorem}
    При изпълнени следните условия:
    \begin{enumerate}
      \item $\mathbf{z} \geq \mathbf{0} \implies \mathscr{V}(\mathbf{z}) \geq 0, \mathscr{V}^+(\mathbf{z}) \geq 0, \mathscr{V}^-(\mathbf{z}) \geq 0$
      \item $z_i = 0 \implies \mathscr{V}_{i}^- = 0$
      \item $\mathscr{F}(\mathbf{0}) = \mathbf{0}, \mathscr{V}(\mathbf{0}) = \mathbf{0}$
      \item $\mathscr{F}(\mathbf{z}) = \mathbf{0} \implies$ всички собствени стойности на $\mathrm{D}\mathbf{G}{\mathbf{0}}$ са с отрицателна реална част
    \end{enumerate}
    в сила за репродукционното число е $\mathscr{R}_0 = \rho(F V^{-1})$, където $\rho$ е спектралния радиус, а $F = D\mathscr{F}(\mathbf{0}), V = D\mathscr{V}(\mathbf{0})$, където $F \geq \mathscr{O}$, а $V$ е несингулярна M-матрица. \\
    Допълнително, $\mathbf{0}$ е локално асимптотично устойчива, ако $\mathscr{R}_0 < 1$ и неустойчива, ако $\mathscr{R}_0 > 1$.
  \end{theorem}
\end{frame}

\begin{frame}[t]{$\mathscr{R}_0$ в многомерни модели}
  $F_{ij}$ е скоростта, с която индивид от група $j$ заразява индивиди от група $i$, а $V^{-1}_{jk}$ е средната продължителност на пребиваване на индивид от група $k$ сред индивидите от група $j$, съответно $(F V^-1)_{ik}$ са средния брой новозаразени от $i$ заради индивид от $k$.
\end{frame}
