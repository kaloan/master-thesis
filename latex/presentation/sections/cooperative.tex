\section{Кооперативни системи}
\begin{frame}[t]{Силновдлъбнати системи}
  \begin{equation}
    \label{eq:GenericSystem}
    \dot{\boldsymbol{x}} = \boldsymbol{f}(t, \boldsymbol{x}),  \quad \boldsymbol{x} \in \mathbb{R}^n, \boldsymbol{f} \in C^1(J \times \mathbb{R}^n, \mathbb{R}^n), J \subset \mathbb{R} \text{ е интервал}
    \end{equation}

\begin{definition}[Силна вдлъбнатост]

  \begin{equation}
     \forall{t \in J} \left(\mathrm{D}_x\boldsymbol{F}(t, \boldsymbol{x}_2) \leq \mathrm{D}_x\boldsymbol{F}(t, \boldsymbol{x}_1) \wedge \mathrm{D}_x\boldsymbol{F}(t, \boldsymbol{x}_2) \neq \mathrm{D}_x\boldsymbol{F}(t, \boldsymbol{x}_1)\right)
    \end{equation}
  \end{definition}

  \begin{nota-bene}
    В презентацията всичките векторни/матрични (не-)равенства се разбират покомпоненто.
  \end{nota-bene}
\end{frame}

\begin{frame}[t]{Неразложими системи}
  \begin{definition}[(Не-)разложима матрица]
    Матрицата $A=(a_{ij})_{n \times n}$ e разложима, ако съществува пермутационна матрица $P$, с която:
    \begin{equation*}
      PAP^T =
      \begin{pmatrix}
        B & C \\
        \mathscr{O} & D
      \end{pmatrix}, \quad B, D \text{ - квадратни}
    \end{equation*}
    Матрици, които не са разложими се наричат неразложими.
  \end{definition}
  \begin{definition}[(Не-)разложима система]
    Система \eqref{eq:GenericSystem} се нарича (не-)разложима, ако Якобианът на дясната страна $\mathrm{D}_x\boldsymbol{F}(t, \boldsymbol{x})$ във всяка точка е (не-)разложим.
  \end{definition}

\end{frame}

\begin{frame}[t]{Кооперативни системи}
  \begin{definition}[Квазимонотонна матрица]
    Матрица $A=(a_{ij})_{n \times n}$ е квазимонотонна, ако
    \begin{equation*}
      \forall{i,j \in \{\overline{1,n}\}} \left(i \neq j \implies a_{ij} \geq 0\right)
    \end{equation*}
  \end{definition}

  \begin{theorem}[Perron-Frobenius]
    Ако $A$ е неразложима и квазимонотонна, то доминантната ѝ собствена стойност $\mu$ е проста и на нея отговаря положителен собствен вектор $\boldsymbol{v} \in \mathbb{R}_{+}^n$.
  \end{theorem}

  \begin{definition}[Кооперативна система]
    \label{def:Cooperative}
    Системата \eqref{eq:GenericSystem} е кооперативна (или още квазимонотонна), ако
    \begin{equation}
      \forall{t \in J} \Hquad \forall{\boldsymbol{x} \in \mathbb{R}_{+}^n} \Hquad \forall{i,j \in \{\overline{1,n}\}} \Hquad \left(i \neq j \implies \pdv{f_i}{x_j}\left(t, \boldsymbol{x}\right) \geq 0\right)
    \end{equation}
  \end{definition}

% \begin{theorem}
%   Ако \eqref{eq:GenericSystem} е линейна (т.е. $\dot{\boldsymbol{x}} = A \boldsymbol{x}, A = (a_{ij})_{n \times n}$) система, то $\mathbb{R}_{+}^n$ е инвариантно. Допълнително, ако A е неразложима, то за кое да е $t > 0$ решението е във $\mathrm{int} \mathbb{R}_{+}^n$, стига началното решение да е ненулево, т.е. $\boldsymbol{x}_0 \neq \boldsymbol{0}$.
% \end{theorem}
\end{frame}

\begin{frame}[t]{Кооперативни системи}
  \begin{theorem}[Сравнение на решения]
    \label{thm:Comparison}
    Нека $\boldsymbol{f}, \boldsymbol{g} \in C^1(\mathrm{int} \mathbb{R}_{+}^n, \mathbb{R}^n)$ са такива, че системите $\dot{\boldsymbol{x}}=\boldsymbol{f}(\boldsymbol{x})$, $\dot{\boldsymbol{y}}=\boldsymbol{g}(\boldsymbol{y})$ са кооперативни, $\boldsymbol{f} \leq \boldsymbol{g}$ и $\boldsymbol{x}_0 \leq \boldsymbol{y}_0$. Тогава $\forall{t>0}(\boldsymbol{x}(t) \leq \boldsymbol{y}(t))$.
  \end{theorem}

  \begin{theorem}
    Система, която е кооперативна, неразложима и силно вдлъбната не може да има повече от една ненулева равновесна точка.
  \end{theorem}
\end{frame}
