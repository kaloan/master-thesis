\section{Кооперативни системи}
\begin{frame}[t]{Кооперативни системи}
  \begin{definition}[Съкратено означение]
    $\mathbb{H} = \mathbb{R}_{+}^n$
  \end{definition}
  \begin{definition}[Кооперативна система]
    \label{def:Cooperative}
    Системата:
    \begin{equation}
      \label{eq:Cooperative}
      \dot{\mathbf{x}} = \mathbf{f}(t, \mathbf{x}),  \quad \mathbf{x} \in \mathbb{R}^n, \mathbf{f} \in C^1(J \times \mathbb{R}^n, \mathbb{R}^n), J \subset \mathbb{R} \text{ е интервал}
      % \mathbb{R}^n \rightarrow \mathbb{R}^n, \quad f \in
    \end{equation}
    е кооперативна (или още квазимонотонна), ако
    \begin{equation}
      \forall{t \in J}\forall{\mathbf{x} \in \mathbb{H}}\forall{i,j \in \{\overline{1,n}\}}\left(i \neq j \implies \pdv{f_i}{x_j}(t, \mathbf{x}) \geq 0\right)
    \end{equation}
  \end{definition}

\end{frame}

\begin{frame}[t]{Кооперативни системи}
  \begin{definition}[Квазимонотонна матрица]
    Матрица $A=(a_{ij})_{n \times n}$ е квазимонотонна, ако
    \begin{equation*}
      \forall{i,j \in \{\overline{1,n}\}} \left(i \neq j \implies a_{ij} \geq 0\right)
    \end{equation*}
  \end{definition}
  \begin{definition}[(Не-)разложима матрица]
    Матрицата $A=(a_{ij})_{n \times n}$ e разложима, ако съществува пермутационна матрица $P$, с която:
    \begin{equation*}
      PAP^T =
      \begin{pmatrix}
        B & C \\
        \mathscr{O} & D
      \end{pmatrix}, \quad B, D \text{ - квадратни}
    \end{equation*}
    Матрици, които не са разложими се наричат неразложими.
  \end{definition}
\end{frame}

\begin{frame}[t]{Кооперативни системи}
  \begin{theorem}[Perron-Frobenius]
    Ако $A$ е неразложима, то доминантната ѝ собствена стойност $\mu$ е проста и на нея отговаря положителен собствен вектор $\mathbf{v} \in \mathbb{H}$.
  \end{theorem}

  \begin{theorem}
    Ако \ref{eq:Cooperative} е линейна (т.е. $\dot{\mathbf{x}} = A \mathbf{x}, A = (a_{ij})_{n \times n}$) система, то $\mathbb{H}$ е инвариантно. Допълнително, ако A е неразложима, то за кое да е $t > 0$ решението е във $\mathrm{int} \mathbb{H}$, стига началното решение да е ненулево, т.е. $\mathbf{x}_0 \neq \mathbf{0}$.
  \end{theorem}

  \begin{definition}[(Не-)разложима система]
    Система \ref{eq:Cooperative} се нарича (не-)разложима, ако Якобианът на дясната страна $\mathrm{D}\mathbf{F}$ във всяка точка е (не-)разложим.
  \end{definition}
\end{frame}

\begin{frame}[t]{Кооперативни системи}
  \begin{theorem}[Сравнение на решения]
    \label{thm:Comparison}
    Нека $\mathbf{f}, \mathbf{g} \in C^1(\mathrm{int} \mathbb{H}, \mathbb{R}^n)$ са такива, че системите $\dot{\mathbf{x}}=\mathbf{f}(\mathbf{x})$, $\dot{\mathbf{y}}=\mathbf{g}(\mathbf{y})$ са кооперативни, $\mathbf{f} \leq \mathbf{g}$ и $\mathbf{x}_0 \leq \mathbf{y}_0$. Тогава $\forall{t>0}(\mathbf{x}(t) \leq \mathbf{y}(t))$.
  \end{theorem}

  \begin{definition}[Силна вдлъбнатост]
    $\dot{\mathbf{x}} = \mathbf{F}(\mathbf{x})$ се нарича силно вдлъбната, ако $\mathbf{0} < \mathbf{x}_1 < \mathbf{x}_2 \implies \mathrm{D}\mathbf{F}(\mathbf{x}_2) < \mathrm{D}\mathbf{F}(\mathbf{x}_1)$
  \end{definition}

  \begin{theorem}
    Система, която е кооперативна, неразложима и силно вдлъбната не може да има две различни неподвижни точки, които да не са тривиалната.
  \end{theorem}
\end{frame}

\begin{frame}[t]{Кооперативни системи}
  С помощта на горните твърдения и техни следствия може да се покаже за модела на Bichara, че е изпълнено точно едно от:
  \begin{itemize}
    \item $\mathscr{R}_0 \leq 1$ и $\mathbf{0}$ е единствената равновесна точка и е глобално асимптотично устойчива.
    \item $\mathscr{R}_0 > 1$ и $\mathbf{0}$ е неустойчива равновесна точка, като ако системата е неразложима, има единствена глобално асимптотично устойчива точка вътрешна за $\bigtimes_{i=1}^{n} [0, N_i] \times \bigtimes_{j=1}^{m} [0, M_j]$ (тоест маларията има ендемичен характер).
  \end{itemize}
\end{frame}
