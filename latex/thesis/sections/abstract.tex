% \begin{abstract}
%   abstract-text
% \end{abstract}

\begin{center}
  \vspace*{1cm}

  \Huge
  \textbf{Резюме}
  \vspace*{0.5cm}

  \large
  В дипломната работа се изследва ефекта на човешката мобилност в модел на малария от тип Ross-Macdonald с две местообитания и употреба на репелент.
  Всяко местообитание може ограничено да управлява пропорцията от предпазени жители с репелента във времето.
  Допуска се, че двете местообитания имат независима здравна политика, чийто стремеж е да задържи пропорцията от заразени жители под зададен таван.
  Поставя се задача да се намерят началните условия, които позволяват управления на епидемията чрез употребата на репелент, така че да е изпълнена здравната политика и в двете местообитания.
  Задачата е решена числено с помощта на метода на криви на ниво на подходяща функция на Белман.
\end{center}

\thispagestyle{empty}
