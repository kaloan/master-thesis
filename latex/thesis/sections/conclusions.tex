\section{\hspace{1em} Заключение}
Поради широкото си разпространение, маларията се нуждае от математически модели за изследване на динамичното \'{и} развитие.
По същество моделите стават се разширяват с времето и включват повече подробности.
Разгледаният модел \eqref{eq:TheProblem} в дипломната работа е развитие на модел с движение на хора между две местообитания \cite{Bichara2016} и модел с употреба на репелент \cite{Rashkov2022}, като и двата са основани на първият диференциален модел на маларията \cite{Smith2012}.
Изведени са основни твърдения за модела, а също така изследвани динамиката на системата за четирите възможни случая на присъствие/отсъствие на мобилност и употреба на репелент.
Така се показва влиянието на тези разширения поотделно и в съвкупност.
Изследвана е задача \eqref{eq:ViabilityKernel} за ограничаване на заразените хора под първоначално решен таван.
За нея са приложени методи \cite{Zidani2013} за характеризиране на множеството начални състояния, позволяващи такива управления, като решение на частно диференциално уравнение на Хамилтон-Якоби-Белман \eqref{eq:HJB}.

Представеният модел не включва раждаемостта и смъртността на хората и смята популационната динамика на комарите за фиксирана.
Така са изпълнени типични допускания за константни по време население на местообитанията и популации от комари в тях.
Възможно е те да се включат в модела, което ще доведе до нови 4 диференциални уравнения.
За съжаление, тогава задачата \eqref{eq:ViabilityKernel} ще се изследва още по-трудно, защото произлизащото от нея уравнение \eqref{eq:HJB} ще трябва да се решава в пространство с размерност равна на броя класове в модела.
Тъй като при добавена популационна динамика, ще има 8 класа, то ще се решава в $\mathbb{R}^8$.

Възможно бъдещо изследване е как се държи Ойлеров модел с две местообитания при миграция.
За съжаление не е ясно дали този модел би описвал добре разпространението.
Тъй като движението представлява дълготрайна миграция на населението по своята същност би бил по-уместен за изследване за дълги периоди от време.
В този случай би било удачно да се прибави популационна динамика.
В \cite{Prosper2012} е изследван малариен Ойлеров модел, като комарите не са пряко моделирани.
Така ако се разглежда само човешката популационна динамика, броят класове ще се запази и съответно \eqref{eq:HJB} ще се решава в $\mathbb{R}^4$, както в настоящата дипломна работа.

В \cite{DeLara2016} също се изследва ядрото на слаба инвариантност на Белман, но там управлението е инсектицид, влияещ на смъртността на комарите, но без методологията \cite{Zidani2013} с уравнение на Хамилтон-Якоби-Белман.
За съжаление авторите не взимат предвид, че това би повлияло на популационната динамика на комарите и вече няма да е изпълнено допускането за константна популация на комарите.
Това би довело до включване на още една категория - за податливите комари.
За така променената задача може да се реши еквивалентното \eqref{eq:HJB} в $\mathbb{R}^3$, което е изчислимо постижимо (освен дипломната работа, доказателство за това е и \cite{Rashkov2021}).

В дипломната работа също така са поместени и някои основни термини и твърдения от теориите за изследване на епидемични модели, кооперативни системи, управляеми системи и многозначни функции.
Препратените статии могат да са добро въведение в маларийните (и по-общо епидемичните) модели, като най-добро начало би биха били \cite{Capasso2008} и \cite{Smith2012}.
% Тогава употребата на репелент би било по-разумно да се моделира спрямо крайното местообитание, а не родното.
