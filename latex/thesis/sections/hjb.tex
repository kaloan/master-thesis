\section{Вариационна задача на Хамилтон-Якоби-Белман}
За намиране на ядрото на допустимост $\eqref{eq:ViabilityKernel}$ се подхожда със задачата на Хамилтон-Якоби-Белман за минимизация на функционал, като $v$ е фунцкията на стойността. Опитваме се да решим:
\begin{equation}
  \label{eq:HJB}
  \min\{\lambda v(\mathbf{z}) + \mathcal{H}(\mathbf{z}, \grad{v}), v(\mathbf{z}) - \Gamma(\mathbf{z})\} = 0
\end{equation}
Хамилтонянът е дефиниран чрез производна по направлението $\grad{v}$, като:
\begin{equation}
  \label{eq:Hamiltonian}
  \mathcal{H}(\mathbf{z}, \grad{v}) = - \max_{\mathbf{u} \in U} \innerproduct{\mathbf{f}(\mathbf{z}, \mathbf{u})}{\grad{v}}
\end{equation}

Използвайки вида на $\mathbf{F}$, след групиране по части зависещи/независещи от управлението, получаваме:
\begin{multline}
  \label{eq:HamiltonianFull}
  \mathcal{H}(\mathbf{z}, \grad{v}) = \\
  \left[\gamma_1 x_1(t) - (1-x_1(t)) \left(b_{11} y_1(t) + b_{12} y_2(t)\right) \right] \pdv{v}{x_1} +
  \left[\gamma_2 x_2(t) - (1-x_2(t)) \left(b_{21} y_1(t) + b_{22} y_2(t)\right) \right] \pdv{v}{x_2} + \\
  \left[\mu_1 y_1(t) - (1-y_1(t)) \left(c_{11} x_1(t) + c_{12} x_2(t)\right) x\right] \pdv{v}{y_1} +
  \left[\mu_2 y_2(t) - (1-y_2(t)) \left(c_{21} x_1(t) + c_{22} x_2(t)\right)\right] \pdv{v}{y_2} + \\
  \max_{\mathbf{u} \in U}\left\{0, (1-x_1(t)) \kappa u_1 \left(b_{11} y_1(t) + b_{12} y_2(t)\right) \pdv{v}{x_1}\right\} +
  \max_{\mathbf{u} \in U}\left\{0, (1-x_2(t)) \kappa u_2 \left(b_{21} y_1(t) + b_{22} y_2(t)\right) \pdv{v}{x_2}\right\} + \\
  \max_{\mathbf{u} \in U}\left\{0, (1-y_1(t)) \kappa \left(c_{11} x_1(t) u_1 + c_{12} x_2(t) u_2 \right) \pdv{v}{y_1}\right\} +
  \max_{\mathbf{u} \in U}\left\{0, (1-y_2(t)) \kappa \left(c_{21} x_1(t) u_1 + c_{22} x_2(t) u_2 \right) \pdv{v}{y_2}\right\}
  \end{multline}
