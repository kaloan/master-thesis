\section{Въведение}
Маларията е заболяване, причинено от едноклетъчни организми, плазмодии.
Предаването на заразяването се основава на ухапване на заразен с плазмодий човек от незаразен комар или ухапване на незаразен човек от заразен комар. \\
В литературата моделирането на малария се основава на модели от типа на Ross-Macdonald \cite{Smith2012}, които имат сходни допускания за разпространението ѝ, като предаването от човек на комар и обратното става по закона за действие на масите.
Налични са модели \cite{Cosner2009, Ruktanonchai2016, Bichara2016} с няколко местообитания с различни параметри за заболяването, които включват и мобилност.
Изследвани са модели с управление за изтребване на комари \cite{DeLara2016}, както и такива за управление на пропорцията хора защитени с репелент против комари \cite{Rashkov2022, Rashkov2021}. \\
В дипломната работа се разглежда модел на малария от типа на Ross-Macdonald с две местообитания, мобилност между тях и употребата на репелент против комари.
Моделът е основан на 4 обикновени диференциални уравнения, за всяко от местообитанията има по едно уравнение за заразените комари и едно за заразените жители.
% хора, които пребивават основно в това местообитание.
Понеже по природа комарите не прелитат големи разстояния, се разглежда само мобилност на хората между двете местообитания, която се приема за константна.
Действието на репелента се моделира като линейно намаляване на честотата на ухапвания.
Всяко местообитание може да управлява пропорцията от предпазени жители с репелента във времето, но се приема, че винаги е под някаква фиксирана стойност.
Допуска се, че двете местообитания имат независима здравна политика, която се изразява в желание заразените жители да са под някакъв константен брой във всеки момент от време.
Поставя се задача да се намерят началните условия, които позволяват управления на употребата на репелент, водещи до решения, които изпълняват здравната политика и на двете местообитания.
