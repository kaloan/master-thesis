\section{\hspace{1em} Въведение}
Маларията е заболяване, причинено от патогенни едноклетъчни същинскоядрени организми, маларийни плазмодии от рода \textit{Plasmodium}.
Заболяването при хората се случва с предаване на патогена чрез ухапване на незаразен човек от заразен комар, а патогенът може чрез ухапване на заразен човек да се предаде и на незаразен комар.
Симптомите се забелязват няколко седмици след заразяване и са с различна интензивност спрямо вида плазмодий (само 4 вида са заразни за хората).
Най-характерният симптом е периодична треска. Често се забелязва патологично разрастване на черния дроб и/или далака.
Маларията може да причини вреди и на бъбреците, а също да доведе до анемия, отоци в белите дробове, мозъка, като в някои случаи тези усложения са смъртоносни \cite[глава~83]{Baron1996}.

В литературата моделирането на малария се основава на модели от типа на Ross-Macdonald \cite{Smith2012}.
Тези модели разделят разглежданите популации от хора и комари на различни класове според емидемичните им статуси и имат сходни допускания за разпространението ѝ, като епидемичната динамика се основава на допускането, че предаването на патогена от човек към комар и обратното става по закона за действие на масите.
Имунитетът е нетраен, затова в най-простия случай се разглеждат 2 класа хора - податливи и заразени.
Тъй като комарите нямат имунна система, веднъж заразени не могат да се защитят от патогена и затова също се разделят на податливи и заразени.
Налични са модели \cite{Cosner2009, Ruktanonchai2016, Bichara2016, Agusto2021, Prosper2012} с няколко местообитания с различни параметри за заболяването, които включват и мобилност на популациите.
Изследвани са модели с управление за борба с комарите \cite{DeLara2016}, както и такива с управление, основано на средства за предпазване от ухапване, например репелент \cite{Rashkov2022, Rashkov2021}.

В дипломната работа се разглежда модел на малария от типа на Ross-Macdonald с две местообитания, мобилност на човешката популация между тях и употребата на репелент против комари.
Моделът е основан на система от 4 обикновени диференциални уравнения, за всяко от местообитанията има по едно уравнение за заразените комари и едно за заразените жители.
Мобилността се моделира чрез постановка на Лагранж \cite{Cosner2009}, която моделира относителното пребиваване на съответната популация в кое да е от двете местообитания.
Понеже по природа комарите не прелитат големи разстояния, се разглежда само мобилност на хората между двете местообитания, която се приема за константна.
Действието на репелента се моделира като линейно намаляване на честотата на ухапвания.
Всяко местообитание може да управлява пропорцията от предпазени жители с репелента във времето до някаква фиксирана стойност.
Допуска се, че двете местообитания имат независима здравна политика, чийто стремеж е да задържи пропорцията от заразени жители под някаква максимална стойност.
Поставя се задача да се намерят началните условия, които позволяват управления на епидемията чрез употребата на репелент, така че да е изпълнена здравната политика и в двете местообитания.
