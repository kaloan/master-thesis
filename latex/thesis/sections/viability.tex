\section{\hspace{1em} Ядро на слаба инвариантност}

\subsection{Случаи без употреба и с максимална употреба на репелент}

\begin{proposition}
  Ако $\mathscr{R}_0(\bar{\mathbf{u}}) > 1$ и $\mathbf{E}^* = (E_1^*, E_2^*)^T$, като $E_1^* > \bar{I}_1$ или $E_2^* > \bar{I}_2$, то ядрото на слаба инвариантност е тривиалното, т.е. $V(\bar{\mathbf{I}}, \bar{\mathbf{u}}) = {\mathbf{0}}$.
  % Във случая $\mathbf{u}(t) \equiv \bar{\mathbf{u}}$, то ако имаме ендемична точка $\mathbf{E}^*$ и $E_1^* > \bar{I}_1 \lor E_2^* > \bar{I}_2$, то търсената от нас задача няма решение.
\end{proposition}

\begin{proof}
  Наистина, по определение $I_1, I_2 \geq 0$, а пък $\mathbf{0}$ е равновесна за системата \eqref{eq:TheDimensionlessProblem}. Така $\mathbf{0} \in V(\bar{\mathbf{I}}, \bar{\mathbf{u}})$.

  Остава да забележим, че $\mathbf{f}(\mathbf{z}, \bar{\mathbf{u}}) \leq \mathbf{f}(\mathbf{z}, \mathbf{u})$ за всички $\mathbf{u} \in \mathscr{U}$.
  Така може да ползваме теоремата \eqref{thm:Comparison} за сравнение на решения на кооперативни системи и така всяка друга система ще е с мажориращо решение за еднакви начални $\mathbf{z}_0$.
  Понеже $\mathbf{E}^*$ е асимптотично устойчива и е изпълнено поне едно от неравенствата в това твърдение, то ще съществува $t > 0$, за което търсеното условие \eqref{eq:AllHospitalised} не е изпълнено.
  Но тогава за $t$ от мажорирането ще следва, че не е изпълнено и за кое да е друго решение.
\end{proof}

Обратната посока не е ясна. В случая $\mathbf{u}(t)=\mathbf{0}$, то ако липсва ендемична точка, решението ще клони към $\mathbf{0}$, но не е ясно дали винаги се намира в желаното множество, или излиза от него. Аналогично ако имаме ендемична точка.

Все пак може да се получи някакво слабо твърдение за задачата.

\begin{proposition}
  Ако \eqref{eq:AllHospitalised} е изпълнено за решението на система \eqref{eq:TheDimensionlessProblem} с $\mathbf{u} \equiv \mathbf{0}$ и начално условие $\mathbf{z}_0 = (\xi_1, \xi_2, 1, 1)^T$, то $\Xi = [0, \xi_1] \times [0, \xi_2] \times [0, 1] \times [0, 1] \subseteq V(\bar{\mathbf{I}}, \bar{\mathbf{u}})$.
\end{proposition}

\begin{proof}
  Веднага се вижда, че $\mathbf{f}(\mathbf{z}, \mathbf{u}) \leq \mathbf{f}(\mathbf{z}, \mathbf{0})$ за всички $\mathbf{u} \in \mathscr{U}$.
  Ползваме теоремата \eqref{thm:Comparison} за сравнение на решения и така за всяко друго решение $\tilde{\mathbf{z}}$ с начално условие $\tilde{\mathbf{z}}_0 \in \Xi$, понеже $\tilde{\mathbf{z}}_0 \leq \mathbf{z}_0$, то е изпълнено $\tilde{z}_1(t) \leq z_1(t) \leq \bar{I}_1, \tilde{z}_2(t) \leq z_2(t) \leq \bar{I}_2$, тоест $\tilde{\mathbf{z}}_0 \in V(\bar{\mathbf{I}}, \bar{\mathbf{u}})$.
\end{proof}

\begin{corollary}
  Ако \eqref{eq:AllHospitalised} е изпълнено за решението на система \eqref{eq:TheDimensionlessProblem} с $\mathbf{u} \equiv \mathbf{0}$ и начално условие $\mathbf{z}_0 = (\bar{I}_1, \bar{I}_2, 1, 1)^T$, то ядрото на слаба инвариатност е възможно най-голямо, т.е. $V(\bar{\mathbf{I}}, \bar{\mathbf{u}}) = \mathscr{I}$.
\end{corollary}

\subsection{Система на Marchaud/Peano}
Питаме се, кога ядрото на слаба инвариантност на Белман $\eqref{eq:ViabilityKernel}$ съществува и не е празно. От \cite{Aubin2011} за съществуването е необходимо да покажем, че $\mathscr{f}(\mathbf{z}(t))=\mathbf{f}(\mathbf{z}(t), \mathscr{U})$ е изображение на Marchaud/Peano.

% \begin{definition}[изображение на Marchaud/Peano]
%   $\mathscr{f}$ e изображение на Marchaud/Peano, ако е нетривиално, отгоре полунепрекъснато, с компактни изпъкнали образи и с линейно нарастване.
% \end{definition}

% \begin{definition}[изображение на Marchaud/Peano]
%   $\mathscr{f}$ e изображение на Marchaud/Peano, ако е нетривиално, отгоре полунепрекъснато, с компактни изпъкнали образи и с линейно нарастване.
% \end{definition}

\begin{definition}[изображение на Marchaud/Peano]
  Многозначното изображение $F:X \rightarrow Y$, $Y \subseteq \mathscr{P}(Z)$, $X, Z$ - Банахови, e изображение на Marchaud/Peano, ако:
  \begin{enumerate}
    \item $F$ е с непразни и затворени графика и домейн.
    \item Образите на $F$ са изпъкнали.
    \item $F$ има линейно нарастване, т.е. $\exists{c>0}\forall{x \in X}(\sup_{v \in F(x)} \|v\| \leq c(\|x\|+1))$
  \end{enumerate}
  % е нетривиално, отгоре полунепрекъснато, с компактни изпъкнали образи и с линейно нарастване.
\end{definition}

\begin{proposition}
За системата \eqref{eq:TheDimensionlessProblem}, $\mathscr{f}(\mathbf{z}(t))=\mathbf{f}(\mathbf{z}(t), \mathscr{U})$ е изображение на Marchaud/Peano.
\end{proposition}

% \begin{corollary}
%   Ядрото на слаба инвариантност на Белман $\eqref{eq:ViabilityKernel}$ е непразно.
% \end{corollary}

\begin{proof}
  % Стига параметрите да не са всички нулеви, то $\mathscr{f}$ е нетривиално. \\
  % От факта, че $\mathbf{f}$ е непрекъснато по всяка компонента (а даже и диференцируемо), то е и отгоре полунепрекъснато, откъдето и $\mathscr{f}$ е. \\
  % $\Omega$ е затворено и ограничено, а е крайномерно, съответно е компактно. Аналогично за $U$. Тогава и $\Omega \cross U$ е компактно. Директно от дефинициите на $\Omega$, $U$ те също така са изпъкнали, тоест $\Omega \cross U$ е изпъкнало. Вече видяхме, че $\Omega$ положително инвариантно за системата. Тогава образите на $\mathscr{f}$ ще са в $\Omega \cross U$, с други думи са компактни и изпъкнали. \\

  Стига параметрите да не са всички нулеви, то $\mathscr{f}$ е нетривиално. \\
  От факта, че $\mathbf{f}$ е непрекъснато по всяка компонента (а даже и диференцируемо), то е и отгоре полунепрекъснато, откъдето и $\mathscr{f}$ е. \\
  $\Omega$ е затворено и ограничено, а е крайномерно, съответно е компактно. Аналогично за $U$. Тогава и $\Omega \cross U$ е компактно. Директно от дефинициите на $\Omega$, $U$ те също така са изпъкнали, тоест $\Omega \cross U$ е изпъкнало. Вече видяхме, че $\Omega$ положително инвариантно за системата. Тогава образите на $\mathscr{f}$ ще са в $\Omega \cross U$, с други думи са компактни и изпъкнали. \\
\end{proof}

% \subsubsection{Линейно нарастване}
\begin{proposition}
  Системата \eqref{eq:TheDimensionlessProblem} има линейно нарастване
\end{proposition}

\begin{proof}
  Ще се използват оценките от доказателството на \eqref{cor:ExistanceAndUniqueness}.
  Понеже $\mathbf{f}(\mathbf{0}, \mathbf{u}) = \mathbf{0}$ за произволни $\mathbf{u} \in \mathscr{U}$, може да запишем:
  \begin{multline}
    \|\mathbf{f}(\mathbf{z}, \mathbf{u})\| = \|\mathbf{f}(\mathbf{z}, \mathbf{u}) - \mathbf{0}\| = \|\mathbf{f}(\mathbf{z}, \mathbf{u}) - \mathbf{f}(\mathbf{0})\| = \\
    b_{11} (2 (2 |y_1| + \kappa |u_1|) + |x_1|) + b_{12}(2  (2 |y_2| + \kappa |u_1|) +  |x_1|) + \gamma_1 |x_1| + \\
    b_{21} (2 (2|y_1| + \kappa |u_2|) + |x_2|) + b_{22} (2 (2|y_2| + \kappa |u_2|) + |x_2|) + \gamma_2 |x_2| + \\
    c_{11}(2  (2|x_1| +  \kappa |u_1|) +  |y_1|) + c_{12} (2 (2|x_2| + \kappa |u_2|) + |y_1|) + \mu_1 |y_1| + \\
    c_{21} (2  (2|x_1| + \kappa |u_1|) + |y_2|) + c_{22} (2 (2|x_2| + \kappa |u_2|) + |y_2|) + \mu_2 |y_2| \leq \\
    \tilde{C}_1|u_1| + \tilde{C}_2|u_2| + \tilde{C}_3 \|\mathbf{z}\| \leq \tilde{C}_1 + \tilde{C}_2 + \tilde{C}_3 \|\mathbf{z}\| \leq \tilde{C}_3(1 + \|\mathbf{z}\|) \\
    \tilde{C}_1 = 2 \kappa (b_{11} + b_{12} + c_{11} + c_{21}) \\
    \tilde{C}_2 = 2 \kappa (b_{21} + b_{22} + c_{12} + c_{22}) \\
    \tilde{C}_3 = 5 (b_{11} + b_{12} + b_{21} + b_{22} + c_{11} + c_{12} + c_{21} + c_{22}) + \gamma_1 + \gamma_2 + \mu_1 + \mu_2 \\
    \kappa < 1 \implies \tilde{C}_3 \geq \tilde{C}_1 + \tilde{C}_2
  \end{multline}

\end{proof}

% &\|\mathbf{f}(\mathbf{z}, \mathbf{u})\| = \|\mathbf{f}(\mathbf{z}, \mathbf{u}) - \mathbf{0}\| = \|\mathbf{f}(\mathbf{z}, \mathbf{u}) - \mathbf{f}(\mathbf{0})\| = \\
% &\frac{\beta_{vh} a_1 p_{11} e^{-\mu_1 \tau}}{p_{11} N_1 + p_{21} N_2} (2 N_1 (2 |y_1| + M_1 \kappa |u_1|) + M_1 |x_1|) + \\
% &\frac{\beta_{vh} a_2 p_{12} e^{-\mu_2 \tau}}{p_{12} N_1 + p_{22} N_2}(2 N_1 (2 |y_2| + M_2 \kappa |u_1|) + M_1 |x_1|) + \gamma_1 |x_1| + \\
% &\frac{\beta_{hv} a_1 p_{11}}{p_{11} N_1 + p_{21} N_2} (2 M_1 (2|x_1| + N_1 \kappa |u_1|) + N_1 |y_1|) + \\
% &\frac{\beta_{hv} a_1 p_{21}}{p_{11} N_1 + p_{21} N_2} (2 M_1 (2|x_2| + N_2 \kappa |u_2|) + N_2 |y_1|) + \mu_1 |y_1| + \\
% &\frac{\beta_{vh} a_1 p_{21} e^{-\mu_1 \tau}}{p_{11} N_1 + p_{21} N_2} (2 N_1 (2|y_1| + M_1 \kappa |u_1|) + M_1 |x_1|) + \\
% &\frac{\beta_{vh} a_2 p_{22} e^{-\mu_2 \tau}}{p_{12} N_1 + p_{22} N_2}(2 N_1 (2|y_2| + M_2 \kappa |u_1|) + M_1 |x_1|) + \gamma_2 |x_2| + \\
% &\frac{\beta_{hv} a_2 p_{12}}{p_{12} N_1 + p_{22} N_2} (2 M_2 (2|x_1| + N_1 \kappa |u_1|) + N_1 |y_2|) + \\
% &\frac{\beta_{hv} a_2 p_{22}}{p_{12} N_1 + p_{22} N_2} (2 M_2 (2|x_2| + N_2 \kappa |u_2|) + N_2 |y_2|) + \mu_2 |y_2| \leq \\
% &\tilde{C}_1|u_1| + \tilde{C}_2|u_2| + \tilde{C}_3 \|(x_1,y_1,x_2,y_2)\| \leq \tilde{C}_1 + \tilde{C}_2 + \tilde{C}_3 \|(x_1,y_1,x_2,y_2)\| \leq \tilde{C}(1 + \|(x_1,y_1,x_2,y_2)\|)

\begin{corollary}
  $V(\bar{\mathbf{I}}, \bar{\mathbf{u}})$ винаги съществува и е непразно
\end{corollary}

\begin{proof}
  От горното твърдение, то може да приложим $\cdots$.
  Веднага може да видим, че $V(\bar{\mathbf{I}}, \bar{\mathbf{u}}) \neq \varnothing$, понеже $\mathbf{0}$ е равновесна точка за кое да е управление и следователно $\mathbf{0} \in V(\bar{\mathbf{I}}, \bar{\mathbf{u}})$, т.е. винаги е непразно.
\end{proof}

\subsection{Числено решение на уравнението на Хамилтон-Якоби-Белман}
Задачата \eqref{eq:HJB} може да се реши с прибавено числено време.
\begin{equation}
  \label{eq:HJBTime}
  \begin{split}
    &\min\left\{\pdv{v}{t}\left(\mathbf{z}, t\right) + \lambda v(\mathbf{z}, t) + \mathcal{H}(\mathbf{z}, \grad{v}), v(\mathbf{z}, t) - \Gamma(\mathbf{z})\right\} = 0, \quad \mathbf{z} \in \mathbb{R}^4 \\
    &v(\mathbf{z}, 0) = \Gamma(\mathbf{z}), \quad \mathbf{z} \in \mathbb{R}^4 \\
    &\mathcal{H}(\mathbf{z}, \grad{v}) = \max_{\mathbf{u} \in \mathscr{U}} \innerproduct{-\mathbf{f}(\mathbf{z}, \mathbf{u})}{\grad{v}}
  \end{split}
\end{equation}
За численото пресмятане задачата \eqref{eq:HJBTime} се решава само в крайна област подмножество на $\mathbb{R}^4$, която ще включва $V(\bar{\mathbf{I}}, \bar{\mathbf{u}})$.
Всъщност от факта, че $\mathbf{z} \in \Omega \setminus \mathscr{I} \implies \mathbf{z} \notin V(\bar{\mathbf{I}}, \bar{\mathbf{u}})$, то може да бъде решена \eqref{eq:HJBTime} не за $\Omega$, а само в околност на $\mathscr{I}$.
След направени симулации с по-груба мрежа, може да се получи грубо приближение на $V(\bar{\mathbf{I}}, \bar{\mathbf{u}})$ и с по-фина мрежа вече да се решава в околност на грубото приближение.
Използвани са равномерна дискретизация по пространството и Weighted Essentially Non-Oscillatory (WENO) \cite[глава~3.4]{Osher2003} апроксимация с числен Хамилтониян от вида Lax-Friedrichs \cite[глава~5.3]{Osher2003}.
Следвайки \cite[глава~3.5]{Osher2003}, използваната дискретизацията по времето е равномерна и по него се апроксимира с подобрения метод на Ойлер.

\begin{remark}
  В литературата могат да се намерят и задачи за управление на други системи, чрез чрез решаване на уравнението на Хамилтон-Якоби-Белман, като например самолет \cite{Assellaou2016, Assellaou2018}, както и такива за рак \cite{Carrere2020}.
\end{remark}
