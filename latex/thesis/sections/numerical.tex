\section{Числено приближение на ядрото на допустимост}
\subsection{Еквивалентна задача}
\cite{Osher2003}
\subsection{WENO}
За численото пресмятане на задачата се използва дискретизация по пространството по метода Weighted Essentially Non-Oscillatory (WENO) за приближаване, което е от ред $O(h^5)$.
\cite{Osher2003}

\subsection{Дискретизация по времето}
Отново по Osher, дискретизираме по времето с подобрения метод на Ойлер, който е добре известно е от ред $O(\tau^2)$.
\cite{Osher2003}
\subsection{Симулация}
