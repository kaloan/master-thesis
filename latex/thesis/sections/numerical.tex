\section{Числено приближение на ядрото на допустимост}
\subsection{Числено решение на уравнението на Хамилтон-Якоби-Белман}
Задачата \eqref{eq:HJB} решаваме с прибавено числено време.
\begin{equation}
  \label{eq:HJBTime}
  \begin{split}
    &\pdv{v}{t}(\mathbf{z}, t) \min\left\{\lambda v(\mathbf{z}, t) + \mathcal{H}(\mathbf{z}, \grad{v}), v(\mathbf{z}, t) - \Gamma(\mathbf{z})\right\} = 0, \quad \mathbf{z} \in \mathbb{R}^4 \\
    &v(\mathbf{z}, 0) = \Gamma(\mathbf{z}), \quad \mathbf{z} \in \mathbb{R}^4 \\
    &\mathcal{H}(\mathbf{z}, \grad{v}) = \max_{\mathbf{u} \in \mathscr{U}} \innerproduct{-\mathbf{f}(\mathbf{z}, \mathbf{u})}{\grad{v}}
  \end{split}
\end{equation}
За численото пресмятане на задачата \eqref{eq:HJBTime} се решава само в околност на $\mathscr{I}$, използва се равномерна дискретизация по пространството и Weighted Essentially Non-Oscillatory (WENO) \cite[3.4]{Osher2003} апроксимация с числен хамилтониян от вида Lax-Friedrichs \cite[5.3]{Osher2003}
. Следвайки \cite[3.5]{Osher2003}, се дискретизацията по времето е равномерна и по него се апроксимира с подобрения метод на Ойлер.

\subsection{Симулация}
