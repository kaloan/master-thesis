\section{\hspace{1em} Числено приближение на ядрото на Белман}
\subsection{Числено решение на уравнението на Хамилтон-Якоби-Белман}
Задачата \eqref{eq:HJB} решаваме с прибавено числено време.
\begin{equation}
  \label{eq:HJBTime}
  \begin{split}
    &\min\left\{\pdv{v}{t}\left(\mathbf{z}, t\right) + \lambda v(\mathbf{z}, t) + \mathcal{H}(\mathbf{z}, \grad{v}), v(\mathbf{z}, t) - \Gamma(\mathbf{z})\right\} = 0, \quad \mathbf{z} \in \mathbb{R}^4 \\
    &v(\mathbf{z}, 0) = \Gamma(\mathbf{z}), \quad \mathbf{z} \in \mathbb{R}^4 \\
    &\mathcal{H}(\mathbf{z}, \grad{v}) = \max_{\mathbf{u} \in \mathscr{U}} \innerproduct{-\mathbf{f}(\mathbf{z}, \mathbf{u})}{\grad{v}}
  \end{split}
\end{equation}
За численото пресмятане задачата \eqref{eq:HJBTime} се решава само в крайна област подмножество на $\mathbb{R}^4$, която ще включва $V(\bar{\mathbf{I}}, \bar{\mathbf{u}})$.
Всъщност от факта, че $\mathbf{z} \in \Omega \setminus \mathscr{I} \implies \mathbf{z} \notin V(\bar{\mathbf{I}}, \bar{\mathbf{u}})$, то може да решим \eqref{eq:HJBTime} не за $\Omega$, а само в околност на $\mathscr{I}$.
Използвани са равномерна дискретизация по пространството и Weighted Essentially Non-Oscillatory (WENO) \cite[глава~3.4]{Osher2003} апроксимация с числен хамилтониян от вида Lax-Friedrichs \cite[глава~5.3]{Osher2003}.
Следвайки \cite[глава~3.5]{Osher2003}, използваната дискретизацията по времето е равномерна и по него се апроксимира с подобрения метод на Ойлер.

\subsection{Симулация}
Може да получим оценка на размера на $V(\bar{\mathbf{I}}, \bar{\mathbf{u}})$ като пресметнем обема на числено полученото множество от отрицателни стойности на $v$.
Тъй като дискретизацията е равномерна, да умножим броя точки от нея с 4-мерната мярка на елементарния 4-измерен паралелотоп за дискретизацията.
За референта мярка на ядрата на инвариантност може да използваме 4-мерната мярка на общото 4-мерно ядро на слаба инвариантност, което се получава в случая без мобилност.
Тогава има две независими системи, за които може да се реши задачата \eqref{eq:BasicViability} с аналогичен метод (виж \cite{Rashkov2022}) да се получи приближено решение.
В този случай ядрото на инвариантност е декартовото произведение на ядрата на инвариантности на независимите задачи, т.е. $V(\bar{\mathbf{I}}, \bar{\mathbf{u}}) = V_1(\bar{I}_1, \bar{u}_1) \times V_2(\bar{I}_2, \bar{u}_2)$.
Така 4-мерната мярка може да получим, като умножим двете 2-мерни мерки (т.е. лицата).

\begin{table}
  \centering
  \begin{tabular}{ | c| c c c c|}
    \hline
    \backslashbox{$p_{22}$}{$p_{11}$}& 0.8 & 0.85 & 0.9 & 0.95 \\
    \hline
    0.95 & 3.427 & 3.447 & 3.467 & 3.486\\
    0.9 & 3.468 & 3.487 & 3.507 & 3.527\\
    0.85 & 3.498 & 3.517 & 3.536 & 3.554\\
    0.8 & 3.519 & 3.540 & 3.559 & 3.580\\
    \hline
  \end{tabular}
  \caption{table}{4-мерната мярка на ядрото на слаба инвариантност на Белман $V(\bar{\mathbf{I}}, \bar{\mathbf{u}})$ за различни стойности на мобилността (спрямо случая без мобилност)}
  \label{tbl:ViabilityKernel-poster}
\end{table}

Трябва да се отбележи, че може да има начални условия, които за някои $p_{11}, p_{22}$ да са в ядрото на слаба инвариантност, но за други да не са.
Затова дори и ядрото да е с по-малък обем за някои стойности отколкото за други, това не непременно значи, че то е подмножество на другото.
За да може да правим такива разсъждения, трябва промените по десните страни на \eqref{eq:TheDimensionlessProblem} да водят до сравнения по дефиниция \ref{def:FunctionLEQ} и съответно да може да ползваме теоремата \ref{thm:Comparison}.
Така например, ако фиксираме всички параметри и меним $\kappa$ или пък $\bar{\mathbf{u}}$ с по-големи стойности, то и ще получим надмножествено ядро на слаба инвариантност и съответно сравнението на 4-мерните мерки на такива ядра на слаба инвариантност е обективна оценка за ефективността на подобрението на репелента или здравната политика.
