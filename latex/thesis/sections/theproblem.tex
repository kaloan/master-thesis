\section{\hspace{1em} Модел на Ross-Macdonald с две местообитания и репелент}
Задачата която се изследва в дипломната работа е:
\begin{equation}
  \label{eq:TheProblem}
  \begin{split}
    &\dot{X}_1(t) = \beta_{vh} (N_1-X_1(t)) \left(\frac{p_{11} e^{-\mu_1 \tau} a_1 (1-\kappa u_1(t)) Y_1(t)}{p_{11} N_1 + p_{21} N_2} + \frac{p_{12} e^{-\mu_2 \tau} a_2 (1-\kappa u_1(t)) Y_2(t)}{p_{12} N_1 + p_{22} N_2}\right) - \gamma_1 X_1(t) \\
    &\dot{X}_2(t) = \beta_{vh} (N_2-X_2(t)) \left(\frac{p_{21} e^{-\mu_1 \tau} a_1 (1-\kappa u_2(t)) Y_1(t)}{p_{11} N_1 + p_{21} N_2} + \frac{p_{22} e^{-\mu_2 \tau} a_2 (1-\kappa u_2(t)) Y_2(t)}{p_{12} N_1 + p_{22} N_2}\right) - \gamma_2 X_2(t) \\
    &\dot{Y}_1(t) = \beta_{hv} a_1 (M_1-Y_1(t)) \frac{p_{11} (1-\kappa u_1(t)) X_1(t) + p_{21} (1-\kappa u_2(t)) X_2(t)}{p_{11} N_1 + p_{21} N_2} - \mu_1 Y_1(t) \\
    &\dot{Y}_2(t) = \beta_{hv} a_2 (M_2-Y_2(t)) \frac{p_{12} (1-\kappa u_1(t)) X_1(t) + p_{22} (1-\kappa u_2(t)) X_2(t)}{p_{12} N_1 + p_{22} N_2} - \mu_2 Y_2(t) \\
    &u_i(t) \in \mathscr{U}_i = \{u_i:\mathbb{R}_+ \rightarrow [0, \bar{u}_i] \vert u_i \text{- измерима}\}, i=1,2, \quad \mathscr{U} = \mathscr{U}_1 \cross \mathscr{U}_2
  \end{split}
\end{equation}

Това е модел обединение на моделите за мобилност \eqref{eq:MigrationProblem} и за репелент против комари \eqref{eq:RepellentProblem} и отново се запазват означенията от дефиниция \ref{def:Parameters}, както на другите въведени в тези модели.
Таблична форма на всичките параметри е поставена долу в таблица \ref{tbl:ParameterDefinitions}. \\
Трябва да се отбележи, че в модела \eqref{eq:TheProblem} се допуска, че хората са лично защитени от репелента (например носят дрехи с него), затова са предпазени от репелента спрямо местообитанието, чиито жители са.
Този тип моделиране е съобразен с нетрайния характер на пребиваване характерен за Лагранжевите модели.
% $t$ е времето, като ще разглеждаме само $t \in [0, \infty)$. \\
% $X_i \in [0, N_i]$ са броят заразени хора, а $Y_i \in [0, M_i]$ - броят заразени комари в местообитания $i=1,2$. \\
% % Бележим $\Omega = \{0 \leq X_1 \leq N_1, 0 \leq Y_1 \leq M_1, 0 \leq X_2 \leq N_2, 0 \leq Y_2 \leq M_2\}$
% $u_i :\mathbb{R}_+ \rightarrow [0, \bar{u}_i]$ са измерими функции управления, отговарящи за това каква част от хората от съответното местообитание са предпазени от репелента, като $\bar{u}_i \leq 1 $ отговарят за максималната предпазена част от населението, вследствие от производствената способност. Надолу се бележи $U = [0, \bar{u}_1] \cross [0, \bar{u}_2]$ и $\mathbf{u} = (u_1, u_2), \bar{\mathbf{u}} = (\bar{u}_1, \bar{u}_2)$ \\
% $\kappa \in [0, 1]$ е константа, която представя ефективността на репелента (т.е. предполага се че едно и също вещество/метод се използва и в двете местообитания и има еднаква ефективност на отблъскване на комарите от двете местообитания). \\
% $p_{11}, p_{12}, p_{21}, p_{22} \in [0, 0.5]$, $p_{11} + p_{12} = 1, p_{21} + p_{22} = 1$ са константи, отговарящи за мобилността, като $p_{ij}$ е частта от хора от $i$, които пребивават временно в $j$. \\
% $\gamma_1, \gamma_2$ са скоростите на оздравяване на хората, а $\mu_1, \mu_2$ - скоростите на смъртност на комарите, които приемаме за константи. \\
% $\tau$ е константа на средното време, за което комарите стават заразни. \\
% $\alpha_1, \alpha_2$ са константи, които бележат колко ухапвания на комари има за единица време. \\
% $\beta_{vh}$ е константна вероятност за заразяване на здрав човек с патогена, когато бъде ухапан от заразен комар, а $\beta_{hv}$ е константна вероятност за заразяване на здрав комар с патогена, когато ухапе заразен човек.

\begin{table}[h]
  \centering
  \begin{tabular}{ |c||c| }
    \hline
    Параметър & Описание\\
    \hline
    $\beta_{vh}$ & Вероятност на прехвърляне на патогена от комар на човек \\
    $\beta_{hv}$ & Вероятност на прехвърляне на патогена от човек на комар \\
    $a_i$ & Честота на ухапвания [$\text{ден}^{-1}$]\\
    $M_i$ & Популация на женски комари\\
    $\mu_i$ & Смъртност на комари [$\text{ден}^{-1}$]\\
    $\tau$ & Инкубационен период при комарите [$\text{ден}$]\\
    $N_i$ & Човешка популация (население) \\
    $\gamma_i$ & Скорост на оздравяване на хора [$\text{ден}^{-1}$]\\
    $p_{ij}$ & Мобилност на хора от местообитание i в j\\
    $\kappa$ & Ефективност на репелент\\
    $\bar{u}_i$ & Максимална възможна предпазена част жители с репелент\\
    $\bar{I}_i$ & Максимална част на заразени хора\\
    \hline
  \end{tabular}
  \caption{Таблица с параметри на модела \eqref{eq:TheProblem}}
  \label{tbl:ParameterDefinitions}
  % (таблична форма на деф. \ref{def:Parameters})}
\end{table}


% \begin{table}[h]
%   \centering
%   \caption{Таблица с параметри на модела}
%   \begin{tabular}{ |c c c|  }
%     \hline
%     Параметър & Описание & Мерни единици\\
%     \hline
%     $\beta_{hv}$ & Вероятност на прехвърляне на патогена от човек на комар & безразмерен\\
%     $\beta_{vh}$ & Вероятност на прехвърляне на патогена от комар на човек & безразмерен\\
%     $a_i$ & Честота на ухапвания & $\text{ден}^{-1}$\\
%     $M_i$ & Популация на женски комари & безразмерен\\
%     $\mu_i$ & Смъртност на комари & $\text{ден}^{-1}$\\
%     $\tau$ & Инкубационен период при комарите & ден\\
%     $N_i$ & Човешка популация & безразмерен\\
%     $\gamma_i$ & Скорост на оздравяване на хора & $\text{ден}^{-1}$\\
%     $p_{ij}$ & Мобилност на хора от местообитание i в j & безразмерен\\
%     $\kappa$ & Ефективност на репелент & безразмерен\\
%     $\bar{u}_i$ & Максимална възможна предпазена част хора с репелент & безразмерен\\
%     $\bar{I}_i$ & Максимална част на заразени хора & безразмерен\\
%     \hline
%   \end{tabular}
%   \label{table:1}
% \end{table}


Моделът подлежи на скалиране на променливите чрез смяната:
\begin{equation}
  (X_1, X_2, Y_1, Y_2)^T \rightarrow (\frac{X_1}{N_1}, \frac{X_2}{N_2}, \frac{Y_1}{M_1}, \frac{Y_2}{M_2})^T = (x_1, x_2, y_1, y_2)^T
\end{equation}
Тоест така се разглежда дяла от жители и комари, които са заразени, вместо техния брой, а за допълнително опростяване на изразите може да бъдат направени следните полагания:
% След това достигаме до:
% \begin{equation}
%   \label{eq:TheDimensionlessProblemFull}
%   \begin{split}
%     &\dot{X}_1(t) = \beta_{vh} (1-X_1(t)) \left(\frac{p_{11} e^{-\mu_1 \tau} a_1 (1-\kappa u_1(t)) M_1 Y_1(t)}{p_{11} N_1 + p_{21} N_2} + \frac{p_{12} e^{-\mu_2 \tau} a_2 (1-\kappa u_1(t)) M_2 Y_2(t)}{p_{12} N_1 + p_{22} N_2}\right) - \gamma_1 X_1(t) \\
%     &\dot{Y}_1(t) = \beta_{hv} a_1 (1-Y_1(t)) \frac{p_{11} (1-\kappa u_1(t)) N_1 X_1(t) + p_{21} (1-\kappa u_2(t)) N_2 X_2(t)}{p_{11} N_1 + p_{21} N_2} - \mu_1 Y_1(t) \\
%     &\dot{X}_2(t) = \beta_{vh} (1-X_2(t)) \left(\frac{p_{21} e^{-\mu_1 \tau} a_1 (1-\kappa u_2(t)) M_1 Y_1(t)}{p_{11} N_1 + p_{21} N_2} + \frac{p_{22} e^{-\mu_2 \tau} a_2 (1-\kappa u_2(t)) M_2 Y_2(t)}{p_{12} N_1 + p_{22} N_2}\right) - \gamma_2 X_2(t) \\
%     &\dot{Y}_2(t) = \beta_{hv} a_2 (1-Y_2(t)) \frac{p_{12} (1-\kappa u_1(t)) N_1 X_1(t) + p_{22} (1-\kappa u_2(t)) N_2 X_2(t)}{p_{12} N_1 + p_{22} N_2} - \mu_2 Y_2(t)
%   \end{split}
% \end{equation}

% В този модел за улеснение на по-нататъшни изрази може да направим полагането:
\begin{equation}
  \begin{split}
    &b_{11} = \beta_{vh} \frac{p_{11} e^{-\mu_1 \tau} a_1 M_1}{p_{11} N_1 + p_{21} N_2} \geq 0 \quad
    b_{12} = \beta_{vh} \frac{p_{12} e^{-\mu_2 \tau} a_2 M_2}{p_{12} N_1 + p_{22} N_2} \geq 0 \\
    &b_{21} = \beta_{vh} \frac{p_{21} e^{-\mu_1 \tau} a_1 M_1}{p_{11} N_1 + p_{21} N_2} \geq 0 \quad
    b_{22} = \beta_{vh} \frac{p_{22} e^{-\mu_2 \tau} a_2 M_2}{p_{12} N_1 + p_{22} N_2} \geq 0 \\
    &c_{11} = \beta_{hv} a_1 \frac{p_{11} N_1}{p_{11} N_1 + p_{21} N_2} \geq 0 \quad
    c_{12} = \beta_{hv} a_1 \frac{p_{21} N_2}{p_{11} N_1 + p_{21} N_2} \geq 0 \\
    &c_{21} = \beta_{hv} a_2 \frac{p_{12} N_1}{p_{12} N_1 + p_{22} N_2} \geq 0 \quad
    c_{22} = \beta_{hv} a_2 \frac{p_{22} N_2}{p_{12} N_1 + p_{22} N_2} \geq 0 \\
  \end{split}
\end{equation}

Крайният вид на модела е:
\begin{equation}
  \label{eq:TheDimensionlessProblem}
  \begin{split}
    &\dot{x}_1(t) = (1-x_1(t)) (1-\kappa u_1(t)) \left(b_{11} y_1(t) + b_{12} y_2(t)\right) - \gamma_1 x_1(t) \\
    &\dot{x}_2(t) = (1-x_2(t)) (1-\kappa u_2(t))\left(b_{21} y_1(t) + b_{22} y_2(t)\right) - \gamma_2 x_2(t) \\
    &\dot{y}_1(t) = (1-y_1(t)) \left(c_{11}(1-\kappa u_1(t)) x_1(t) + c_{12}(1-\kappa u_2(t)) x_2(t)\right) - \mu_1 y_1(t) \\
    &\dot{y}_2(t) = (1-y_2(t)) \left(c_{21}(1-\kappa u_1(t)) x_1(t) + c_{22} (1-\kappa u_2(t)) x_2(t)\right) - \mu_2 y_2(t) \\
    &u_i(t) \in \mathscr{U}_i = \{u_i:\mathbb{R}_+ \rightarrow [0, \bar{u}_i] \vert u_i \text{- измерима}\}, i=1,2, \quad \mathscr{U} = \mathscr{U}_1 \cross \mathscr{U}_2
  \end{split}
\end{equation}

Надолу \eqref{eq:TheDimensionlessProblem} ще се записва и във векторен вид по следния начин:
\begin{equation}
  \begin{pmatrix}
    \dot{\mathbf{x}} \\
    \dot{\mathbf{y}}
  \end{pmatrix}
  =
  \mathbf{f}(\mathbf{x}, \mathbf{y}, \mathbf{u}), \quad
  \mathbf{x} = (x_1, x_2)^T, \quad \mathbf{y} = (y_1, y_2)^T, \quad \mathbf{f}=(f_{x_1}, f_{x_2}, f_{y_1}, f_{y_2})^T
\end{equation}
Или пък във вида:
\begin{equation}
  \dot{\mathbf{z}} = \mathbf{f}(\mathbf{z}, \mathbf{u}), \quad \mathbf{z} = (\mathbf{x}, \mathbf{y})^T
\end{equation}
Бележим $\Omega = \{0 \leq x_1 \leq 1, 0 \leq x_2 \leq 1, 0 \leq y_1 \leq 1, 0 \leq y_2 \leq 1\} = \{\mathbf{z} \in [0, 1]^4\}$.
Началното условие $\mathbf{z}(0)$ ще се записва във вида $\mathbf{z}(0) = \mathbf{z}_0 = (x_1^0, x_2^0, y_1^0, y_2^0)^T \in \Omega$.

Нека $\bar{\mathbf{I}} = (\bar{I}_1, \bar{I}_2)^T, \bar{I}_1, \bar{I}_2 \in [0, 1]$ са константи, отговарящи за максималната част от населението в съответното местообитание, което може да получи адекватна здравна помощ при заразяване с малария.
Нека се означи $\mathscr{I} = [0, \bar{I}_1] \times [0, \bar{I}_2] \times [0, 1] \times [0, 1]$.
Питаме се има ли такива управления $\mathbf{u}(t)$, за които във всеки момент всички заразени да имат възможност да получат помощ от здравната система, т.е. :
\begin{equation}
  \label{eq:AllHospitalised}
  \forall t \geq 0 (x_1(t) \leq \bar{I}_1 \wedge x_2(t) \leq \bar{I}_2) \iff \forall t \geq 0 (\mathbf{x}(t) \in [0, \bar{I}_1] \times [0, \bar{I}_2]) \iff \forall t \geq 0 (\mathbf{z}(t) \in \mathscr{I})
\end{equation}
Тъй като първоначалният брой заразени хора и комари влияят на развитието на системата ще търсим множеството на начални условия, за които всички траектории на \eqref{eq:TheDimensionlessProblem} излълняват \eqref{eq:AllHospitalised}:
\begin{equation}
  \label{eq:ViabilityKernel}
  V(\bar{\mathbf{I}}, \bar{\mathbf{u}}) = \{\mathbf{z}_0 \text{ начално условие} \vert \exists \mathbf{u} (\eqref{eq:TheDimensionlessProblem} \text{ има решение} \wedge \eqref{eq:AllHospitalised} \text{ е изпълнено})\}
\end{equation}
$V(\bar{\mathbf{I}}, \bar{\mathbf{u}})$ се нарича ядро на слаба инвариантност на Белман.
Веднага може да видим, че $V(\bar{\mathbf{I}}, \bar{\mathbf{u}}) \neq \varnothing$, понеже $\mathbf{0}$ е равновесна точка за кое да е управление и следователно $\mathbf{0} \in V(\bar{\mathbf{I}}, \bar{\mathbf{u}})$, т.е. винаги е непразно.

За намиране на ядрото на Белман $\eqref{eq:ViabilityKernel}$ се подхожда по метода от \cite{Zidani2013}. Въвежда се значна функция на разстоянието до границата на $\mathscr{I}$:
\begin{equation}
  \Gamma(\mathbf{z}) =
  \begin{cases}
    \min\limits_{\mathbf{z}' \in \mathscr{I}} |\mathbf{z}-\mathbf{z}'|, \quad \mathbf{z} \in \Omega \setminus \mathscr{I} \\
    -\min\limits_{\mathbf{z}' \in \Omega \setminus \mathscr{I}} |\mathbf{z}-\mathbf{z}'|, \quad \mathbf{z} \in \mathscr{I}
  \end{cases}
\end{equation}

Нека сега $\lambda > L > 0$, където $L$ е константата на Липшиц (горна оценка за нея е получена в \eqref{eq:LipschitzContinuity}). Въвеждаме функцията на Белман $v$:
\begin{equation}
  v(\mathbf{z}_0) = \inf_{\mathbf{u} \in \mathscr{U}} \sup_{t \in (0, +\infty)} e^{-\lambda t} \Gamma(\mathbf{z}(t; \mathbf{z}_0; \mathbf{u}))
\end{equation}

Тук с $\mathbf{z}(t; \mathbf{z}_0; \mathbf{u})$ е означено решението на \eqref{eq:TheDimensionlessProblem} в момент $t$ при начално условие $\mathbf{z}_0$ и управление $\mathbf{u}$. Така е в сила:
\begin{equation}
  V(\bar{\mathbf{I}}, \bar{\mathbf{u}}) = \{\mathbf{z}_0 \in \Omega \vert v(\mathbf{z}_0) \leq 0 \}, \quad \partial V(\bar{\mathbf{I}}, \bar{\mathbf{u}}) = \{\mathbf{z}_0 \in \Omega \vert v(\mathbf{z}_0) = 0\}
\end{equation}

Изпълнен е принцип за динамично програмиране:
\begin{equation}
  \forall{t>0} \left(v(\mathbf{z}_0) = \inf_{\mathbf{u} \in \mathscr{U}} \max\{e^{-\lambda t} v(\mathbf{z}_0), \sup_{s \in (0, t]} e^{-\lambda t} \Gamma(\mathbf{z}(s; \mathbf{z}_0; \mathbf{u}))\}\right)
\end{equation}

Използвайки стандартни аргументи от \cite{Bardi1997} се установявва, че $v$ e вискозното решение на уравнение на Хамилтон-Якоби-Белман за минимизация на функционал:
\begin{equation}
  \label{eq:HJB}
  \begin{split}
    &\min\left\{\lambda v(\mathbf{z}) + \mathcal{H}(\mathbf{z}, \grad{v}), v(\mathbf{z}) - \Gamma(\mathbf{z})\right\} = 0, \quad \mathbf{z} \in \mathbb{R}^4 \\
    &\mathcal{H}(\mathbf{z}, \mathbf{w}) = \max_{\mathbf{u} \in \mathscr{U}} \innerproduct{-\mathbf{f}(\mathbf{z}, \mathbf{u})}{\mathbf{w}}
  \end{split}
\end{equation}

Използвайки вида на $\mathbf{f}$, след групиране по части зависещи/независещи от управлението в \eqref{eq:HJB}, получаваме:
% \begin{multline}
%   \label{eq:HamiltonianFull}
%   \mathcal{H}(\mathbf{z}, \grad{v}) = \\
%   \left[\gamma_1 x_1(t) - (1-x_1(t)) \left(b_{11} y_1(t) + b_{12} y_2(t)\right) \right] \pdv{v}{x_1} +
%   \left[\gamma_2 x_2(t) - (1-x_2(t)) \left(b_{21} y_1(t) + b_{22} y_2(t)\right) \right] \pdv{v}{x_2} + \\
%   \left[\mu_1 y_1(t) - (1-y_1(t)) \left(c_{11} x_1(t) + c_{12} x_2(t)\right) x\right] \pdv{v}{y_1} +
%   \left[\mu_2 y_2(t) - (1-y_2(t)) \left(c_{21} x_1(t) + c_{22} x_2(t)\right)\right] \pdv{v}{y_2} + \\
%   \max_{\mathbf{u} \in \mathscr{U}} \bigg\{\kappa u_1(1-x_1(t)) \left(b_{11} y_1(t) + b_{12} y_2(t)\right) \pdv{v}{x_1} +
%     \kappa u_2 (1-x_2(t)) \left(b_{21} y_1(t) + b_{22} y_2(t)\right) \pdv{v}{x_2} + \\
%     (1-y_1(t)) \kappa \left(c_{11} x_1(t) u_1 + c_{12} x_2(t) u_2 \right) \pdv{v}{y_1}+
%   (1-y_2(t)) \kappa \left(c_{21} x_1(t) u_1 + c_{22} x_2(t) u_2 \right) \pdv{v}{y_2}\bigg\}
% \end{multline}

\begin{multline}
  \label{eq:HamiltonianFull}
  \mathcal{H}(\mathbf{z}, \grad{v}) = \\
  \left[\gamma_1 x_1 - (1-x_1) \left(b_{11} y_1 + b_{12} y_2\right) \right] \pdv{v}{x_1} +
  \left[\gamma_2 x_2 - (1-x_2) \left(b_{21} y_1 + b_{22} y_2\right) \right] \pdv{v}{x_2} + \\
  \left[\mu_1 y_1 - (1-y_1) \left(c_{11} x_1 + c_{12} x_2\right) x\right] \pdv{v}{y_1} +
  \left[\mu_2 y_2 - (1-y_2) \left(c_{21} x_1 + c_{22} x_2\right)\right] \pdv{v}{y_2} + \\
  \max_{\mathbf{u} \in \mathscr{U}} \bigg\{\kappa u_1(1-x_1) \left(b_{11} y_1 + b_{12} y_2\right) \pdv{v}{x_1} +
    \kappa u_2 (1-x_2) \left(b_{21} y_1 + b_{22} y_2\right) \pdv{v}{x_2} + \\
    (1-y_1) \kappa \left(c_{11} x_1 u_1 + c_{12} x_2 u_2 \right) \pdv{v}{y_1}+
  (1-y_2) \kappa \left(c_{21} x_1 u_1 + c_{22} x_2 u_2 \right) \pdv{v}{y_2}\bigg\}
\end{multline}

В максимума има членове зависещи само от $u_1$ и $u_2$, така че може да разбием на сума от два максимума по всяко от управленията. Функциите са линейни спрямо управленията и съответно максимумите ще се достигат в един от двата края на допустимите интервали. Крайният вид е:
% \begin{multline}
%   \label{eq:HamiltonianShort}
%   \mathcal{H}(\mathbf{z}, \grad{v}) = \\
%   \left[\gamma_1 x_1(t) - (1-x_1(t)) \left(b_{11} y_1(t) + b_{12} y_2(t)\right) \right] \pdv{v}{x_1} +
%   \left[\gamma_2 x_2(t) - (1-x_2(t)) \left(b_{21} y_1(t) + b_{22} y_2(t)\right) \right] \pdv{v}{x_2} + \\
%   \left[\mu_1 y_1(t) - (1-y_1(t)) \left(c_{11} x_1(t) + c_{12} x_2(t)\right) x\right] \pdv{v}{y_1} +
%   \left[\mu_2 y_2(t) - (1-y_2(t)) \left(c_{21} x_1(t) + c_{22} x_2(t)\right)\right] \pdv{v}{y_2} + \\
%   \max\bigg\{0, \kappa \bar{u}_1 (1-x_1(t)) \left(b_{11} y_1(t) + b_{12} y_2(t)\right) \pdv{v}{x_1} + c_{11} \kappa \bar{u}_1 x_1(t) (1-y_1(t))\pdv{v}{y_1} + c_{21} \bar{u}_1 x_1(t) (1-y_2(t)) \pdv{v}{y_2}
%   \bigg\} + \\
%   \max\bigg\{0, \kappa \bar{u}_2 (1-x_2(t)) \left(b_{21} y_1(t) + b_{22} y_2(t)\right) \pdv{v}{x_2} + c_{12} \bar{u}_2 x_2(t) (1-y_1(t))\pdv{v}{y_1} + c_{22}  \bar{u}_2 x_2(t) (1-y_2(t))\pdv{v}{y_2} \bigg\}
% \end{multline}

\begin{multline}
  \label{eq:HamiltonianShort}
  \mathcal{H}(\mathbf{z}, \grad{v}) = \\
  \left[\gamma_1 x_1 - (1-x_1) \left(b_{11} y_1 + b_{12} y_2\right) \right] \pdv{v}{x_1} +
  \left[\gamma_2 x_2 - (1-x_2) \left(b_{21} y_1 + b_{22} y_2\right) \right] \pdv{v}{x_2} + \\
  \left[\mu_1 y_1 - (1-y_1) \left(c_{11} x_1 + c_{12} x_2\right) x\right] \pdv{v}{y_1} +
  \left[\mu_2 y_2 - (1-y_2) \left(c_{21} x_1 + c_{22} x_2\right)\right] \pdv{v}{y_2} + \\
  \max\bigg\{0, \kappa \bar{u}_1 (1-x_1) \left(b_{11} y_1 + b_{12} y_2\right) \pdv{v}{x_1} + c_{11} \kappa \bar{u}_1 x_1 (1-y_1)\pdv{v}{y_1} + c_{21} \bar{u}_1 x_1 (1-y_2) \pdv{v}{y_2}
  \bigg\} + \\
  \max\bigg\{0, \kappa \bar{u}_2 (1-x_2) \left(b_{21} y_1 + b_{22} y_2\right) \pdv{v}{x_2} + c_{12} \bar{u}_2 x_2 (1-y_1)\pdv{v}{y_1} + c_{22}  \bar{u}_2 x_2 (1-y_2)\pdv{v}{y_2} \bigg\}
\end{multline}
